{\let\clearpage\relax\let\cleardoublepage\relax
\chapter{Modello formazione planetaria nello schema di core accretion}
}

\section{Core accretion o instabilit\'a gravitazionale?}

\begin{workout}[Refs GI vs CA]
planetesimal hypoithesis: chamberlin 05, Safronov 69, Hayashi 77. Formation on dynamical scale via GI: Kuiper 51, Cameron 62.
\end{workout}


Il modello di formazione planetaria da me considerato, core accretion (CA) vedi \ref{part:CAdesc}, si basa sulla sedimentazione della componente polverosa nel disco protoplanetario in embrioni planetari che, se abbastanza massicci sono in grado di accrescere il gas presente nel disco protoplanetario.

L'altro schema di formazione considera l'instabilit\'a gravitazionale del disco. 

Esistono alcuni dati osservativi in favore del primo modello, considerando il sistema solare: arricchimento dei pianeti del sistema solare rispetto a composizione solare di metalli (la composizione di Urano/Nettuno contiene $1\%$ di $H_2$, $He$ contenuto in pianeta di composizione solare con pari massa di metalli), ipotizzando una rimozione della parte gassosa il pianeta deve avere il tempo di sedimentare la componente metllica, \'e necessario un disco di massa solare e i pianeti si formerebbero a distanza maggiore, non spiega la formazione dei corpi minori.

\chapter{Modello disco di accrescimento}

\begin{workout}[Refs dischi protoplanetari]
Ciesla Dullemond10 
\end{workout}

\begin{workout}[Classificazione YSO: convenzioni]
Sorgenti con SED declinante in mid infrared ($2.5-10\si{\micro\meter}$): $1.5<\alpha_{IR}=<0$. Active/passive: mass infall convert G into thermalradiation/reprocessed starlight.
\end{workout}

\begin{workout}[Viscous/turbulent disk]
laminar (high momentum diffusion)
mass inflow Gullbring98 \SIrange{e-9}{e-7}{\per\year}$\msun{}$
\end{workout}


\begin{workout}[MRI]
fig 10.3:
gammie 1996
\end{workout}


\begin{workout}[Accretion disk sources]
o218, 296: disk formation, constrains from solar system
the alpha disk pg 18: theory of turbulent accretion disk
\end{workout}

\begin{workout}[Isothermal cloud collapse!]

\end{workout}


\begin{workout}[Intro a disco accrescimento]
Nei modelli globali la formazione planetaria \'e simulata partendo dalla fase finale di accrescimento di massa sulla stella centrale (per semplicit\'a considero stelle di $1\msun{}$ singole): il collasso di nube molecolare produce strutture appiattite la cui evoluzione \'e determinata dal trasporto di momento angolare verso l'esterno, l'interazione con l'oggetto centrale (campi magnetici vento stellare) e con l'ambiente circostante.
L'ipotesi su cui si basano le simulazioni considerata \'e che la componente polverosa formi corpi pi\u massicci fino a masse di frazioni di masse terrestri  ed infine accrescere gas: lo scenario di core accretion (CA).
\end{workout}

\begin{workout}[Modelli 1D disco accrescimento: struttura verticale]
auto gravitazione trascurabile quindi la struttura verticale \'e determinata dalla componente lungo z dell'attrazione del corpo centrale, profilo termico determinato da equilibrio.
\end{workout}

\begin{workout}[Descrizione trasporto momento angolare tramite parametrizzazione viscosit\'a]
Mass conservation + momentum conservation in viscous flow: angular momentum evolution. Phenomena: Shear viscosity - turbulence - MRI.
Nei modelli 1D per disco di accrescimento si parametrizza viscosit\'a tramite $\nu=\frac{\eta}{\rho}\to\alpha c_s H$: la viscosit\'a molecolare \'e troppo bassa  per trasportare all'esterno il momento angolare sui tempi-scala osservati.

Theory of turbulent accretion disk: pg 18, 8.
Flusso di x-momentum lungo y $\rho\exv{u_xu_y}$, dove considero la velocit\'a dell'elemento di fluido $(v_x+u_x,u_y)$ dove le $u$ rappresentano fluttuazioni della velocit\'a: $\sigma_{xy}=-\rho\exv{u_xu_y}$.
Ricordando la definizione di tensore degli stress e con $\vec{v}=v_x(y)\hat{x}$

\begin{equation}
\sigma_{ij}=\eta(\partial_jv_i+\partial_iv_j-\frac{2}{3}\delta_{ij}\partial_kv_k+\zeta\delta_{ij}\partial_kv_k \to \eta\TDy{y}{v_x}
\end{equation}

\begin{equation}
\PDof{t}(\rho v_i)+\PDof{x_i}(\rho v_iv_j+T_{ij})=F_i
\end{equation}
Fluttuazioni: $\sigma_{ij}=-\rho\exv{u_iu_j}$.
Viscosit\'a molecolare: $\exv{u_iu_j}=-\nu\TDy{x_j}{v_i}$.
Turbulence: Reynpold stress $\tau_{ij}=-\rho\exv{v_jv_i}$, $\exv{u_iu_j}=-\nu_T\TDy{x_j}{v_i}$
\begin{equation}
\PDy{t}{\Sigma}=3\frac{1}{r}\PDof{r}[r\expy{1/2}\PDof{r}(\nu\Sigma r\expy{1/2})]\label{eq:sigmaevol}
\end{equation}
\end{workout}

\begin{workout}[Photoievaporation: X-EUV-FUV (Alexander13: The Dispersal of Protoplanetary Disks)]
Protoplanetary disc evolution and dispersal: the implications of X-ray photoevaporation
\end{workout}

\begin{workout}[alpha prescription]
$\vec{v}=(u_r,r\Omega)$, stress tensor $\sigma_{r\phi}=-\Sigma\exv{u_ru_{\phi}}$.
Enhanced turbulent viscosity: $-\Sigma\exv{u_ru_{\phi}}=\Sigma\nu r\TDy{r}{\Omega}$, $\nu=v_TH$, $\alpha=v_T/c_s$
\end{workout}

\begin{workout}[Modello di disco di accrescimento - (Mordasini18: 4) - Introduzione descrizione fenomeni grazie a PPS]
Un modello di disco  di accrescimento usato nelle simulazioni considera l'evoluzione della densit\'a superficiale tramite l'equazione \eqref{eq:diskaccrphev-m18}

\begin{align}
&\TDy{t}{\Sigma}=\frac{1}{r}\PDof{r}[3r\expy{1/2}\PDof{r}(\nu\Sigma r\expy{1/2})]+\dot{\Sigma}_w(r)+\dot{\Sigma}_p(r)\label{eq:diskaccrphev-m18}\\
&\dot{\Sigma}_w(a)=\left\{\begin{array}{c}0\\\frac{\dot{M}_w}{2\pi(a_{max}-R_g)a}\\\end{array}\right.
\end{align}

Densit\'a superficiale iniziale:
\begin{equation}
\Sigma(a,t=0)=\Sigma_0(\frac{r}{1AU})\expy{p_g}\Exp{[-(\frac{r}{R_o})\expy{2+p_g}]}(1-\sqrt{\frac{r}{R_i}})
\end{equation}
4-Mordasini18 (Hayashi81). $p_g\approx1$ (Andrews10).
\end{workout}


\begin{workout}[Foto-evaporazione]
(Photoevaporation: veras armitage 2003, Alexander13, Mordasini12. Internal/External).
EUV($E\approx13.6eV$): ionization, FUV($E\approx6-13.6eV$): dissociation, X-ray
\end{workout}

\section{Formazione planetesimi}

Protoplanetary disk and their evolution pg 29
Protoplanetary dust: Apai Lauretta - particles dynamics pg 100

Assumendo conversione completa di polvere in planetesimi, la densit\'a superficiale di planetesimi \'e
\begin{equation}
\Sigma_p(t=0,r)=f_{dg}\eta_{ice}\Sigma_g(t=0,r)
\end{equation}

\begin{workout}[Particle-gas dynamics. dust midplane sedimantation]
Weidenschilling cuzzi 06: Particle-gas dynamics and primary accretion, Apai Lauretta Protoplanetary Dust pg 100
Stopping time $t_s=\frac{\rho_sa}{\rho c_s}$ ($a<\lambda_g$). L'evoluzione dinamica delle particelle solide \'e determinata flussi macroscopici dovuti all'evoluzione del disco, gas-drag, settling
\end{workout}

\begin{workout}[Effectsof protplanets atmosphere]
Enhanced collisional growth of a protoplanet that has an atmosphere (Inaba Ikoma 03)
\end{workout}

\section{Accrescimento pianetsemi: formazione proto-pianeti.}

La massa dei core varia secondo
\begin{align}
&\dot{M}_c=\Omega\Sigma_pR^2_{capt}F_G\\
&\dot{\Sigma}_p(r)=-\frac{1}{2\pi aB_LR_H}\dot{M}_c
\end{align}

\begin{workout}[feeding zone]
$B_LR_H$
\end{workout}

\begin{workout}[runaway to oligarchic]
 NEW CONDITION FOR THE TRANSITION FROM RUNAWAY TO OLIGARCHIC GROWTH (Ormel 10)
 \end{workout}

\begin{workout}[Accrescimento pianetesimi: orderly, runaway, oligarchic]
The growth of planetary embryos:  orderly, runaway, or oligarchic? (Rafikov 2002)
\begin{align}
\TDy{t}{M_e}\approx\pi R_e^2\Omega mN\frac{v}{v_z}[1+\frac{2GM_e}{R_ev^2}]
\end{align}
Accrescimento ordinato: $\frac{1}{M_e}\TDy{t}{M_e}\propto M_e\expy{-1/3}$. Accrescimento runaway: $\frac{1}{M_e}\TDy{t}{M_e}\propto M_e\expy{1/3}$. Accrescimento oligarchico: $\frac{1}{M_e}\TDy{t}{M_e}\propto M_e\expy{-1/3}$.
\end{workout}

\begin{workout}[Timescale of oligarchic growth]
kokubo ida 02 (ida lin 04)
\begin{align}
&\dot{M}_c=\frac{M_c}{\tau_c}\\
&\tau_c=\SI{1.2e5}{\year}(\frac{\Sigma_p}{\SI{10}{\gram\per\cubic\cm}})\expy{-1}(\frac{a_p}{\SI{1}{\astronomicalunit}})\expy{1/2}(\frac{M_p}{\mearth{}})\expy{1/3}(\frac{M_*}{\msun{}})\expy{-1/6}[(\frac{\Sigma_g}{\SI{2400}{\gram\per\cubic\cm}})\expy{-1/5}(\frac{a_p}{\SI{1}{\astronomicalunit}})\expy{1/20}(\frac{m}{\SI{e18}{\gram}})\expy{1/15}]^2
\end{align}
\end{workout}

\begin{workout}[Protoplanets accretion of solids]
Other refs: Planet formation coagulation: focus on U,N (Goldreich 04), Final stages of planets formation (Goldreich 04), Formation of giant planets core: evaluating key processes (levison 09).
Safronov69: $\dot{M}_c=\Omega\Sigma_pR^2_{capture}F_G$, $R_{capture}$ larger than core radius due to gas drag, $F_G(e,i)$ gravitational focus (give rise to different growth regimes: runaway, oligarchic, orderly).
Runaway growth until protoplanets $100-1000km$ (dep on position).
Oligarchic growth: velocity of planetesimal raised by viscous stirring (gravitational scattering)/ damped by gas (until disk present).
Enhanced radius by atmospheric drag.
\end{workout}

\begin{workout}[Planetesimal dynamics: viscous stirring]
The origin of anisotropic velocity dispersion of particles in a disc potential (Ida, Makino 93, Stewart-Wetherill 1988 Ida1990)
\end{workout}
 
\chapter{Accrescimento gas}

Refs: Planet formation models: the interplay with the planetesimal disc (Fortier 2013), Characterization of exoplanets from their formation I. Models of combined planet formation and evolution (Mordasini 12)

Si risolvono numericamente le equazioni di conservazione di massa, equilibrio idrostatico, conservazione/trasporto energia 1D

\begin{workout}[Attached phase: boundary conditions]
\begin{align}
&P_{pl}=P_{Neb}\\
T_{Pl}=T_{Neb}\\
R_{Pl}=Min(R_H,R_H)
\end{align}
\end{workout}


\begin{workout}[Detached phase: transition condition and boundary condition]
Transition attached/detched phase $\approx10\mearth{}$.
Accretion shock for free-falling materials from Hill radius (more realistic circumplanetary disk: Papailoizou Nelson 05)
\begin{align}
&\dot{M}_{XY}^{max}\\
&v_{ff}^2=2GM(\frac{1}{R}-\frac{1}{R_H})\\
&P=P_{neb}+\frac{\dot{M}_{XY}}{4\pi R^2}v_{ff}+\frac{2g}{3\kappa}\\
&\tau=max(\rho_{neb}\kappa_{neb}R,2/3)\\
&T^4_{int}=\frac{3\tau L_{int}}{8\pi\sigma R^2},\ T^4=(1-A)T_{neb}^4+T_{int}^4
\end{align}
$R\approx1.5-5\rjupiter{}$ depending on entropy: THE PLANETARY ACCRETION SHOCK:I. FRAMEWORK FOR RADIATION-HYDRODYNAMICAL SIMULATIONS AND FIRST RESULTS (m17), Characterization of exoplanets from their formation III: The statistics of planetary luminosities (m17)
Bondi accretion rate: $\dot{M}_{e, Bondi}\approx\frac{\Sigma}{H}(R_H/3)^3\Omega$ or viscous accretion rate $\dot{M}_{e, visc}\approx f_{lub}3\pi\nu\Sigma_g$
\end{workout}

\begin{workout}[Higher mass gap formation reduces accretion rate]
\begin{equation}
f_{va04}=1.668(\frac{M_p}{\mjupiter{}})\expy{1/3}\exp{-\frac{M_p}{1.5\mjupiter{}}}+0.04
\end{equation}
\end{workout}

\begin{workout}[Planet-Disk exchange in hydrodynamic manner]
Ormel 15/ Cimerman 17
\end{workout}


\chapter{Migrazione}

\begin{workout}[Migration refs]
Refs: \cite{ward1997protoplanet}, \cite{terquem2000disks}, \cite{menou2004low}, (
Planet-disk interaction and orbital evolution (kley Nelson 2012), 
Baruteau 2016)
crida06: On width and shape of gaps in PPD,crida phd, 
Trilling 98:Orbital evolution and migration of giant planet: modellingextrasolar planets (Angular momentum injection rate, torque from spinning star, Impact of planet migration models on planetary populations)
Dittkrist 14: Impact of planetsmigration models n planetarypopulation (Migration II inward/outward: non equilibrium gas flow, problem of isothermal disk migration timescale: migration I regime, Fig 7,
\end{workout}

Il disco di accrescimento esercita momento torcente sul pianeta che produce un trasferimento di momento angolare dal pianeta al disco $\Lambda(a,R)$
\begin{align*}
&J=M_p\sqrt{GM_*a_p}\\
&\TDy{t}{a}=2a_p\frac{\Gamma_t}{J}=-(\frac{a}{GM_*})(\frac{4\pi}{M_P})\int_{r_{int}}^{r_{out}}r\Lambda\Sigma\,dr
\end{align*}
$r_{int}, r_{out}$ inner/outer disk radius (rif to disk ar planet)

\begin{workout}[Migration/resonance]
N-body, resonanceand migration
The dynamics of two massive planets on inclined orbits (Veras Armitage 04)
\end{workout}

\section{Migrazione tipo I: regime lineare}

Unperturbed disk: axisymmetric, keplerian rotation $\Omega(r)=\sqrt{GM_*/r^3}$, vanishing radial velocity. Planet potential, periodic in $\phi$, $\phi_p=\Omega_pt$:
\begin{align*}
&\psi_p(r,\phi,t)=-\frac{Gm_p}{|\vec{r}_p(t)-\vec{r}|}=\sum_{m=0}^{\infty}\psi_m(r)\cos{m[\phi-\phi_p(t)]}
\end{align*}
Total torque excertes by disk on planet
\begin{equation*}
\Gamma_t=-\int_{disk}\Sigma(\vecp{r}{F})\,df=\int_{disk}\Sigma(\vecp{r}\wedge\nabla\psi_p)\,df=\int_{disk}\Sigma\PDy{\phi}{\psi_p}\,df
\end{equation*}
$\Sigma$ surface density,  $\vec{F}$ specific force, $df$ surface element.
Whenever frequency of individual potential component as seen by fluid particle in disk $\omega=m(\Omega(r)-\Omega_p)$, matches a natural  oscillation frequency of the disk we have resonant condition: torques are calculated at resonant locations.


Lindblad (due to m-component of planet potential) and corotation torques:
\begin{align*}
&\Gamma_m^L=\left.\sign{(\Omega-\Omega_p)}\frac{\pi^2\Sigma}{3\Omega\Omega_p}(r\TDy{r}{\psi_m}+\frac{2m^2(\Omega-\Omega_p)}{\Omega}\psi_m)^2\right|_{r=r_L}\\
&\Gamma_m^C=\left.\frac{m\pi^2}{2}\frac{\psi_m}{r\TDy{r}{\Omega}}\TDof{r}(\frac{\Sigma}{B})\right|_{r=r_C}
\end{align*}

Differnetial Lindblad torque
$B=\frac{\kappa^2}{4\Omega}$ is the second Oort constant and represents z component of flow vorticity $\nabla\wedge\vec{v}|_z$

\begin{workout}[Migration I in isothermal disk: pressure. Adiabatic, irradiated?]
Low mass planet in isothermal disk: linear analysis of perturbed flow.
When pressure can't be neglected the resonance condition becomes $m(\Omega(r)-\Omega_p=\sqrt{\kappa^2(r)(1+\xi^2)}$ where $\xi=\frac{mc_s}{\Omega r}$ where we used isothermal sound speed and let $c_s=H\Omega$: for $m\to\infty$ L. res position becomes $r_L=r_p+\frac{2H}{3}$ (torque cutoff).

Linear. Occur if $R_H$ is smaller then H, disk scale heigth, and if viscous torque are dominant compared to gravitational torque.
Subtypes: locally isothermal, adiabatic, un/satured-corotation torque: cooling behaviour of gas (Baruteau masset08, Casoli masset 08, paardekooper10,kley 09); timescales: dittkrist 14.
Migration timescale in isothermal approx (ida lin 08 a:):
\begin{align*}
&\tau_I=\frac{1}{2.728+1.082p_{\Sigma}}(\frac{c_s}{a_p\Omega})^2\frac{M_*}{M_p}\frac{M_*}{a_p^2\Sigma_g}\Omega\expy{-1}\\
s&\dot{a}_p=-\frac{a_p}{\tau_I}
\end{align*}
Paardekooper10, Dittkrist14: total torque $\Gamma_t=\frac{1}{\gamma}(C_0+C_1p_{\Sigma}+C_2p_T)\Gamma_0$ where $C_i$ depends upon sub-regime.
Benitez Llambay 15: heating torque, Paardekooper14 Pierens 15: dynamic corotation torque.
\end{workout}

\section{Massive enough planet to open gap: migration II}
Refs: On the tidal interaction between protoplanets and the primordial solar nebula. II - Self-consistent nonlinear interaction (1986)
Non-linear. For larger planet mass, when disk density distro is changed cons., the linear hteory is no more adeguate. If angular momentum isdeposited locally (viscous dissipation/shock waves)  the disk receeds from planets
Ida-lin04: angular momentum trasport in viscous accretion disk without planets that in type II migration act as relays that transmit angular momentum at that rateacross their gap via tidal torques.
Alibert05: the planet follow the gas except when massive than local disk.
Duffell14, Durmann kley 15: questioning that migration II follows viscous evolution of the disk but due to torque??


Planet-disk interactions: exchange of angular momentum. Low mass planet-linear theory.(''Planet-disk interactions and orbital evolution'')
Low mass planets: Lindblad and corotation torque. Rapid migration of intermediate mass planets. Gap forming massive planets: slower migration determined by viscous evolution of disk. (Role of disk self-gravity and MHD turbulence as function of planet mass).


\section{Migrazione III:}
Planet-disk interactions: exchange of angular momentum. Type III migration: planet of intermediate mass. (‘’Planet-disk interactions and orbital evolution’’).

Flow-through corotation torque: $\Gamma_{flow}=2\pi\Sigma_s\dot{a}_p\Omega_pa_p^3x_s$.

\begin{align*}
&\dot{a}_p=\frac{2\Gamma^L}{\Omega_pa_p(m_p’-\delta m)}
\end{align*}

\section{Eccentricity and inclination evolution}
Planet-disk interactions: normal, tangential, radial forces. eccentricity and inclination evolution. (‘’Planet-disk interactions and orbital evolution’’)
Decomposition of planet’s potential: non-circular orbit
\begin{align*}
&\psi_p(r,\phi,t)=-\frac{Gm_p}{|\vec{r}_p(t)-\vec{r}|}=\sum_{m=0}^{\infty}\sum_{n=-m}^m\psi_{m,n}(r)\cos{m\phi+(n-m)\Omega_pt]}
\end{align*}

Eccentricity damping.
For each m,n there are 3 resonant locations: external L resonance act to increase, corotation res. to damp (if unsaturted), and coorbital L res damp $e_p$. Linear analysis for low mass planets indicates rapid exponential damping of $e_p$:
\begin{align*}
&\TDy{t}{e_p}\propto-e_p,\ \tau_e\approx(\frac{H}{r})^2\tau_{mig}\\
&e_P>\frac{H}{r}: \TDy{t}{e_p}\propto-e_p\expy{-2}
\end{align*}
Risultati analoghi per inclinazione.
