{\let\clearpage\relax\let\cleardoublepage\relax
	\chapter{Simulazione accrescimento su embrioni protoplanetarii}
}

\begin{workout}[Regimi accrescimento planetesimi]
Once the mass of runaway body is isolated from continuus powe-law mass distribution of planetesimal we can approximate the system by a two component system: protoplanet and small planetesimal.
\end{workout}

\begin{workout}[modello globale: evoluzione dal semplice al complesso]
	Global model of planets formation and evolution: more detailed submodels
	Il modello pi\'u semplice di disco di accrescimento usa distribuzione esponenziale per andamento densit\'a superficiale e temperature, assumendo disco otticamente sottile, e relazione $L_*\propto M_*^4$ di sequenza principale. Le lacune di questo modello sono
	\begin{itemize}
		\item i dischi sono otticamente spessi con transizioni nell'opacit\'a
		\item non \'e presente evoluzione temporale consistente
	\end{itemize}
\end{workout}

\section{Condizioni iniziali componente solida: efficienza massima formazione planetesimi}
%\section{Distribuzione iniziale planetesimi ed evoluzione}

\begin{workout}[conversione polvere in planetesimi??]
Armitage
Perryman
\end{workout}

Assumendo conversione completa di polvere in planetesimi, la densit\'a superficiale di planetesimi \'e
\begin{equation}
\Sigma_p(t=0,r)=f_{dg}\eta_{ice}\Sigma_g(t=0,r)
\end{equation}
con $f_{dg}$ rapporto gas/polvere ($\approx Z$) \'e una parametro casuale con distribuzione che riflette distribuzione di metallicit\'a stellari, $\eta_{ice}$ tiene conto della discontinuit\'a nella distribuzione superficiale di solidi all'iceline.

Sulla scorta della piccola differenza tra composizione solare e composizione meteoritica e assumendo inoltre che il ferro sia un buon indicatore della componente solida  si usa la formula
\begin{equation}
\frac{f_{D/G}}{f_{D/G\odot}}=10\expy{[\cel{Fe}{}{}{}/\cel{H}{}{}{}]}
\end{equation}
con $f_{D/G\odot}\approx\numrange{0.01}{0.02}$ (\cite{lodders2003solar}).

Si introduce un fattore correttivo in $f_{D/G}$ poich\'e la distribuzione di planetesimi \'e pi\'u concentrata entro i \SI{20}{\astronomicalunit} di un fattore compreso tra $2-4$ (\cite{kornet2004alternative}) rispetto alla distribuzione iniziale omogenea della polvere, dovuto alla dinamica della coagulazione della componente solida e interazioni con disco di gas (\cite{kornet2001diversity}); infine si sfrutta la distribuzione della metallicit\'a per stelle di massa solare nelle vicinanze del Sole:
\begin{equation}
p([Fe/H])=\frac{1}{\sigma\sqrt{2\pi}}\exp{-\frac{([Fe/H]-\mu)^2}{2\sigma^2}}
\end{equation}
Per le stelle campionate da CORALIE (\cite{udry2000coralie}, 1650 stelle nane entro \SI{50}{\parsec}) si ha $\mu=-0.02$ e $\sigma=0.22$.

\begin{workout}[quale distribuzione per $f_{dg}$]

\end{workout}

%Il drift dei planetesimi dovuto all'interazione col gas e la formaziane gap nella distribuzione dei planetesimi sono di solito trascurati.

\section{Accrescimento planetesimi su embrioni planetari}
(Vedi \cite{kokubo2012dynamics})

\begin{workout}[Some refs about planetesimal accretion]
	\begin{itemize}
		\item ida, guillot, morbidelli 08: accretion and destruction of planetesimal in turbulent disks.
		\item rafikov 04: Fast accretion of planetesimal by protoplanet cores. Drag regimes; mass spectrum of solid body
		\item rafikov 02: growth of planetary embryos: orderly, runaway, or oligarchic?
		\item kokubo ida 12: Dynamics and accretion of planetesimal. Tempi caratteristici.
		\item hasegawa pudritz 14: planet traps and planet cores: origin of planet metallicity correlation.
		\item coulomb logarithm in planetesimal scattering \cite{rafikov2003planetesimal} (eqs 71-72)
	\end{itemize}
\end{workout}

La distribuzione di velocit\'a dei planetesimi evolve attraverso interazione con gas, che smorza eccentricit\'a e inclinazione, e interazioni a 2 corpi con altri planetesimi.
Si definisce la componente casuale della velocit\'a v dei planetesimi rispetto a orbita kepleriana circolare
\begin{equation}
v=\sqrt{e^2+i^2}v_K
\end{equation}
Un ensamble di planetesimi evolve tramite quello che viene definito viscous stirring verso configurazione di eccentricit\'a e inclinazione che seguano distribuzione di Rayleigh
\begin{equation}
f(e,i)=4\frac{\Sigma_p}{m}\frac{ei}{\exv{e^2}\exv{i^2}}\Exp{-\frac{e^2}{\exv{e^2}}-\frac{i^2}{\exv{i^2}}}
\end{equation}
e la dispersione di velocit\'a varia come
\begin{equation}
\sigma=\sqrt{\exv{e^2}+\exv{i^2}}v_K\propto t\expy{1/4}
\end{equation}
fino a che non si raggiunge equilibrio con smorzamento da parte del disco di gas.

L'accrescimento di massa degli embioni planetari procede, in approssimazione di interazione a 2 corpi, secondo
\begin{equation}
\TDy{t}{M_e}=\pi R_e^2\rho_{pl}v F_g=A\pi R_e^2\Sigma_p\Omega F_g\label{eq:Gaccretionpl}
\end{equation}
con $F_g$ fattore che tiene conto dell'interazione gravitazionale
\begin{equation}
F_g=(1+(\frac{v_e}{v_{rel}})^2)
\end{equation}
$v_{rel}\approx v$ velocit\'a relativa e $v_e$ velocit\'a di fuga, $\rho_{pl}$ densit\'a di planetesimi
\begin{align}
	&\rho_{pl}\approx\frac{\Sigma_p}{2a\sin{i}}=A\frac{\Sigma_p\Omega}{v}
\end{align}
dove A dipende dalle propriet\'a della distribuzione di velocit\'a dei planetesimi.
%la velocit\'a relativa media aumenta fino al valore della velocit\'a di fuga dal corpo maggiore, denominato protopianeta raggiunto diametro migliaia di kilometri.

Nella prima fase di accrescimento runaway si produce una distribuzione di massa di planetesimi esponenziale per la densit\'a numerica di planetesimi
\begin{equation}
\TDy{m}{n_p}\propto m\expy{\alpha}
\end{equation}
e alcuni corpi decisamente pi\'u massicci di massa $M_e$. L'accrescimento di massa relativo nel regime runaway \'e della forma
\begin{equation}
\frac{1}{M_e}\TDy{t}{M_e}\propto M_e\expy{1/3}\label{eq:runawayaccretion}
\end{equation}
La massa dei planetesimi domina il sistema.
L'evoluzione orbitale di quelli che diventeranno embrioni planetari \'e caratterizzata da separazione orbitale $\Delta>5r_h$ e piccola velocit\'a casuale. 

\cite{ida1993scattering} hanno studiato il regime di accrescimento in cui l'embrione planetario \'e diventato sufficientemente massiccio da dominare l'evoluzione della velocit\'a casuale dei planetesimi
\begin{align}
&2\Sigma_MM>\Sigma_mm\\
&\Sigma_M=\frac{M}{2\pi a\Delta a_{stir}}
\end{align}
e $\Delta a_{stir}\propto\exv{e_M}\expy{1/2}\propto M_e\expy{1/3}$ \'e la larghezza dell'anello di planetesimi in cui \'e efficace il viscous stirring dell'embrione planetario. Questa fase di accrescimento \'e detta oligarchica e si ha
\begin{equation}
\frac{1}{M_e}\TDy{t}{M_e}\propto M_e\expy{-1/3}\label{eq:oligarchicaccretion}
\end{equation}
La simulazione inizia quindi alla fine dell'accrescimento runaway con la determinazione della posizione iniziale degli embrioni planetari, di massa fissata tra $\numrange{e-8}{e-2}\mearth$, scelta in maniera casuale: la probabilit\'a $p(a)\,da$ che l'embrione si sia formato in intervallo $da$ \'e inversamente proporzionale alla distanza orbitale $\Delta\propto a$ quindi la distribuzione \'e uniforme in $\log{a}$ nella range di semiassi maggiori scelti per la simulazione, qui \SIrange{0.05}{40}{\astronomicalunit}. Salvo verificare che il contenuto di solidi della sua fascia di accrescimento sia maggiore della sua massa:
\begin{equation}
M_{iso}=2\pi a_{start}B_LR_h\Sigma_p=\frac{(4\pi B_La_{start}^2\Sigma_p)\expy{3/2}}{(3M_*)\expy{1/2}}
\end{equation}
%$p(a)\,da\propto\frac{da}{\Delta}\propto\,d\log{a}$
L'evoluzione della distribuzione dei planetesimi tiene conto dell'acrescimento sugli embrioni planetari
\begin{equation}
\dot{\Sigma}_p=-\frac{(3M_*)\expy{1/3}}{6\pi a_p^2B_LM_p\expy{1/3}}\dot{M}_c\label{eq:surfaceplanetesimal}
\end{equation}
e dell'evoluzione di $\exv{e^2}\expy{1/2}$, $\exv{i^2}\expy{1/2}$ (\cite{fortier2013planet}).
Il tempo di comparsa dell'embrione planetario, tempo per accrescere massa iniziale fissata, \'e determinato risolvendo (\ref{eq:surfaceplanetesimal}) e (\ref{eq:Gaccretionpl}).
La simulazione termina dopo \SI{10}{\mega\year}; la fase finale di crescita caotica evolve il sistema verso una configurazione dinamica stabile ed ha tempi tipici di \SIrange{10}{100}{\mega\year}.

\begin{workout}[Calculation of planetesimal random velocity for accretion time]
Pollack 96
Greenzweig Lissauer 92
\end{workout}
\begin{workout}[Accretion timescale]
\begin{equation}
\tau_{acc}\approx\SI{1.2e5}{\year}(\frac{\Sigma_p}{\SI{10}{\gram\per\square\cm}})\expy{-1}(\frac{a_p}{\SI{1}{\astronomicalunit}})\expy{1/2}(\frac{M_c}{\mearth})\expy{1/3}(\frac{M_*}{\msun})\expy{-1/6}[(\frac{\Sigma_g}{\SI{2400}{\gram\per\square\cm}})\expy{-1/5}(\frac{a_p}{\SI{1}{\astronomicalunit}})\expy{1/20}(\frac{m}{\SI{e18}{\gram}})\expy{1/15}]^2
\end{equation}
\end{workout}

% Costante proporzionalit\'a \'e $\frac{\sqrt{3}}{2}$ assumendo relazione di dispersione per velocit\'a planetesimi isotropa.
%Per embrioni planetarii di decine di chilometri l'attrazione gravitazionale aumenta la sezione d'urto: 
%\begin{equation}
%\frac{1}{M_e}\TDy{t}{M_e}\propto M_e\expy{1/3}
%\end{equation}
%questo regime di accrescimento (runaway growth) \'e responsabile crescita diametro embrioni da \SI{10}{\kilo\meter} a \SI{100}{\kilo\meter} in \SIrange{e4}{e5}{\year}.
%(higher radial excursion ae)
%La massa dei core varia per l'accrescimento dei planetesimi secondo
%\begin{align}
%	&\dot{M}_c=\Omega\Sigma_pR^2_{capt}F_G
%	%&\dot{\Sigma}_p(r)=-\frac{1}{2\pi aB_LR_H}\dot{M}_c
%\end{align}

\begin{workout}[Transizione accrescimento oligarchico]
	Transition to oligarchic growth (\cite{thommes2003oligarchic}): dominance of protoplanet-planetesimal scattering: $2\Sigma_MM>\Sigma_mm$ $\Sigma_M=\frac{M}{2\pi a\Delta a_{stir}}$
\end{workout}

\begin{errata}[Valori medi iniziali e,i]
	I valori iniziali sono $\exv{i^2}=\exv{e^2}=[2\frac{r_H}{a}]^2$
\end{errata}

\begin{workout}[Oligarchic growth is more important in terms of mass accretion than runaway growth]

\end{workout}

\begin{workout}[Isolation mass:]
	Quando la massa degli embrioni domina su quella dei planetesimi si ha transizione a regime oligarchico in cui l'accrescimento \'e pi\'u lento fino a che non \'e stata accresciuta la massa presente nella zona di pertinenza gravitazionale dell'embrione,detta isolation mass (\cite{lissauer1993planet}):
	\begin{align}
		&M_{iso}=\frac{(4\pi Br^2\Sigma_p)\expy{3/2}}{(3M_*)\expy{1/2}}=\num{2.10e-3}(\frac{Br^2\Sigma_p}{2\sqrt{3}})\expy{3/2}(\frac{\msun{}}{M_*})\expy{1/2}\mearth{}
	\end{align}
	kokubo ida 12:
	\[M_{iso}\approx2\pi ab\Sigma_d=0.16f_d\expy{3/2}\epsilon_{ice}\expy{3/2}(\frac{b}{10r_h})\expy{3/2}(a(\si{\astronomicalunit}))\expy{3/4}(\frac{M_*}{\msun{}})\expy{-1/2}\mearth{}\]
	\begin{equation}
	M_{iso}\propto M_*\expy{-1/2}\Sigma_p\expy{3/2}r^3
	\end{equation}
\end{workout}

\begin{workout}[Protoplanets accretion of solids]
	Other refs: Planet formation coagulation: focus on U,N (Goldreich 04), Final stages of planets formation (Goldreich 04), Formation of giant planets core: evaluating key processes (levison 09).
	Safronov69: $\dot{M}_c=\Omega\Sigma_pR^2_{capture}F_G$, $R_{capture}$ larger than core radius due to gas drag, $F_G(e,i)$ gravitational focus (give rise to different growth regimes: runaway, oligarchic, orderly).
	Runaway growth until protoplanets $100-1000km$ (dep on position).
	Oligarchic growth: velocity of planetesimal raised by viscous stirring (gravitational scattering)/ damped by gas (until disk present).
	Enhanced radius by atmospheric drag: Enhanced collisional growth of a protoplanet that has an atmosphere (Inaba Ikoma 03)
\end{workout}

\begin{workout}[orderly growth]
	lissauer pg 144: equapartition effect saturation. This regime is never observed in simulations.
	\begin{equation}
	\frac{1}{m_p}\TDy{t}{m_p}\propto m_p\expy{-1/3}
	\end{equation}
\end{workout}

\begin{workout}[runaway growth]
	per planetesimi circa 1km (perch\'e?)
	\begin{equation}
	\frac{1}{m_p}\TDy{t}{m_p}\propto m_p\expy{1/3}
	\end{equation}
\end{workout}

\begin{workout}[feeding zone: jacobi integral]
	Jacobi integral
	\begin{align}
		&E_J=\frac{1}{2}(e^2+i^2)a^2\Omega^2-\frac{3}{8}b^2\Omega^2+\frac{9}{2}R_H^2\Omega^2\\
		&\tilde{E}_J=\frac{E_J}{a^2h^2\Omega^2},\ h=R_H/a
	\end{align}
	Planetesimi con $\tilde{E}_J>0$ possono accrescere il protopianeta  cio\'e entro $w_{feed}=B_LR_H$, per orbite circolari $B_L=2\sqrt{3 }$.
\end{workout}

\begin{workout}[runaway to oligarchic. Isolation mass]
	NEW CONDITION FOR THE TRANSITION FROM RUNAWAY TO OLIGARCHIC GROWTH (Ormel 10)
	\begin{equation}
	M_{iso}\propto M_*\expy{-1/2}\Sigma_p\expy{3/2}r^3
	\end{equation}
	$0.07\mearth{}$ 1AU, $9\mearth{}$ 5 AU.
	Accretion among preplanetary bodies: runaway growth, summary of collision interactions
	Understanding how planets become massive
\end{workout}

\begin{workout}[Accrescimento pianetesimi: orderly, runaway, oligarchic]
	The growth of planetary embryos:  orderly, runaway, or oligarchic? (Rafikov 2002)
	\begin{align}
		\TDy{t}{M_e}\approx\pi R_e^2\Omega mN\frac{v}{v_z}[1+\frac{2GM_e}{R_ev^2}]
	\end{align}
	Accrescimento ordinato: $\frac{1}{M_e}\TDy{t}{M_e}\propto M_e\expy{-1/3}$. Accrescimento runaway: $\frac{1}{M_e}\TDy{t}{M_e}\propto M_e\expy{1/3}$. Accrescimento oligarchico: $\frac{1}{M_e}\TDy{t}{M_e}\propto M_e\expy{-1/3}$.
\end{workout}

\begin{workout}[oligarchic growth rate]
	La fase successiva che porta a dimensioni di migliaia di chilometri, accrescimento oligarchico ha andamento in funzione della massa
	\begin{equation}
	\frac{1}{M_e}\TDy{t}{M_e}\propto M_e\expy{-1/3}
	\end{equation}
	e tempo caratteritico $\SI{e5}{\year}$. Nel dettaglio \'e determinata 
\end{workout}

\begin{workout}[viscous stirring, Dynamical friction]
	Una volta formato un corpo molto pi\'u massiccio degli altri (embrione planetario), l'equipartizione di energia tra il corpo massiccio e i planetesimi produce diminuzione di eccentricit\'a/inclinazione dell'embrione planetario a scapito di aumento di queste per i corpi di massa minore, questo fenomeno \'e detto dynamical friction.
	
	Dynamical friction: stewart wetherill 88
\end{workout}

\begin{workout}[Dynamical evolution of planetesimal swarm: refs]
	Accretional evolution of planetesimal swarm (weidenschilling 97)
	Goldreich 2004: Planet Formation by Coagulation: A Focus on Uranus and Neptune
	Goldreich 04: FINAL STAGES OF PLANET FORMATION
	rafikov 03: DYNAMICAL EVOLUTION OF PLANETESIMALS IN PROTOPLANETARY DISKS.
	rafikov 02: The growth of planetary embryos: orderly, runaway, or oligarchic?
	thommes 03: Oligarchic growth of giant planets
\end{workout}

\begin{workout}[10m-10km, 100km-1000km, 1000km-10000km: refs]
	Lissauer pg 142-143
	Perryman pg 226
\end{workout}

\begin{workout}[Planetesimal dynamics: viscous stirring]
	The origin of anisotropic velocity dispersion of particles in a disc potential (Ida, Makino 93, Stewart-Wetherill 1988 Ida1990)
\end{workout}

\begin{workout}[Gravitational stirring: Viscous stirring and dynamical friction]
	\begin{itemize}
		\item Viscous stirring increses i,e
		\item Dynamical stirring: tends to equalize energy of random motion among bodies having different masses and velocities
	\end{itemize}
\end{workout}

\begin{workout}[orbital elements distribution: rayleigh distro]
	I planetesimi hanno distribuzione di velocit\'a casuale, localmente equivalente a
	\begin{equation}
	f(e,i)=4\frac{\sigma}{m}\frac{ei}{\exv{e^2}\exv{i^2}}\Exp{[-\frac{e^2}{\exv{e^2}}-\frac{}{\exv{i^2}}]}
	\end{equation}
	\'e una distro gaussiana triassiale in coordinate cilindriche (Lissauer Stewart 93)
	Lissauer pg 142-143
\end{workout}

\begin{workout}[Atmospheric drag increases accretion rate]
	
\end{workout}


\begin{workout}[Initial planetesimal distro]
	Dust converted early everywhere fully-efficient (mordasini18: pg 12),Thommes pg8)
	\begin{align*}
		&\Sigma_p(t=0,r)=f_{dg}\eta_{ice}\Sigma_g(t=0,r)\\
		&\dot{\Sigma}_p(r)=-\frac{1}{2\pi aB_LR_H}\dot{M}_c
	\end{align*}
	Icelines Mordasini 1141: inside where T exceeds sublimation
	{Embryo starting position}
	Usually a distro uniform in log(a): relative spacing of few Hill spheres (Kokubo Ida 10)
	Ida, lin10: asymptotical isolation mass ''Toward a Deterministic Model of Planetary Formation VI'',
	Trapped evolution model: Hasegawa pudritz 11, Cridland 16
\end{workout}

%\section{Caratteristiche iniziali degli embrioni planetarii e accrescimento}

%{\let\clearpage\relax\let\cleardoublepage\relax
%\chapter{Schema di accrescimento componente solida: formazione planetesimi - da riassumere nel capitolo ''schema CA a volo d'uccello''}
%}
%\section{Sedimentazione polvere e formazione planetesimi}
%Assumo  sedimenta verso il centro del disco e forma aggregati di dimensioni maggiori:
%\section{Accrescimento planetesimi: formazione proto-pianeti.}
%\subsection{Distribuzione velocit\'a planetesimi}
%Refs: kokubo Ida 12 `'Dynamics and accretion of planetesimal''
%e si ha equipartizione di energia tra pianetesimi di diversa massa
%i planetesimi di massa minore hanno maggiore dispersione di velocit\'a 
%due planetesimi hanno stessa velocit\'a relativa prima e dopo interazione ma aumenta la velocit\'a relativa al moto kepleriano in maniera casuale: 
%\'E ragionevole supporre che si arrivi rapidamente alla condizione $v>\Omega R_H$.
%v deviazione dei planetesimi da velocit\'a kepleriana
%\subsection{Regimi accrescimento dei protopianeti}

\begin{workout}[Planet host stars are enriched in nikel and silicon]
	Robinson06
\end{workout}

\begin{workout}[Refs GI vs CA]
	planetesimal hypothesis: chamberlin 05, Safronov 69, Hayashi 77. Formation on dynamical scale via GI: Kuiper 51, Cameron 62.
	Core accretion o instabilit\'a gravitazionale?
	La correlazione tra metallicit\'a del disco protoplanetario e il numero di pianeti (giganti) \'e in accordocon osservazioni?
	The runt of the litter: why planets formed through gravitational instability can only be failed binary stars (2010).
	Planetary formation scenario revisited: CA vs GI (2007)
\end{workout}

\begin{workout}[Succo beamer formazione]
	
\end{workout}

\begin{workout}[Succo beamer formazione]
	Per piccole deviazioni dalla velocit\'a kepleriana si ha
	\begin{equation}
	v_k-v_g\approx -v_k\frac{1}{\rho}\TDy{r}{P}/(2g)\label{eq:velocitykg}
	\end{equation}
	con $g=GM_*/r^2$.
\end{workout}

\begin{workout}[Dust dynamics]
	L'equazione del moto per particella di polvere \'e
	\begin{equation}\label{eq:motiondust}
	m_p\TDy{t}{\vec{v}}=\vec{F}_D-m_p\Omega^2z\hat{z}
	\end{equation}
	$\vec{F}_D$ \'e la forza esercitata dal gas sulla particella di polvere che si oppone al moto relativo con velocit\'a $v$
	\begin{equation}
	F_D=\frac{1}{2}C_D\pi s^2\rho_gv^2
	\end{equation}
	nel caso cammino libero medio delle molecole di gas sia maggiore delle dimensioni della particella $C_D=\frac{8}{3}\frac{v_{th}}{v_z}$ (Epstein drag) per particelle pi\'u grandi \'e funzione del numero di Reynold.
	%La pressione non supporta i grani: troppo pesanti
\end{workout}

\begin{workout}[Velocit\'a stazionaria settling verticale]
	Dall'equazione \eqref{eq:motiondust} si che in condizioni stazionarie:
	\begin{equation}
	v_{set}=\frac{\Omega^2}{v_{th}}\frac{\rho_d}{\rho_g(z)}sz
	\end{equation}
	Per particelle di \SI{1}{\micro\meter} tempo di sedimentazione \'e $\SI{2e5}{\year}$, inoltre turbolenza e coagulazione delle particelle hanno un ruolo non banale.
	%t=mv/F
\end{workout}

\begin{workout}[Planetesimal formation mechanisms]
	La velocit\'a relativa tra gas e polvere causa un drift verso la stella di quest'ultima massimo per particelle di \SIrange{0.01}{10}{\meter}  a \SI{1}{\astronomicalunit} con tempo caratteristico di \SI{100}{\year} (\cite{lissauer1993planet}).
	
	Per passare a corpi di dimensioni kilometriche si ipotizzano 2 scenarii:
	\begin{itemize}
		\item La componente solida del disco di accrescimento sedimenta rapidamente in disco sottile: secondo il modello di Goldreich-Ward il disco di polvere \'e instabile e le condensazioni generano i planetesimi.
		\item In assenza di sedimentazione la formazione procede tramite urti a 2 corpi e la turbolenza creando addensamenti, pu\'o accelerare la formazione di planetesimi.
	\end{itemize}
\end{workout}

\begin{workout}[refs planetesimal formation]
	Armitage 07: lecture notes on formation and early evolution
\end{workout}

\begin{workout}[Velocit\'a sedimentazione stazionaria]	
	Grani piccoli raggiungono rapidamente la velocit\'a stazionaria di regime
	\begin{equation}
	v_{set}=\frac{\Omega^2}{v_{th}}\frac{\rho_d}{\rho_g(z)}sz
	\end{equation}
\end{workout}

\begin{workout}[Dust settling model]
	weidenschilling 89: dust settling time
	Dullemond dominique 04, johansen klahr 05, carballido 05, fromand papaloizou 06, turner 06,07
	Furlan 06: Effects of dust growth and settling in T Tauri disks
	Nomura 06: Dust Size Growth and Settling in a Protoplanetary Disk
	\cite{lissauer1993planet}
\end{workout}

\begin{workout}[Dust dynamics refs]
	Weidenschilling cuzzi 06: Particle-gas dynamics and primary accretion, Apai Lauretta Protoplanetary Dust pg 100
	Protoplanetary disk and their evolution pg 29
	Protoplanetary dust: Apai Lauretta - particles dynamics pg 100
\end{workout}

\begin{workout}[Particle-gas dynamics. dust midplane sedimantation]
	Weidenschilling cuzzi 06: Particle-gas dynamics and primary accretion, Apai Lauretta Protoplanetary Dust pg 100
	Stopping time $t_s=\frac{\rho_sa}{\rho c_0}$ ($a<\lambda_g$). L'evoluzione dinamica delle particelle solide \'e determinata flussi macroscopici dovuti all'evoluzione del disco, gas-drag, settling
\end{workout}

\begin{workout}[Formazione planetesimi: meccanismo Goldreich-Ward]
	Weidenschilling cuzzi 06: Particle-gas dynamics and primary accretion, Apai Lauretta Protoplanetary Dust pg 100
	Stopping time $t_s=\frac{\rho_sa}{\rho c_0}$ ($a<\lambda_g$). L'evoluzione dinamica delle particelle solide \'e determinata flussi macroscopici dovuti all'evoluzione del disco, gas-drag, settling
\end{workout}

\begin{workout}[allargamento schema accrescimento parte solida]
	\begin{itemize}
		\item peebles: observations show that mm and smaller particles are retained in PPD (Cleeves16). Peebles remain coupled to gad -> areodynamic assisted or peeble accretion (Ormel klahr 10, Lambrechts jOhansen 12)
	\end{itemize}
\end{workout}

{\let\clearpage\relax\let\cleardoublepage\relax
\chapter{Accrescimento di gas ed evoluzione disco di accrescimento.}\label{chap:gasaccretion}
}% e fase isolata

La simulazione calcola l'accrescimento di gas e l'evoluzione orbitale degli embrioni fino alla scomparsa del disco di accrescimento, circa \SI{10}{\mega\year}, mentre l'evoluzione termodinamica del pianeta, che ne determina massa e raggio, per \SI{10}{\giga\year} tenendo conto dell'evoluzione stellare (\cite{dell2012pisa}).

\begin{workout}[Notazione omogenea: disco accrescimento]
$T_{neb}$, $T_b$, $T_i$; $k_l$, $B_L$; ...
\end{workout}
\begin{workout}[Grain opacity and the bulk composition of extrasolar planets (Alibert mordasini klahr)]
	$b=10.4+\log{f}$
\end{workout}

\begin{workout}[Higher mass gap formation reduces accretion rate]
	\begin{equation}
	f_{va04}=1.668(\frac{M_p}{\mjupiter{}})\expy{1/3}\exp{-\frac{M_p}{1.5\mjupiter{}}}+0.04
	\end{equation}
\end{workout}

\begin{workout}[Some refs]
	\begin{itemize}
		\item Hasegawa Pudritz 13: Planetary population in mass-period diagram: a statistical treatment of exoplanet formation and role of planet traps.
		\item Rafikov 05: Atmospheres of protoplanetary cores. Espressioni per $R_H$, $R_B$.
		\item shiraiashi ida 08:Infall of planetesimal onto growing giant planets. KH contraction timescale; accretion of planetesimal and contraction phase.
	\end{itemize}
\end{workout}

\begin{workout}[Main point: of gas accretion]
	\begin{itemize}
		\item timescale to accrete mass in runaway mode is shorter than disk timescale: planetary desert in $10-100\mearth{}$??
	\end{itemize}
\end{workout}

\section{Calibrazione efficienza viscosit\'a e fotoevaporazione}

Il tempo caratteristico del disco di accrescimento \'e determinato dalla viscosit\'a $\alpha$: fissata $\alpha$ compatibile con tempo caratteristico osservato $\tau_{disk}^{obs}\approx\SI{3}{\mega\year}$ e assumendo la distribuzione uniforme nel logaritmo di $\dot{M}_w$, si calcola  $t_{disk}(\alpha,\Sigma_0,\dot{M}_w)$ determinando gli estremi della foto-evaporazione per riprodurre tempi di vita osservati.

Valori tipici sono $\dot{M}_w=\SIrange{5e-10}{3e-8}\msun{}/\si{\year}$ per $\alpha=\num{7e-3}$.

Nelle popolazione planetaria considerata $\alpha$ \'e fissato compatibilmente con le osservazioni  ed \'e omogeneo e costante; l'evoluzione della densit\'a superficiale di gas si determina tramite:
\begin{align}
	&\TDy{t}{\Sigma}=\frac{1}{r}\PDof{r}[3r\expy{1/2}\PDof{r}(\nu\Sigma r\expy{1/2})]+\dot{\Sigma}_w(r)+\dot{\Sigma}_{embryo}(r)\label{eq:diskaccrphev-m18}\\
	&\dot{\Sigma}_w(a)=\left\{\begin{array}{c}0\quad a<R_g\\\frac{\dot{M}_w}{2\pi(a_{max}-R_g)a}\quad \text{altrimenti}\\\end{array}\right.
\end{align}
con $\dot{\Sigma}_w$ contributo della foto-evaporazione, $R_g=\SI{5}{\astronomicalunit}$ e $a_{max}$ estremo del disco,  $\dot{\Sigma}_{embryo}$ termine di accrescimento di gas sugli embrioni planetari.
%La densit\'a superficiale iniziale \'e assunta essere:
%\begin{equation}
%\Sigma(a,t=0)=\Sigma_0(\frac{r}{1AU})\expy{p_g}\Exp{[-(\frac{r}{R_o})\expy{2+p_g}]}(1-\sqrt{\frac{r}{R_i}})
%\end{equation}
%dove i parametri possono essere scelti casualmente secondo una distribuzione di probabilit\'a ricavata dalle osservazioni.


\begin{workout}[X-ray photoevaporation]
(owen12)
thermal equilibrium: T of optically thin X-ray heated gas approx monotonic function of ionization parameter $\xi=\frac{L_X}{nr^2}$. The flow start subsonically at base of streamline and is accelerated by $g_{eff}$ and pressure gradient. Viscous effects are negligible: specific angular momentum of each element fluid is conserved being at keplerian value at launch from radius $R_b$ with value $h^2=GM_*R_b$:
\[\vec{g}_{eff}=-\frac{GM_*}{r^2}\hat{r}+\frac{GM_*R_b}{R^3}\hat{R}\]
where $\hat{r}$ and $\hat{R}$ are radial unit vector in spherical and cylindrical coordinate
\end{workout}

\begin{workout}[opacit\'a e rapporto $M_Z/M_{env}$]
	\cite{mordasini2014grain}
\end{workout}

\begin{workout}[envelope mass as function of core mass]
	Armitage 17 eq 232 (`'lecture nite on formation and early evolution of PS)
	\begin{equation}
	M_{env}\approx\int_{R_c}^{R_o}4\pi r^2\rho\,dr\propto\frac{\sigma}{\kappa_RL}(\frac{\mu m_pGM_t}{4k_b})^4\ln{\frac{R_o}{R_c}}
	\end{equation}
\end{workout}

\begin{workout}[Rfes per espressione raggio bondi]
	
\end{workout}


\begin{workout}[Hydrostatic equilibrium hypothesis]
	Characterization of exoplanets from their formation I (eq 10)
\end{workout}


\begin{workout}[Rfes espressione accrescimento gas]
	Rate di accrescimento limitato dalla velocit\'a di raffreddamento:
	\begin{equation}
	\dot{M}_{XY}\propto\ \frac{M_p}{M_*}<(H_P/R_p)^3/\sqrt{3}
	\end{equation}
	quindi la massa di gas aumenta esponenzialmente a partire da $M_c\approx10\mearth{}$.
\end{workout}

\section{Accrescimento limitato da velocit\'a di raffreddamento}

Seguendo (\cite{mordasini2012characterization}) la struttura del pianeta \'e determinata integrando le equazioni di conservazione di massa, momento e l'equazione del trasporto di energia:
\begin{align}
&\TDy{r}{m}=4\pi r^2\rho\\
%&\TDy{r}{l}=0\\
&\TDy{r}{P}=-\frac{Gm}{r^2}\rho\\
&\TDy{r}{T}=\frac{T}{P}\TDy{r}{P}\nabla(T,P)\\
&\nabla(T,P)=\TDly{P}{T}=\min{(\nad{},\nrad{})}\\
%&(\PDy{r}{l}=4\pi r^2\rho(\epsilon-P\PDy{t}{V}-\PDy{t}{u}))
\end{align}
dove $\nrad{}$ e $\nad{}$ indicano il gradiente radiativo e adiabatico.

La luminosit\'a del pianeta \'e determinata tramite
\begin{equation}
E_t=E_g+E_i=-\int_0^M\frac{Gm}{r}\,dm+\int_{M_z}^Mu\,dm=-\xi\frac{GM^2}{2R}
\end{equation}
che sostituita nell'equazione di conservazione dell'energia da:
\begin{equation}
-\TDof{t}E_t=L=L_M+L_R+L_{\xi}=\xi\frac{GM}{R}\dot{M}-\xi\frac{GM^2}{2R^2}\dot{R}+\frac{GM^2}{2R}\dot{\xi}\label{eq:planetL}
\end{equation}
con $\dot{M}=\dot{M}_Z+\dot{M}_{XY}$ e $u$ energia interna per unit\'a di massa.
Per determinare $L_{\xi}$ si risolve la struttura utilizzando i primi due termini in (\ref{eq:planetL}) e si calcola $L$ imponendo la conservazione dell'energia.

\begin{workout}[Chiarire luminosit\'a]
Luminosit\'a totale $L=L_{cont}+L_{acc}$ is the sum of energy gained by contraction and by accretion of planetesimal:
\begin{align}
&L_{contr}=-\frac{E_t(t+dt)-E_t(t)-E_{gas,acc}}{dt}\\
&E_{gas,acc}=dt\dot{M}_{gas}u_{int}
\end{align}
is the energy gained by accretion of nebular gas with specific energy $u_{int}$ at rate $\dot{M}_{gas}$; accetion due to planetesimal is assumed to deposit energy to the core
\begin{equation}
L_{acc}=G\frac{\dot{M}_{core}M_{core}}{R_{core}}
\end{equation}
Energy equation
\begin{equation}
\TDy{m_r}{L}=\epsilon_{pla}-T\TDy{t}{S}
\end{equation}
\end{workout}

Ho indicato la massa di gas legata al core con
\begin{equation}
M_{XY}=4\pi\int_{R_c}^R\rho(r')r'^2\,dr'
\end{equation}
$R_C$ \'e il raggio del core.

Condizioni al bordo:
\begin{align}
&R=\frac{R_B}{1+R_B/(k_lR_H )},\ P=P_{neb}\\
&\tau=\max{(\rho_{neb}\kappa_{neb}R),2/3)},\ T_i^4=\frac{3\tau L_{int}}{8\pi\sigma R^2}\\
&T^4=T_{neb}^4+T_{int}^4,\ L(R)=L_{int}
\end{align}
$k_l\approx3-4$, quindi $R_p\approx \min{(R_B,k_lR_H)}$ e introduco il raggio di Bondi
\begin{equation}
R_B=G\frac{M_c}{c_0^2}%\approx\SI{4e10}{\cm}a(AU)\expy{1/2}\frac{M_c}{\mearth{}
\end{equation}
con $c_0=\sqrt{\frac{kT}{\mu m_h}}$,
definito come raggio in cui energia termica e potenziale gravitazionale si equivalgono. %il core perturba la pressione del gas del disco.
La velocit\'a di accrescimento \'e determinata dalla velocit\'a di contrazione della struttura planetaria in equilibrio idrostatico.

\section{Accrescimento limitato dalla disponibilit\'a di gas del disco}

\begin{wrapfigure}[20]{l}{0.45\textwidth}
	\includegraphics[trim={2cm 2.5cm 2cm 0.1cm},clip, keepaspectratio,width=0.45\textwidth]{massenvvscore}
	\caption{Accrescimento di gas in funzione del tempo: raggiunta massa critica del core solido si ha fase di accrescimento molto rapido (tratto quasi verticale in figura). Da \cite{alibert2005models}.}\label{fig:massenvvscore}
\end{wrapfigure}

Se il protopianeta raggiunge un valore critico di massa, dipendente debolmente da luminosit\'a generata da accrescimento dei pianetesimi e opacit\'a non si ha pi\'u una soluzione idrostatica. In questa seconda fase l'accresscimento di massa diviene cos\'i rapido (curva verticale di figura (\ref{fig:massenvvscore})) da esaurire il gas disponibile nella regione. L'accrescimento di gas ha come limite superiore il flusso di gas dovuto a evoluzione viscosa del disco.

Da modelli numerici risulta
\begin{equation}
M_c^{crit}=10\mearth{}(\frac{\dot{M}_c}{\num{e-7}\mearth{}\si{\per\year}})\expy{q}(\frac{\kappa}{\SI{0.1}{\square\meter\per\kilo\gram}})\expy{s}
\end{equation}
con $q,s\approx0.2-0.3$ (\cite{ikoma2000formation}).

Se $H\approx R_H$ il pianeta perturba in maniera non trascurabile il disco (il rate di accrescimento di gas \'e maggiore di quello fornito dal disco). Il raggio del pianeta \'e determinato dalle condizioni al bordo per materia accresciuta tramite free-fall da $R_H$ a $R_p$:
\vspace{2cm}
\begin{align}
&\dot{M}_{XY}=\dot{M}_{XY,max},\ v_{ff}^2=2GM(\frac{1}{R}-\frac{1}{R_H})\\
&P=P_{neb}+\frac{\dot{M}_{XY}}{4\pi r^2}v_{ff}+\frac{2g}{3\kappa},\ \tau=\max{(\rho_{neb}\kappa_{neb}R,2/3)}\\
&T_{int}^4=\frac{3\tau L_{int}}{8\pi\sigma R^2},\ T^4=(1-A)T_{neb}^4+T_{int}^4
\end{align}

In questa fase la velocit\'a di accrescimento di gas \'e determinata dall'evoluzione viscosa del disco:
\begin{equation}
\dot{M}_{e,visc}=f_{hyd}3\pi\nu\Sigma_g
\end{equation}
$f_{hyd}\approx0.9$ valore determinato da simulazioni idrodinamiche (\cite{lubow1999disk}).

\begin{workout}[Bondi accretion rate]
Unperturbed viscous flow
\begin{equation}
\dot{M}_{e,B}\approx\frac{\Sigma_g}{H}(\frac{R_H}{3})^3\Omega
\end{equation}
\end{workout}

\begin{workout}[Detached phase accretion rate]
Characterization of exoplanets from their formation pg 8
\end{workout}

\begin{workout}[Wien displacement]
$\lambda_{max}T\approx \SI{3e-3}{\meter\kelvin}$
\end{workout}

\begin{workout}[Critical core mass]
From toward deterministic
\begin{equation}
M_{c,crit}\approx10(\frac{\dot{M}_c}{\num{e-6}\mearth{}\si{\per\year}})\expy{0.2-0.3}(\frac{\kappa}{\SI{1}{\square\cm\per\gram}})\expy{0.2-0.3}\mearth{}
\end{equation}•
\end{workout}

\begin{workout}[gas accretion refs]
Lissauer 09: Models of Jupiter’s growth incorporating thermal and hydrodynamic constraints
Rafikov 10: ''Constraint on giant planet production by core accretion''
Rafikov 04 Atmospheres of protoplanetary cores: critical mass for nucleated instability.
Refs: Planet formation models: the interplay with the planetesimal disc (Fortier 2013), Characterization of exoplanets from their formation I. Models of combined planet formation and evolution (Mordasini 12)
\end{workout}

\begin{workout}[Planet-Disk exchange in hydrodynamic manner]
Ormel 15/ Cimerman 17
\end{workout}

%\section{Fase isolata}

\begin{workout}[Fase isolata: fonti energia]
fonti energia: tidal heating radiogeninc heat, star flux
\end{workout}


{\let\clearpage\relax\let\cleardoublepage\relax
%\chapter{Evoluzione orbite proto-pianeti: migrazione planetaria (e N-body integrator: implementazione; la parte di teoria nella prima parte dei vincoli e l'implementazione dell'integratore nell seconda parte?).}
\chapter{Interazioni N-body tra embrioni planetari  e con il disco di gas}
}

\begin{wrapfigure}[14]{l}{0.5\textwidth}
\includegraphics[trim={14.5cm 11cm 11cm 0},clip, keepaspectratio,width=0.5\textwidth]{pdres}
\caption{Simulazione evoluzione orbite planetarie in disco di accrescimento fino a cattura in risonanza $2:1$: le regioni pi\'u dense sono in rosso. Da \cite{kley2012planet}.}\label{fig:pdres}
\end{wrapfigure}

L'osservazione di una sottopopolazione di esopianeti giganti su orbite strette, per cui sembra improbabile una formazione in loco, e la presenza di sistemi multipli in risonanza sono indicatori di evoluzione orbitale, d'altra parte nel Sistema solare fenomeni di migrazione fornisco una spiegazione all'orbita di plutone, alla presenza di numerosi oggetti della fascia di Kuiper in risonanza $3:2$ con Nettuno e al periodo di collisioni intense testimoniato da craterizzazione (Late heavy bombardment).

%\vspace{1.5cm}

Fenomeni che possono dar luogo a migrazione planetaria:
\begin{itemize}
\item Interazione con disco proto-planetario.

Il potenziale periodico del pianeta perturba la densit\'a del gas: si ha scambio di momento angolare alle risonanze di Lindblad e alle orbite nella regione di corotazione.

%Considerando le perturbazioni lineari nella densit\'a del disco prodotte dal potenziale del pianeta si determina la risultante dei momenti torcenti $\Gamma$ dovuti alle risonanze di Lindblad e alla regione di corotazione.
Velocit\'a e direzione della migrazione dipendono dalle caratteristiche del disco, massa del pianeta e moto relativo pianeta-gas.
\item Interazione con disco residuo di planetesimi

\item Interazione tra due o pi\'u pianeti giganti

\item Interazione con stella in sistema di stelle binarie

\item Interazione mareale con la stella

\end{itemize}

\begin{errata}[Elenco schematicamente alcuni risultati da \cite{armitage2007lecture} e \cite{crida2006planetary}]
 La migrazione di tipo I ha tempi caratteristici
\begin{equation}
\tau_I\propto\frac{M_p}{\Gamma}\propto M_p\expy{-1}
\end{equation}
per pianeta $5\mearth{}$ a \SI{5}{\astronomicalunit} si ha $\tau_I=\SI{0.5}{\mega\year}$.
La migrazione di tipo II \'e caratterizzata da formazione di un gap attorno all'orbita del pianeta: la transizione tra migrazione I e II avviene se $R_H>H$ e momento torcente dovuto alla viscosit\'a del disco \'e minore del momento esercitato dal pianeta sul disco. La velocit\'a di migrazione di tipo II \'e determinata dall'evoluzione viscosa del disco:
\begin{equation}
\dot{a}_p=-\frac{3}{2}\frac{\nu}{r}
\end{equation}
Il momento esercitato sul pianeta dalla regione di corotazione pu\'o dar luogo a feedback positivo su pianeta di massa intermedia con velocit\'a radiale $\dot{a}_p$.
\end{errata}

Nel modello di Berna si determina l'evoluzione orbitale degli embrioni planetari risolvendo numericamente il problema a N corpi
\begin{align}
&\ddvec{r}_i=-G(M_*+m_i)\frac{\vec{r}_i}{r_i^3}-G\sum_{j\neq i}m_j\left[\frac{\vec{r}_i-\vec{r}_j}{|\vec{r}_i-\vec{r}_j|^3}+\frac{\vec{r}_j}{r_j^3}\right]&\intertext{con l'aggiunta di termini che descrivono l'interazione del pianeta col disco di gas (\cite{tanaka2002three},\cite{tanaka2004three})}\\
&\ddvec{r}_i^{\,m}=-\frac{\dvec{x}_i}{t_m}\\
&\ddvec{r}_i^{\,e}=-2\frac{(\dvec{r}_i\cdot\vec{r}_i)\vec{r}_i}{r_i^2t_e}\\
&\ddvec{r}_i^{\,i}=-\frac{(\dvec{x}_i\cdot\hat{z})}{t_i}\hat{z}
\end{align}
Per piccole $e$, $i$ e pianeta non abbastanza massiccio da aprire un gap attorno alla propria orbita il tempo di smorzamento di $e$, $i$ \'e molto minore del tempo di migrazione radiale. 
Per il tempo caratteristico di migrazione radiale si ha:
\begin{equation}
\tau_m=\frac{r_p}{|\dot{r}_p|}=\frac{1}{2}\frac{J_p}{\Gamma}
\end{equation}
\cite{lubow2010planet} usando l'approssimazione impulsiva, valida per $\tau_{int}\ll\tau_{orb}$, hanno determinato l'ordine di grandezza del momento torcente subito dal pianeta. Per pianeti di piccola massa e distanza di interazione $\Delta r\approx H$
\begin{equation}
|\Gamma|\propto\Sigma_p\Omega_p^2a^4(\frac{M_p}{M_*})^2(\frac{a}{H})^2
\end{equation}
tipico del regime di migrazione detto di tipo I; per piccole distanze dall'orbita del pianeta il flusso di gas presente all'interno/esterno dell'orbita del pianeta, indicato con $\dot{M}_i$ e $\dot{M}_o$, durante deflessione a U incontrando il pianeta cambia momento angolare per l'azione del pianeta che quindi subisce momento torcente uguale e contrario
\begin{equation}
T_{co}\propto\dot{M}_o(1-\frac{\dot{M}_i}{\dot{M}_o})a\Omega_pw\approx\Sigma_g\Omega_p^2w^4\TDly{r}{(\Sigma_g/B)}
\end{equation}
dove $w\approx$ \'e larghezza di regione di corotazione, B \'e la costante di Oort tale
\begin{equation}
2Br=\TDy{r}{(r^2\Omega)}
\end{equation}
e
\begin{equation}
w\propto a_p\sqrt{\frac{q}{h}}, \ h=\frac{H}{a},\ q=\frac{m_p}{M_*}
\end{equation}
Nel caso il pianeta sia abbastanza massiccio si apre un gap attorno all'orbita del pianeta; la condizione perch\'e ci\'o avvenga \'e:
\begin{equation}
\frac{M_p}{M_*}\geq C\sqrt{\frac{\nu}{a^2\Omega_p}}(\frac{H}{a})\expy{3/2},\quad C\approx1
\end{equation}
e la migrazione in questa condizione \'e detta di tipo II. Il rate di migrazione \'e determinato da evoluzione viscosa del disco con
\begin{equation}
t_{vis}\propto\frac{a^2}{\nu}\approx\frac{a^2}{\alpha cH}\approx(\frac{a}{H})^2\frac{1}{\alpha\Omega_p}
\end{equation}

\begin{workout}[Magnitude torque type I/II and corotation]
\cite{lubow2010planet}
Lubow, ida: ''Planet migration'' eq 18-25 pg 6-7.
Type I: $|T|\propto\Sigma_p\Omega_p^2a^4(\frac{M_p}{M_*})^2(\frac{a}{H})^2$ (collisionles particles)
coorbital torque: $T_{co}\propto\dot{M}_o(1-\frac{\dot{M}_i}{\dot{M}_o})a\Omega_pw\approx\Sigma_g\Omega_p^2w^4\TDly{r}{(\Sigma_g/B)}$
\cite{dittkrist2014impacts}: thermal, density structure
Type II: $t_{vis}\propto\frac{a^2}{\nu}\approx\frac{a^2}{\alpha cH}\approx(\frac{a}{H})^2\frac{1}{\alpha\Omega_p}$ and gap opening conditions
\begin{align*}
&T_c\approx\Sigma_ga^2\Omega_p\nu\\
	&T_o\geq T_c: \frac{M_p}{M_*}\geq C_g\sqrt{\frac{\nu}{a^2\Omega_p}}(\frac{H}{a})\expy{3/2}
\end{align*}
\end{workout}
%{\let\clearpage\relax\let\cleardoublepage\relax
%\chapter{N-body interactions inside proto disk}
%}


\begin{workout}[Long-term evolution: laplace equations]
The HARPS search for southern extra-solar planets-XXVIII. Up to seven planets orbiting HD 10180: probing the architecture of low-mass planetary systems
\end{workout}
