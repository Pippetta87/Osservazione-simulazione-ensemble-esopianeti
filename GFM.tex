\chapter{Accretion disk model}

\begin{workout}[Accretion disk sources]
o218, 296: disk formation, constrains from solar system
the alpha disk pg 18: theory of turbulent accretion disk
\end{workout}

\begin{workout}[Isothermal cloud collapse!]

\end{workout}


\begin{workout}[Intro a disco accrescimento]
Nei modelli globali la formazione planetaria \'e simulata partendo dalla fase finale di accrescimento di massa sulla stella centrale (per semplicit\'a considero stelle di $1\msun{}$ singole): il collasso di nube molecolare produce strutture appiattite la cui evoluzione \'e determinata dal trasporto di momento angolare verso l'esterno, l'interazione con l'oggetto centrale (campi magnetici vento stellare) e con l'ambiente circostante.
L'ipotesi su cui si basano le simulazioni considerata \'e che la componente polverosa formi corpi pi\u massicci fino a masse di frazioni di masse terrestri  ed infine accrescere gas: lo scenario di core accretion (CA).
\end{workout}

\begin{workout}[Modelli 1D disco accrescimento: struttura verticale]
auto gravitazione trascurabile quindi la struttura verticale \'e determinata dalla componente lungo z dell'attrazione del corpo centrale, profilo termico determinato da equilibrio.
\end{workout}

\begin{workout}[Descrizione trasporto momento angolare tramite parametrizzazione viscosit\'a]
Mass conservation + momentum conservation in viscous flow: angular momentum evolution. Phenomena: Shear viscosity - turbulence - MRI.
Nei modelli 1D per disco di accrescimento si parametrizza viscosit\'a tramite $\nu=\frac{\eta}{\rho}\to\alpha c_s H$: la viscosit\'a molecolare \'e troppo bassa  per trasportare all'esterno il momento angolare sui tempi-scala osservati.

Theory of turbulent accretion disk: pg 18, 8.
Flusso di x-momentum lungo y $\rho\exv{u_xu_y}$, dove considero la velocit\'a dell'elemento di fluido $(v_x+u_x,u_y)$ dove le $u$ rappresentano fluttuazioni della velocit\'a: $\sigma_{xy}=-\rho\exv{u_xu_y}$.
Ricordando la definizione di tensore degli stress e con $\vec{v}=v_x(y)\hat{x}$

\begin{equation}
\sigma_{ij}=\eta(\partial_jv_i+\partial_iv_j-\frac{2}{3}\delta_{ij}\partial_kv_k+\zeta\delta_{ij}\partial_kv_k \to \eta\TDy{y}{v_x}
\end{equation}

\begin{equation}
\PDof{t}(\rho v_i)+\PDof{x_i}(\rho v_iv_j+T_{ij})=F_i
\end{equation}
Fluttuazioni: $\sigma_{ij}=-\rho\exv{u_iu_j}$.
Viscosit\'a molecolare: $\exv{u_iu_j}=-\nu\TDy{x_j}{v_i}$.
Turbulence: Reynpold stress $\tau_{ij}=-\rho\exv{v_jv_i}$, $\exv{u_iu_j}=-\nu_T\TDy{x_j}{v_i}$
\begin{equation}
\PDy{t}{\Sigma}=3\frac{1}{r}\PDof{r}[r\expy{1/2}\PDof{r}(\nu\Sigma r\expy{1/2})]\label{eq:sigmaevol}
\end{equation}
\end{workout}

\begin{workout}[Photoievaporation: X-EUV-FUV (Alexander13: The Dispersal of Protoplanetary Disks)]
Protoplanetary disc evolution and dispersal: the implications of X-ray photoevaporation
\end{workout}

\begin{workout}[alpha prescription]
$\vec{v}=(u_r,r\Omega)$, stress tensor $\sigma_{r\phi}=-\Sigma\exv{u_ru_{\phi}}$.
Enhanced turbulent viscosity: $-\Sigma\exv{u_ru_{\phi}}=\Sigma\nu r\TDy{r}{\Omega}$, $\nu=v_TH$, $\alpha=v_T/c_s$
\end{workout}

\begin{workout}[Modello di disco di accrescimento - (Mordasini18: 4) - Introduzione descrizione fenomeni grazie a PPS]
Un modello di disco  di accrescimento usato nelle simulazioni considera l'evoluzione della densit\'a superficiale tramite l'equazione \eqref{eq:diskaccrphev-m18}

\begin{align}
&\TDy{t}{\Sigma}=\frac{1}{r}\PDof{r}[3r\expy{1/2}\PDof{r}(\nu\Sigma r\expy{1/2})]+\dot{\Sigma}_w(r)+\dot{\Sigma}_p(r)\label{eq:diskaccrphev-m18}\\
&\dot{\Sigma}_w(a)=\left\{\begin{array}{c}0\\\frac{\dot{M}_w}{2\pi(a_{max}-R_g)a}\\\end{array}\right.
\end{align}

Densit\'a superficiale iniziale:
\begin{equation}
\Sigma(a,t=0)=\Sigma_0(\frac{r}{1AU})\expy{p_g}\Exp{[-(\frac{r}{R_o})\expy{2+p_g}]}(1-\sqrt{\frac{r}{R_i}})
\end{equation}
4-Mordasini18 (Hayashi81). $p_g\approx1$ (Andrews10).
\end{workout}


\begin{workout}[Foto-evaporazione]
(Photoevaporation: veras armitage 2003, Alexander13, Mordasini12. Internal/External).
EUV($E\approx13.6eV$): ionization, FUV($E\approx6-13.6eV$): dissociation, X-ray
\end{workout}

\section{Formazione planetesimi}

Protoplanetary disk and their evolution pg 29
Protoplanetary dust: Apai Lauretta - particles dynamics pg 100

Assumendo conversione completa di polvere in planetesimi, la densit\'a superficiale di planetesimi \'e
\begin{equation}
\Sigma_p(t=0,r)=f_{dg}\eta_{ice}\Sigma_g(t=0,r)
\end{equation}

\begin{workout}[Particle-gas dynamics. dust midplane sedimantation]
Weidenschilling cuzzi 06: Particle-gas dynamics and primary accretion, Apai Lauretta Protoplanetary Dust pg 100
Stopping time $t_s=\frac{\rho_sa}{\rho c_s}$ ($a<\lambda_g$). L'evoluzione dinamica delle particelle solide \'e determinata flussi macroscopici dovuti all'evoluzione del disco, gas-drag, settling
\end{workout}

\section{Accrescimento pianetsemi: formazione proto-pianeti.}

La massa dei core varia secondo
\begin{align}
&\dot{M}_c=\Omega\Sigma_pR^2_{capt}F_G\\
&\dot{M}_c=\Omega\Sigma_pR^2_{capture}F_G
\end{align}

\begin{workout}[Accrescimento pianetesimi: orderly, runaway, oligarchic]
The growth of planetary embryos:  orderly, runaway, or oligarchic? (Rafikov 2002)
\begin{align}
\TDy{t}{M_e}\approx\pi R_e^2\Omega mN\frac{v}{v_z}[1+\frac{2GM_e}{R_ev^2}]
\end{align}
Accrescimento ordinato: $\frac{1}{M_e}\TDy{t}{M_e}\propto M_e\expy{-1/3}$. Accrescimento runaway: $\frac{1}{M_e}\TDy{t}{M_e}\propto M_e\expy{1/3}$. Accrescimento oligarchico: $\frac{1}{M_e}\TDy{t}{M_e}\propto M_e\expy{-1/3}$.
\end{workout}


\begin{workout}[Protoplanets accretion of solids]
Other refs: Planet formation coagulation: focus on U,N (Goldreich 04), Final stages of planets formation (Goldreich 04), Formation of giant planets core: evaluating key processes (levison 09).
Safronov69: $\dot{M}_c=\Omega\Sigma_pR^2_{capture}F_G$, $R_{capture}$ larger than core radius due to gas drag, $F_G(e,i)$ gravitational focus (give rise to different growth regimes: runaway, oligarchic, orderly).
Runaway growth until protoplanets $100-1000km$ (dep on position).
Oligarchic growth: velocity of planetesimal raised by viscous stirring (gravitational scattering)/ damped by gas (until disk present).
\end{workout}

\begin{workout}[Planetesimal dynamics: viscous stirring]
The origin of anisotropic velocity dispersion of particles in a disc potential (Ida, Makino 93, Stewart-Wetherill 1988 Ida1990)
\end{workout}

\chapter{Accrescimento gas}

Planet formation models: the interplay with the planetesimal disc (Fortier 2013)

\chapter{Migrazione}
