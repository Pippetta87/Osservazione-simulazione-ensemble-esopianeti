{\let\clearpage\relax\let\cleardoublepage\relax
\chapter{Simulazione polazioni planetarie}
}

\section{Distribuzione iniziale planetesimi}
Assumendo conversione completa di polvere in planetesimi, la densit\'a superficiale di planetesimi \'e
\begin{equation}
\Sigma_p(t=0,r)=f_{dg}\eta_{ice}\Sigma_g(t=0,r)
\end{equation}

\begin{workout}[Initial planetesimal distro]
Dust converted early everywhere fully-efficient (mordasini18: pg 12),Thommes pg8)
\begin{align*}
&\Sigma_p(t=0,r)=f_{dg}\eta_{ice}\Sigma_g(t=0,r)\\
&\dot{\Sigma}_p(r)=-\frac{1}{2\pi aB_LR_H}\dot{M}_c
\end{align*}
Icelines Mordasini 1141: inside where T exceeds sublimation
\end{workout}

\section{Accrescimento dei protopianeti e gas}

La massa dei core varia secondo
\begin{align}
&\dot{M}_c=\Omega\Sigma_pR^2_{capt}F_G\\
&\dot{\Sigma}_p(r)=-\frac{1}{2\pi aB_LR_H}\dot{M}_c
\end{align}

Si risolvono numericamente le equazioni di conservazione di massa, equilibrio idrostatico, conservazione/trasporto energia 1D

\begin{workout}[Attached phase: boundary conditions]
\begin{align}
&P_{pl}=P_{Neb}\\
&T_{Pl}=T_{Neb}\\
&R_{Pl}=Min(R_H,R_H)
\end{align}
\end{workout}


\begin{workout}[Detached phase: transition condition and boundary condition]
Transition attached/detched phase $\approx10\mearth{}$.
Accretion shock for free-falling materials from Hill radius (more realistic circumplanetary disk: Papailoizou Nelson 05)
\begin{align}
&\dot{M}_{XY}^{max}\\
&v_{ff}^2=2GM(\frac{1}{R}-\frac{1}{R_H})\\
&P=P_{neb}+\frac{\dot{M}_{XY}}{4\pi R^2}v_{ff}+\frac{2g}{3\kappa}\\
&\tau=max(\rho_{neb}\kappa_{neb}R,2/3)\\
&T^4_{int}=\frac{3\tau L_{int}}{8\pi\sigma R^2},\ T^4=(1-A)T_{neb}^4+T_{int}^4
\end{align}
$R\approx1.5-5\rjupiter{}$ depending on entropy: THE PLANETARY ACCRETION SHOCK:I. FRAMEWORK FOR RADIATION-HYDRODYNAMICAL SIMULATIONS AND FIRST RESULTS (m17), Characterization of exoplanets from their formation III: The statistics of planetary luminosities (m17)
Bondi accretion rate: $\dot{M}_{e, Bondi}\approx\frac{\Sigma}{H}(R_H/3)^3\Omega$ or viscous accretion rate $\dot{M}_{e, visc}\approx f_{lub}3\pi\nu\Sigma_g$
\end{workout}

\begin{workout}[Higher mass gap formation reduces accretion rate]
\begin{equation}
f_{va04}=1.668(\frac{M_p}{\mjupiter{}})\expy{1/3}\exp{-\frac{M_p}{1.5\mjupiter{}}}+0.04
\end{equation}
\end{workout}


{\let\clearpage\relax\let\cleardoublepage\relax
\chapter{Funzione di massa sistemi planetari}
}

{Impact of planet migration model on planetary populations: effects of saturation, cooling and stellar irradiation}
Outward migration helps some planets to become massive, accumulation zone at certain semiaxis, at what mass corotation saturate?
TOdo: mifration model
