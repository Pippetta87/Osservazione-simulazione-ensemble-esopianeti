{\let\clearpage\relax\let\cleardoublepage\relax
\chapter{Simulazione polazioni planetarie}
}

\section{Distribuzione iniziale planetesimi}
Assumendo conversione completa di polvere in planetesimi, la densit\'a superficiale di planetesimi \'e
\begin{equation}
\Sigma_p(t=0,r)=f_{dg}\eta_{ice}\Sigma_g(t=0,r)
\end{equation}

\begin{workout}[Initial planetesimal distro]
Dust converted early everywhere fully-efficient (mordasini18: pg 12),Thommes pg8)
\begin{align*}
&\Sigma_p(t=0,r)=f_{dg}\eta_{ice}\Sigma_g(t=0,r)\\
&\dot{\Sigma}_p(r)=-\frac{1}{2\pi aB_LR_H}\dot{M}_c
\end{align*}
Icelines Mordasini 1141: inside where T exceeds sublimation
\end{workout}

\section{Accrescimento dei protopianeti}

La massa dei core varia secondo
\begin{align}
&\dot{M}_c=\Omega\Sigma_pR^2_{capt}F_G\\
&\dot{\Sigma}_p(r)=-\frac{1}{2\pi aB_LR_H}\dot{M}_c
\end{align}

{\let\clearpage\relax\let\cleardoublepage\relax
\chapter{Funzione di massa sistemi planetari}
}

{Impact of planet migration model on planetary populations: effects of saturation, cooling and stellar irradiation}
Outward migration helps some planets to become massive, accumulation zone at certain semiaxis, at what mass corotation saturate?
TOdo: mifration model
