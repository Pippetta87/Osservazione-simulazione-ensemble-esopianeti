{\let\clearpage\relax\let\cleardoublepage\relax
\chapter{Modelli di formazione globali e semplificazioni introdotte}
}

\begin{workout}[Ref PPS]
Towarddeterminist model of planetary formation iV: effectsof type I migration
									: accumulation neare iceline
									: dynamical intaraction and coagulation of multiple rocky embrios (isolation mass, semi-analytic vs n-body)
									: eccentricity distribution of gas giant
Theoretical models of planetary system formation: mass vs. semi-major axis	(alibert carron 13)			Lecture 15 -planetary ...
Modelling planetary system formation with N-body simulation (11)
Global model of planet formation and evolution
Planetary population synthesis
planet population synthesis					
\end{workout}

\section{Modello disco di accrescimento e distribuzione condizioni iniziali}

\begin{workout}[modello globale: evoluzione dal semplice al complesso]
Global model of planets formation and evolution: more detailed submodels
\end{workout}

Il modello pi\'u semplice di disco di accrescimento usa distribuzione esponenziale per andamento densit\'a superficiale e temperature, assumendo disco otticamente sottile, e relazione $L_*\propto M_*^4$ di sequenza principale. Le lacune di questo modello sono che i dischi sono otticamente spessi con transizioni nell'opacit\'a, non \'e presente evoluzione temporale consistente.

Modelli pi\'u recenti  risolvono l'equazione per l'evoluzione viscosa:
\begin{align}
&\TDy{t}{\Sigma}=\frac{1}{r}\PDof{r}[3r\expy{1/2}\PDof{r}(\nu\Sigma r\expy{1/2})]+\dot{\Sigma}_w(r)+\dot{\Sigma}_p(r)\label{eq:diskaccrphev-m18}\\
&\dot{\Sigma}_w(a)=\left\{\begin{array}{c}0\\\frac{\dot{M}_w}{2\pi(a_{max}-R_g)a}\\\end{array}\right.
\end{align}
con $\dot{Sigma}_w$ contributo della foto-evaporazione,  $\dot{Sigma}_p$ contributo di accrescimento di gas dei pianeti e con densit\'a superficiale iniziale:
\begin{equation}
\Sigma(a,t=0)=\Sigma_0(\frac{r}{1AU})\expy{p_g}\Exp{[-(\frac{r}{R_o})\expy{2+p_g}]}(1-\sqrt{\frac{r}{R_i}})
\end{equation}
in i parametri possono essere scelti casualmente secondo una distribuzione di probabilit\'a ricavata dalle osservazioni.

\begin{workout}[viscosit\'a disco]
Nella popolazione planetaria simulata $\alpha$ \'e fissato sulla base delle osservazioni  ed \'e omogeneo e costante.
\end{workout}

\begin{workout}[Modello di disco di accrescimento - (Mordasini18: 4) - Introduzione descrizione fenomeni grazie a PPS]
Refs: garaud lin 07, chiang goldreich 97
Un modello di disco  di accrescimento usato nelle simulazioni considera l'evoluzione della densit\'a superficiale tramite l'equazione \eqref{eq:diskaccrphev-m18}
\begin{align}
&\TDy{t}{\Sigma}=\frac{1}{r}\PDof{r}[3r\expy{1/2}\PDof{r}(\nu\Sigma r\expy{1/2})]+\dot{\Sigma}_w(r)+\dot{\Sigma}_p(r)\label{eq:diskaccrphev-m18}\\
&\dot{\Sigma}_w(a)=\left\{\begin{array}{c}0\\\frac{\dot{M}_w}{2\pi(a_{max}-R_g)a}\\\end{array}\right.
\end{align}
Densit\'a superficiale iniziale:
\begin{equation}
\Sigma(a,t=0)=\Sigma_0(\frac{r}{1AU})\expy{p_g}\Exp{[-(\frac{r}{R_o})\expy{2+p_g}]}(1-\sqrt{\frac{r}{R_i}})
\end{equation}
4-Mordasini18 (Hayashi81). $p_g\approx1$ (Andrews10).
\end{workout}


\section{Distribuzione iniziale planetesimi}

Assumendo conversione completa di polvere in planetesimi, la densit\'a superficiale di planetesimi \'e
\begin{equation}
\Sigma_p(t=0,r)=f_{dg}\eta_{ice}\Sigma_g(t=0,r)
\end{equation}
con $f_{dg}$ rapporto gas/polvere (circa metallicit\'a) \'e una parametro casuale con distribuzione che riflette distribuzione di metallicit\'a stellari, $\eta_{ice}$ tiene conto della discontinuit\'a nella distribuzione superficiale di solidi all'iceline.

\begin{workout}[Initial planetesimal distro]
Dust converted early everywhere fully-efficient (mordasini18: pg 12),Thommes pg8)
\begin{align*}
&\Sigma_p(t=0,r)=f_{dg}\eta_{ice}\Sigma_g(t=0,r)\\
&\dot{\Sigma}_p(r)=-\frac{1}{2\pi aB_LR_H}\dot{M}_c
\end{align*}
Icelines Mordasini 1141: inside where T exceeds sublimation

{Embryo starting position}
Usually a distro uniform in log(a): relative spacing of few Hill spheres (Kokubo Ida 10)
Ida, lin10: asymptotical isolation mass ''Toward a Deterministic Model of Planetary Formation VI'',
Trapped evolution model: Hasegawa pudritz 11, Cridland 16

\end{workout}

\section{Accrescimento dei protopianeti e gas}

La massa dei core varia secondo
\begin{align}
&\dot{M}_c=\Omega\Sigma_pR^2_{capt}F_G\\
&\dot{\Sigma}_p(r)=-\frac{1}{2\pi aB_LR_H}\dot{M}_c
\end{align}

Si risolvono numericamente le equazioni di conservazione di massa, equilibrio idrostatico, conservazione/trasporto energia 1D

\begin{workout}[Attached phase: boundary conditions]
\begin{align}
&P_{pl}=P_{Neb}\\
&T_{Pl}=T_{Neb}\\
&R_{Pl}=Min(R_H,R_H)
\end{align}
\end{workout}


\begin{workout}[Detached phase: transition condition and boundary condition]
Transition attached/detched phase $\approx10\mearth{}$.
Accretion shock for free-falling materials from Hill radius (more realistic circumplanetary disk: Papailoizou Nelson 05)
\begin{align}
&\dot{M}_{XY}^{max}\\
&v_{ff}^2=2GM(\frac{1}{R}-\frac{1}{R_H})\\
&P=P_{neb}+\frac{\dot{M}_{XY}}{4\pi R^2}v_{ff}+\frac{2g}{3\kappa}\\
&\tau=max(\rho_{neb}\kappa_{neb}R,2/3)\\
&T^4_{int}=\frac{3\tau L_{int}}{8\pi\sigma R^2},\ T^4=(1-A)T_{neb}^4+T_{int}^4
\end{align}
$R\approx1.5-5\rjupiter{}$ depending on entropy: THE PLANETARY ACCRETION SHOCK:I. FRAMEWORK FOR RADIATION-HYDRODYNAMICAL SIMULATIONS AND FIRST RESULTS (m17), Characterization of exoplanets from their formation III: The statistics of planetary luminosities (m17)
Bondi accretion rate: $\dot{M}_{e, Bondi}\approx\frac{\Sigma}{H}(R_H/3)^3\Omega$ or viscous accretion rate $\dot{M}_{e, visc}\approx f_{lub}3\pi\nu\Sigma_g$
\end{workout}

\begin{workout}[Higher mass gap formation reduces accretion rate]
\begin{equation}
f_{va04}=1.668(\frac{M_p}{\mjupiter{}})\expy{1/3}\exp{-\frac{M_p}{1.5\mjupiter{}}}+0.04
\end{equation}
\end{workout}

\section{Popolazione sintetiche}

Considero o risulati di alcuni modelli di formazione planetari (\cite{mordasini2018planetary}). 

\begin{figure}
\includegraphics[width=0.9\textwidth,keepaspectratio]{track1}
\caption{Da \cite{mordasini2018planetary}. Formazione di un sistema planetario nel diagramma $a-M$: alla scomparsa del disco protoplanetario si hanno 2 pianeti giganti, un nettuniano caldo e 3 pianeti terrestri. La scala di colore indica la composizione $\frac{M_e}{M_c}$.}\label{fig:track1}
\end{figure}

\begin{figure}
\includegraphics[width=0.9\textwidth,keepaspectratio]{ma-synth}
\caption{Da \cite{mordasini2018planetary}. Simulazione popolazione planetaria di 504 sistemi Punti rossi: pianeti giganti con $M_e/M_c>1$; blu hanno accresciuto core oltre la iceline, verdi all'interno dell'iceline .}
\end{figure}

\begin{figure}
\includegraphics[width=0.9\textwidth,keepaspectratio]{MaLR-freq-synth}
\caption{Da \cite{mordasini2018planetary}. }
\end{figure}

\begin{figure}
\includegraphics[width=0.9\textwidth,keepaspectratio]{giant-Zsynth}
\caption{Da \cite{mordasini2018planetary}. }
\end{figure}

\begin{figure}
\includegraphics[width=0.9\textwidth,keepaspectratio]{envelopecoresynth}
\caption{Da \cite{mordasini2018planetary}. }
\end{figure}

{\let\clearpage\relax\let\cleardoublepage\relax
\chapter{Confronto semi-quantitativo tra caratteristiche popolazioni sintetiche e osservate}
}

{\let\clearpage\relax\let\cleardoublepage\relax
\chapter{Raffinamento dei modelli}
}


\begin{workout}[Effects of saturation, cooling and irradiation.]
Impact of planet migration model on planetary populations: effects of saturation, cooling and stellar irradiation (??)
Outward migration helps some planets to become massive, accumulation zone at certain semiaxis, at what mass corotation saturate?
Migration of protoplanets in radiative disks
\end{workout}
