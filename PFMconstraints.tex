{\let\clearpage\relax\let\cleardoublepage\relax
\chapter{Sistema solare}
}



\begin{workout}[Sistema solare - sezione del capitolo dati/vincoli osservativi: testcostruzione modelli planetarii e dettagli]

\end{workout}

\begin{workout}[et\'a corpi sistema solare]
helioseismology and meteoritic dating
\end{workout}

\begin{workout}[Densit\'a corpi sistema solare]

\end{workout}

\begin{workout}[Full packing/spacing]

\end{workout}

\begin{workout}[Modello pianeta gassoso: Giove]

\end{workout}

\begin{workout}[Noble gas enrichment]

\end{workout}


\begin{workout}[Orbit of terrestrial planets]
Constrain for core accretion model ofgiant planets: dynamical shakeup Nagasawa 05 Thommes08c
\end{workout}


\begin{workout}[Planetesima migration: pluto eccentricity, large population of Kuiper belt object]

\end{workout}



\begin{workout}[Planet obliquity:]
Inconsistent with isotropic distro (Tremaine 1991). Randomly directed component of spin angular momentum (Kokubo Ida 07) cause large then observed i,e (Harris Ward 82).
Planetesimal/protoplanet collision would imply stachastic rather than smooth accretion.
Excittion of giant planet spin obliquity if spin axis precession freq pass through resonance with orbital prec freq during migration.
Uranus: tilt due to collision or migration: Bergstralh 91 or Bou\'e Laskar 10.
Non of obliquity of terrestrial planet are belived primordial (secular orbit perturbation)+tidal dissipation.
Effects of inhomogenous infall on obliquity (tremaine 91, bate 10)
\end{workout}



{\let\clearpage\relax\let\cleardoublepage\relax
\chapter{Alcune caratteristiche della popolazione di esopianeti}
}

La popolazione degli esopianeti osservata riflette quella reale tramite gli effetti di selezione dovuti alla probabilit\'a direvilamento della tecnica particolare. Le osservazioni pi\'u numerose sono basate sull'osservazione della velocit\'a lungo la linea di vista dovuto al moto della stella attorno al baricentro del sistema stella pianeta (RV) o differenza nella luminosit\'a dovuto al transito del pianeta tra stella e osservatore.

\section{Popolazione esopianeti rilevati tramite RV}

Se l'asse z \'e lungo la linea di vista la posizione della stella \'e $z=r(t)\sin{i}\sin{(\omega+\nu)}$ e per la velocit\'a radiale:
\begin{align}
&v_r=\dot{z}=K[\cos{(\omega+\nu)}+e\cos{\omega}]\label{eq:vrsignal}\\
&K=\frac{2\pi}{P}\frac{a_*\sin{i}}{(1-e^2)\expy{1/2}}=(\frac{2\pi G}{P})\expy{1/3}\frac{M_p\sin{i}}{(M_p+M_*)\expy{2/3}}\frac{1}{(1-e^2)\expy{1/1}}
\end{align}
dove $K$ \'e la semi-ampiezza.


\begin{workout}[Simulated radialo detection limit]
Perrymann pg 34 fig 2.25 d
paragrafo detectability and selection effects pg 14 perryman
\end{workout}

\begin{workout}[Radial velocity: dtectability and selection effects (pg 14)]
fig 2.6,fig 2.18
\end{workout}

\begin{figure}[!ht]
\includegraphics[width=0.7\textwidth]{}
\end{figure}

\begin{workout}[Radial velocity: keplerian observables (Appendice RV)]
La coordinata z della stella $z=r(t)\sin{i}\sin{(\omega+\nu)}$ e per la velocit\'a radiale
\begin{align}
&v_r=\dot{z}=K[\cos{(\omega+\nu)}+e\cos{\omega}]\label{eq:vrsignal}\\
&K=\frac{2\pi}{P}\frac{a_*\sin{i}}{(1-e^2)\expy{1/2}}=(\frac{2\pi G}{P})\expy{1/3}\frac{M_p\sin{i}}{(M_p+M_*)\expy{2/3}}\frac{1}{(1-e^2)\expy{1/1}}
\end{align}
Le osservabili $e$, $P$, $t_p$, $\omega$, $K=f(a,e,P,i)$ sono fittate per ogni pianeta.
La velocit\'a radiale della stella \'e
\begin{equation}
v_r(t)=K[\cos{(\omega+\nu(t)}+e\cos{\omega}]+\gamma+d(t-t_0)
\end{equation}
Per sistemi multipli consideriamo la sovrapposizione lineare di $n_p$ termini \eqref{eq:vrsignal}: viene sottratto il segnale Kepleriano dominante fino a che resta solo rumore.
\end{workout}

\begin{workout}[Radial velocity: keplerian observables (Appendix RV)]
La coordinata z della stella $z=r(t)\sin{i}\sin{(\omega+\nu)}$ e per la velocit\'a radiale
\begin{align}
&v_r=\dot{z}=K[\cos{(\omega+\nu)}+e\cos{\omega}]\label{eq:vrsignal}\\
&K=\frac{2\pi}{P}\frac{a_*\sin{i}}{(1-e^2)\expy{1/2}}=(\frac{2\pi G}{P})\expy{1/3}\frac{M_p\sin{i}}{(M_p+M_*)\expy{2/3}}\frac{1}{(1-e^2)\expy{1/1}}
\end{align}
Le osservabili $e$, $P$, $t_p$, $\omega$, $K=f(a,e,P,i)$ sono fittate per ogni pianeta.
La velocit\'a radiale della stella \'e
\begin{equation}
v_r(t)=K[\cos{(\omega+\nu(t)}+e\cos{\omega}]+\gamma+d(t-t_0)
\end{equation}
\end{workout}
Per sistemi multipli consideriamo la sovrapposizione lineare di $n_p$ termini \eqref{eq:vrsignal}: viene sottratto il segnale Kepleriano dominante fino a che resta solo rumore.
\begin{workout}[Radial velocity: dtectability and selection effects (pg 14)]
fig 2.6,fig 2.18
\end{workout}


\section{Popolazione esopianeti rilevati tramite transito}

pg 256 
\begin{workout}[M-R diagram]
fig11.2, 11,4 (EOS), 11.7, 11.8 (M-R)
Ternary diagram 11.9, M-R realtion
\end{workout}

pg 103
\begin{workout}[transiti: osservabilie probabilit\'a di rilevamento]
pg 117, eccentric orbit pg 121-122
\end{workout}
\begin{workout}[Transiti: Propriet\'a dei pianeti a transito]
pg 143, fig 6.33, 6.34, 6.35
Mass-radius relation: chabrier 09
\end{workout}


\begin{workout}[Atmosphere and starting conditions]
fig 11.12
\end{workout}

\begin{workout}[Astrometry]
L'ellisse percorsa dalla stella attorno al baricentro stella pianeta ha ampiezza angolare
\begin{equation}
\alpha=\frac{M_p}{M_p+M_*}a\approx\frac{M_p}{M_*}a=(\frac{M_p}{M_*})(\frac{a}{1AU})(\frac{d}{1pc})\expy{-1}\si{\arcsec}
\end{equation}

Variation in photocentre, radial velocity and total flux
Astrometry: Photonoise $\sigma_{ph}=\frac{\lambda}{4\pi D}\frac{1}{S/N}$
pg 65

\end{workout}

\begin{workout}[Timing: ritardo segnale]
Pg 75
$\tau_p=\frac{1}{c}\frac{a\sin{i}M_p}{M_*}$
\end{workout}

\begin{workout}[Microlensing]
$\frac{M_L}{\msun{}}=0.123\frac{\theta_E^2}{\upvarpi_{rel}}$
Pg 83. fig 5.7. Applications pg 93
(Gould 10: ''fREQUENCY OF SOLAR LIKE SYSTEM AND ICE DAS GIANT BEYOND SNOW LINE FROM MICROLENSING...'')
\end{workout}

\begin{workout}[Imaging: star-planet brightness ration]
pg 149
\begin{equation*}
\frac{f_p()}{f_*()}=
\end{equation*}
\end{workout}
\begin{workout}[Imaging: Observational limits]
ground based (pg 158) space based imaging (pg162)
fig 7.10
\end{workout}


{\let\clearpage\relax\let\cleardoublepage\relax
\chapter{YSO: distribuzione propriet\'a dischi protoplanetari}
}

