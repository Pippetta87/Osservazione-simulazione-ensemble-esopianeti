\begin{workout}[vincoli e condizioni iniziali: less modell dependent that I can]
\begin{itemize}
\item SISTEMA SOLARE
\item[-] Datazione: tempi scala formazione
\item[-] struttura orbitale e composizione sistema solare: MMSN
\item[-] composizione pianeti
\item[-] modelli sistema solare: walsh 11 (grand tack) (with comparison with exoplanets??)
\item ESOPIANETI
\item[-] Survey osservativi e completezza
\item[-] diversity: distribuzione di massa-distribuzione bimodale per piccole/grandi masse con minimo a $30\mearth$, crescita esponenziale a basse masse, secondo picco a masse gioviane
\item[-] distribuzione  raggio/densit\'a: eos, evaporazione, composizione
\item[-] Propriet\'a statistiche esopianeti: ??
\item Dischi protopalnetarii
\item[-] Osservazioni: distribuzione di massa, tempo di vita e condizioni iniziali e metallicit\'a stellare.
\item[-] Modello di disco di accrescimento
\end{itemize}
\end{workout}
% trim={5cm 0 0 0},clip
%vincoli e condizion iniziali per pps??
{\let\clearpage\relax\let\cleardoublepage\relax
\chapter{Sistema solare}
}

\begin{workout}[Vincoli ai modelli di formazione: sistema solare]
Struttura e composizione pianeti: orbite su un piano quindi disco accrescimento, chiarire et\'a del sistema solare/Sole e tempi scala CA/GI; tempi scala formazione e migrazione giove saturno: late haevy bombardament. Massa minima disco protoplanetario. Modello pianeti gassosi per determinare profilo densit\'a e quindi composizione.
\end{workout}

Il sistema solare \'e il punto di partenza delle teorie di formazione planetaria; inoltre si hanno informazioni accurate sui pianeti e loro orbite, sulla popolazione di corpi minori e, tramite datazione radiochimica, si hanno riferimenti temporali sui tempi di solidificazione.

L'et\'a del Sole \'e definita come tempo in sequenza principale: tecniche eliosismologiche determinano un'et\'a di \SI{4.57+-0.11}{\giga\year} (\cite{bonanno2002age}) mentre l'analisi chimica dei meteoriti \SI{4.57+-0.07}{\giga\year} (\cite{bahcall1995solar}).
La core terrestre si \'e formato completamente circa \SIrange{10}{29}{\mega\year} dopo i meteoriti.

\begin{workout}[rivedere ''core terrestre quando si \'e formato]
''A short timescale for terrestrial planet formation from Hf-W chronometry of meteorites''
\end{workout}

\begin{wrapfigure}[6]{r}{0.5\textwidth}
\includegraphics[trim={0cm 5cm 0 0cm},clip, width=0.5\textwidth]{Snebuladensity}\caption{Andamento minimum mass solar nebula (MMSN). Da \cite{weidenschilling1977distribution}.}\label{fig:Snebuladensity}
\end{wrapfigure}

Un limite inferiore alla massa del disco protoplanetario si ottiene  considerando la massa presente nei pianeti e nella fascia asteroidale arricchita di H/He per avere composizione solare e distribuendo questa massa in un anello di spessore proporzionale al raggio di Hill e centrato sulle orbite (\cite{weidenschilling1977distribution}): si ha l'andamento della densit\'a superficiale, proporzionale a $e\expy{-3/2}$, mostrato in figura \ref{fig:Snebuladensity}; \cite{hayashi1981structure} ha determinato l'andamento della densit\'a di gas, ghiaccio e rocce:
%$\Sigma\propto r\expy{-3/2}$.
\begin{align}
&\Sigma_{rock}=7.1(\frac{a}{\si{\astronomicalunit}})\expy{-3/2}\si{\gram\per\square\cm}:\ \frac{a}{\si{\astronomicalunit}}=0.35-2.7\\
&\Sigma_{rock+ice}=30(\frac{a}{\si{\astronomicalunit}})\expy{-3/2}\si{\gram\per\square\cm}:\ \frac{a}{\si{\astronomicalunit}}=2.7-36\\
&\Sigma_{gas}=\num{e3}(\frac{a}{\si{\astronomicalunit}})\expy{-3/2}\si{\gram\per\square\cm}:\ \frac{a}{\si{\astronomicalunit}}=0.35-36
\end{align}

\begin{workout}[Massa disco proto solare: weidenschilling/Hayashi]
W: $M\approx0.01-0.1\msun$, H: $M\approx0.013\msun$
\end{workout}

\begin{workout}[accrescimento di gas su nucleo solido prevede arricchimento di metalli ]

\end{workout}

%La formazione per accrescimento di gas su nucleo solido prevede arricchimento di metalli rispetto alla composizione di partenza.
La composizione dei pianeti giganti \'e determinata grazie a modelli di struttura planetaria. Vincoli alla distribuzione di massa interna sono imposti dalla misura dei momenti di multipolo del potenziale gravitazionale: risolvendo un modello di pianeta che rientri nei vincoli osservativi si determinano la massa del core e la metallicit\'a dell'inviluppo gassoso dei 4 pianeti gassosi (vedi  tabella \ref{tab:JSUNcomp}).

\begin{table}[!htb]
    \begin{minipage}{.5\linewidth}
      \centering
        \begin{tabular}{|ccc|}
\hline
&\parbox{1.5cm}{Jupiter $317.8\mearth{}$}&\parbox{1.5cm}{Saturn $95.1\mearth{}$}\\
  $M_c$&$0-11\mearth{}$&$9-22\mearth{}$\\
\hline
$M_Z$&$1-39\mearth{}$&$1-8\mearth{}$\\
\hline
$M_Z^{tot}$&$8-39\mearth{}$&$13-28\mearth{}$\\
\hline
$Z/Z_{\odot}$&$1-6$&$6-14$\\
\hline
 \end{tabular}
    \end{minipage}%
    \begin{minipage}{.5\linewidth}
      \centering
        \begin{tabular}{|ccc|}
\hline
&\parbox{1.5cm}{Uranus $14.5\mearth{}$}&\parbox{1.5cm}{Neptune $17.1\mearth{}$}\\
\hline
$M_{rock}$&$3.7\mearth{}$&$4.2\mearth{}$\\
\hline
$M_{ice}$&$9.3\mearth{}$&$10.7\mearth{}$\\
\hline
$M_{H/He}$&$1.5\mearth{}$&$2.2\mearth{}$\\
\hline
        \end{tabular}
    \end{minipage} 
    \caption{Composizione Giove, Saturno, Urano e Nettuno. Da \cite{baraffe2009physical}.}\label{tab:JSUNcomp}
\end{table}

Per riprodurre alcune importanti caratteristiche del sistema solare \'e stato necesaario considerare fenomeni di migrazione:
\begin{itemize}
\item  il modello di Nizza del sistema solare (\cite{tsiganis2005origin}) riproduce l'eccentricit\'a e l'inclinazione dei pianeti giganti considerando l'attraversamento della risonanza $2:1$ tra Giove e Saturno dovuta a migrazione in disco di planetesimi (\cite{hahn1999orbital})
\item considerando la migrazione prodotta da interazione gravitazionale tra pianeti giganti e gas del disco protoplanetario (\cite{lin1986tidal}) lo scenario del grande Tack (vedi \cite{walsh2011low}) considera la formazione di Giove e sua migrazione verso l'interno finch\'e a \SI{1.5}{\astronomicalunit} la cattura di Saturno in risonanze $2:3$ non causa inversione della migrazione; la distanza di inversione della migrazione di Giove \'e determinata da simulazioni numeriche che mostrano che una distribuzione di planetesimi troncata a 1 AU (\cite{hansen2009formation}) riproduce la struttura orbitale dei pianeti terrestri.
\end{itemize}

%riproduce caratteristiche orbitali dei pianeti giganti considerando la loro migrazione in una distribuzione di planentesimi di $30-50\mearth{}$ ; 
%la presenza di numerosa popolazione di planetesimi \'e corroborata dalla craterizzazionedei pianeti terrestri e corpi minori come dall'esistenza di numerosi oggetti di piccole dimensioni superstiti.

\begin{workout}[Cosa dire dei modelli planetarii: core giganti gassosi: modelli planetari e raffronto isolation mass in MMSN]
parte relativa ai momenti multipolo gravitazionale. 
In hydrostatic equilibrium the external gravitational potential is
\begin{align}
&V(r,\theta)=\frac{GM_p}{r}[1-\sum_1^{\infty}(\frac{R_+}{r})^{2n}J_{2n}P_{2n}(\theta)]\\
&J_{2n}=-\frac{1}{M_pR_+^{2n}}\int\rho(r) r\expy{2n}P_{2n}(\theta)\,d^3r
\end{align}
zharkov trubitsyn 74
Noble gas enrichment: ''FORMATION OF GIANT PLANETS ? AN ATTEMPT IN MATCHING OBSERVATIONAL CONSTRAINTS''
Degree of super-adiabaticity: Struttura interna (assimption for pre-runaway accretion)
See: characterization of exoplanets from their formation (pg4)
Degree of super-adiabaticity: baraffe 02, Rafikof 07
\end{workout}

\begin{workout}[Planetesimal and late heavy bombardament]
??
\end{workout}

\begin{workout}[Ages]
Stellar age: isochrone fitting: Jorgensen Lindegren 05.
Earth age pg 297; allegre 95.
et\'a corpi sistema solare (age and chronology): age of the sun  $4.57Gyr$ (Bahcall95); CAI: $4567Myr$ +3Myr chondrule
\end{workout}

\begin{workout}[few well separeted homogeneous region: Struttura interna (assimption for pre-runaway accretion)]
See: characterization of exoplanets from their formation (pg4)
Baraffe 07: composition gradients, core dissolution: Wilson, Militzer 12, planetesimal deconstruction: mordasini 05, multiconvective layers: Leconte chabrier 12.
\end{workout}

\begin{workout}[Solar system orbital properties]
Full packing/spacing

Planetesima migration: pluto eccentricity, large population of Kuiper belt object

Orbit of terrestrial planets]
Constrain for core accretion model ofgiant planets: dynamical shakeup Nagasawa 05 Thommes08c

Planet obliquity:
Schlichting sari 07: The effect of Semi-Collisional Accretion on Planetary Spins
Inconsistent with isotropic distro (Tremaine 1991). Randomly directed component of spin angular momentum (Kokubo Ida 07) cause large then observed i,e (Harris Ward 82).
Planetesimal/protoplanet collision would imply stachastic rather than smooth accretion.
Excittion of giant planet spin obliquity if spin axis precession freq pass through resonance with orbital prec freq during migration.
Uranus: tilt due to collision or migration: Bergstralh 91 or Bou\'e Laskar 10.
Non of obliquity of terrestrial planet are belived primordial (secular orbit perturbation)+tidal dissipation.
Effects of inhomogenous infall on obliquity (tremaine 91, bate 10)
\end{workout}


{\let\clearpage\relax\let\cleardoublepage\relax
\chapter{Caratteristiche della popolazione di esopianeti osservati}
}

\begin{wrapfigure}[10]{l}{0.65\textwidth}
	\includegraphics[trim={0cm 0 0 0},clip, width=0.65\textwidth]{obsMa}
	\caption{Diagramma massa-distanza degli esopianeti in ''The extrasolar planet encyclopedia''. Rosso, celeste, magenta e verde sono pianeti rivelati tramite RV, T, osservazione diretta e microlensing. Da \cite{mordasini2018planetary}.}\label{fig:Maplot}
\end{wrapfigure}

Il miglioramento delle capacit\'a osservative ha permesso l'osservazione di centinaia di esopianeti (vedi \cite{schneider2011defining} per una lista aggiornata): la distribuzione dei pianeti nel diagramma $a-M_p$ rappresenta la variet\'a della popolazione planetaria osservata. Per un confronto con modello di formazione planetaria \'e necessario correggere, in maniera statistica, i bias osservativi dovuti alla tecnica usata per rivelare il pianeta, alle caratteristiche dello strumento e della stella che lo ospita. Adesso considero le caratteristiche dei pianeti come risultano da survey che usano le tecniche di velocit\'a radiale e transito.

\section{Popolazione esopianeti rilevati tramite velocit\'a radiale}

Considero il moto stellare attorno al centro di massa del sistema stella pianeta:
se l'asse z \'e lungo la linea di vista la posizione della stella \'e $z=r(t)\sin{i}\sin{(\omega+\nu)}$ e per la velocit\'a radiale:
\begin{equation}
v_r=\dot{z}=K[\cos{(\omega+\nu)}+e\cos{\omega}]\label{eq:vrsignal}
\end{equation}
\begin{equation}
K=\frac{2\pi}{P}\frac{a_*\sin{i}}{(1-e^2)\expy{1/2}}=\frac{\SI{28.4}{\meter\per\second}}{\sqrt{1-e^2}}(\frac{M_p\sin{i}}{M_J})(\frac{P}{\si{\year}})\expy{-1/3}(\frac{M_*}{\msun})\expy{-2/3}
\end{equation}
%K=(\frac{2\pi G}{P})\expy{1/3}\frac{M_p\sin{i}}{(M_p+M_*)\expy{2/3}}\frac{1}{(1-e^2)\expy{1/2}}
dove $K$ \'e la semi-ampiezza.
Un numero di misure della velocit\'a radiale permettono di determinare alcuni parametri orbitali ($e$, $P$, $t_p$, $\omega$) e la massa minima $M_p\sin{i}$.

\begin{workout}[Detection probability - Global efficiency]
Per probabilit\'a di rivelamento del $50\%$ la minima massa planetaria \'e
\begin{equation}
M_{p,min}\approx4\mearth{}(\frac{N}{20})\expy{-1/2}(\frac{\sigma}{\SI{1}{\meter\per\second}})(\frac{P}{\SI{1}{\day}})\expy{1/3}(\frac{M_*}{\msun{}})\expy{2/3}
\end{equation}
dove $\sigma$ tiene conto errori strumentali e rumore stellare.
Global efficiency of doppler measurement in given survey
This efficiency include precision of spectrograph, sample star characteristics and observing strategy (measurement frequency, exposure time long enough to diminish intrinsic stellar noise and photon noise).
cumming 2004/08
precisione strumentale:
rapporto segnale rumore $K/s$ per avere probabilit\'a di rivelazione $50/99\%$.
\end{workout}
Per sistemi multipli si considera la sovrapposizione lineare di $n_p$ termini \eqref{eq:vrsignal}: viene sottratto il segnale Kepleriano dominante fino a che resta solo rumore.

La probabilit\'a di rivelamento di un pianeta che orbita attorno a stella con date caratteristiche \'e determinata dall'ampiezza del moto radiale della stella attorno al centro di massa K, il rumore strumentale e stellare, la durata dell'osservazione e la forma dell'orbita del pianeta.
Riporto i risultati del survey HARPS+CORALIE (\cite{mayor2011harps}), capace di rivelare giganti gassosi entro $P<\SI{12}{\year}$ e super Terre/pianeti nettuniani con $P<\SI{100}{\day}$.

\begin{workout}[ricopiare risultati mayor11]
	?
\end{workout}

\begin{figure}[!ht]
\centering
\begin{subfigure}[b]{0.4\textwidth}
\includegraphics[trim={0cm 17cm 1cm 0},clip, width=0.99\textwidth]{PMfreq-e23}\label{fig:PMfreq-e23}
\end{subfigure}
~
\begin{subfigure}[b]{0.4\textwidth}
\includegraphics[trim={0cm 10cm 0 0},clip, width=0.99\textwidth]{PMfreq-e12}\label{fig:PMfreq-e12}
\end{subfigure}%

\begin{subfigure}[b]{0.4\textwidth}
\includegraphics[trim={0cm 8cm 0 0},clip, width=0.99\textwidth]{PMfreq-short}\label{fig:PMfreq-short}
\end{subfigure}
\caption{Diagramma massa-periodo: frazione di stelle che ospitano pianeta con caratteristiche nella regione evidenziata corretta per bias osservativi. Da sinistra a destra: $P<10\si{\year}$ e $M>50\mearth{}$, $P<1\si{\year}$ e $M>3-100\mearth{}$, $P<50\si{\day}$ e $M=1-3/3-10/10-30/30-100/100-300/300-1000$. Da \cite{mayor2011harps}.}\label{fig:PMfreqs}
\end{figure}

\begin{workout}[rileggere l'articolo mayor 11: harps neptunian etc]
risultati e pianeti abitabili
\end{workout}

I tre diagrammi (\ref{fig:PMfreqs}) mostrano la frequenza dei pianeti in delimitate regioni del diagramma periodo-massa ($P-M$) determinata utilizzando un sottocampione di stelle per cui la probabilit\'a di rivelazione relativa alla regione del diagramma $(P-M)$ \'e $99\%$: accanto alle linee continue nei diagrammi $(P,M)$ \'e indicata la percentuale di stelle per cui \'e possibile ottenere una probabilit\'a di rivelamento del $99\%$.

\begin{figure}[!ht]
\begin{subfigure}[b]{0.49\textwidth} \centering \includegraphics[trim={0cm 5cm 0 0},clip, width=0.94\textwidth]{freqvsM}
\caption{Distribuzione di massa di massa dei pianeti: in rosso distribuzione corretta per bias osservativi. Da \cite{mayor2011harps}.}\label{fig:freqvsM}
\end{subfigure}
~
\begin{subfigure}[b]{0.49\textwidth} \centering \includegraphics[trim={0cm 8cm 0 0},clip, width=0.94\textwidth]{freqvsMcomb}
\caption{Distribuzione di massa dei pianeti osservata. Da \cite{mayor2011harps}.}\label{fig:freqvsMcomb}
\end{subfigure}
\end{figure}

\begin{figure}[!ht]
\centering \includegraphics[trim={0cm 8cm 0 0},clip, width=0.43\textwidth]{PfreqvsFeH}
\caption{Numero di pianeti in funzione della metallicit\'a stellare: nero dei giganti gassosi, rosso dei pianeti con $M\leq30\mearth{}$ e blu combinata. Da \cite{mayor2011harps}.}\label{fig:PfreqvsFeH}
\end{figure}

La distribuzione di massa (figure \subref{fig:freqvsM}, \subref{fig:freqvsMcomb}) mostra un'abbondante popolazione di pianeti di massa piccola e una rapida diminuzione attorno a $10\mearth{}$; la distribuzione dei periodi mostra aumento dei pianeti giganti ($M>30\mearth{}$) a maggiore periodo orbitale (figura \subref{fig:freqvsPgiant}) e concentrazione pianeti piccola massa a $P=10-100\si{\day}$ (figura \subref{fig:freqvsPlowM}).

La figura \ref{fig:PfreqvsFeH} mostra l'assenza di correlazione tra la metallicit\'a della stella e la frequenza di pianeti di piccola massa fino a $30-40\mearth{}$ al contrario di quello che accade per pianeti giganti.

\begin{figure}[!ht]
\begin{subfigure}[b]{0.49\textwidth} \centering \includegraphics[trim={0cm 6cm 0 0},clip, width=0.98\textwidth]{freqvsPgiant}\caption{Frequenza pianeti con $M\geq 30\mearth{}$ in funzione del periodo orbitale; in rosso  tratteggiato con correzione bias osservativi. Da \cite{mayor2011harps}.}\label{fig:freqvsPgiant} \end{subfigure}
~
\begin{subfigure}[b]{0.49\textwidth} \centering \includegraphics[trim={0cm 10cm 0 0},clip,width=0.98\textwidth]{freqvsPlowM} \caption{Frequenza pianeti con $M\leq 30\mearth{}$ in funzione del periodo orbitale; in rosso con correzione bias osservativi. Da \cite{mayor2011harps}.}\label{fig:freqvsPlowM}
\end{subfigure}
\end{figure}

\begin{workout}[Spostamento picco superterre-aumento massa pianeti giganti all'aumentare del periodo]
Una comprensione pi\'u ampia si ha studiando la distribuzione dei pianeti nel diagramma massa-periodo (M-P): elenco alcune caratteristiche della distribuzione dei pianeti osservati (da \cite{mayor2011harps}).
\begin{itemize}
\item Il picco della densit\'a di super-terre si sposta da $6\mearth{}$ a $10\mearth{}$ per periodi da \SI{10}{\day} a \SI{100}{day}
%\item Non sono osservati pianeti fino a $50\mearth{}$ con $P>\SI{1000}{\day}$
\item il pianeta gassoso pi\'u massiccio aumenta da $1\mjupiter{}$ a $15\mjupiter{}$ passando da $P\approx\SI{1}{\day}$ a $P\approx\SI{15}{\year}$.
\end{itemize}
\end{workout}


\begin{workout}[Origin of metallicity excess: stellar structure]
Pinsonneault depoy caffee 01
\end{workout}•

\begin{workout}[minimo in mass distro]
\'e un minimo? ha un significato?
\end{workout}

\begin{workout}[dtectability and selection effects (pg 14)]
Perrymann pg 34 fig 2.25 d
fig 2.6,fig 2.18
\end{workout}

\begin{workout}[multplanet system: mean motion resonances, orbital spacing,...]
dynamical classification (barnes 08): tidal dominated if ($a<0.1$), resonant dom. if one resonant argument librates, secular otherwise.
Many systems are found dyn full and stability: packed planetary system hypothesis (Barnes Quinn 04, Barnes Raymond 04, Raymond Barnes 05, Raymond 06), orbital stability for closely packed (smith lissauer 09), outer solar system is very stable and dyna full (Levinson duncan 93, inner is less stable but dyna full (sussman wisdom 92, laskar 94)
n-body integration-orbit fitting: deprojected mass and relative inclination
perryman pg 38
Interazione n-body protopianeti vs evoluzione a lungo termine: i pianeti arrivano a risonanze attraverso migrazione
Kepler: latham 11 (first comparison of kepler planet), burke 14
wright09: ten multiplanet sysytem and systematic
Winn, Fabricky 15
fig 2.30 pg 39 Perrymann
(ford rasio 08: dynamical outcomes of PP scattering)

\begin{figure}[!ht]
\begin{subfigure}[b]{0.47\textwidth}
\centering
\includegraphics[trim={0cm 0 0 0},clip, width=0.9\textwidth]{exoresonance}
\caption{Da \cite{winnfabrycky15}.}\label{fig:exoresonance}
\end{subfigure}
~
\begin{subfigure}[b]{0.47\textwidth}
\centering
\includegraphics[trim={0cm 0 0 0},clip,width=0.9\textwidth]{singlemulti}
\caption{Da \cite{wright09}.}\label{fig:singlemulti}
\end{subfigure}
\end{figure}
\end{workout}

\begin{workout}[eccentricity distro: occurrence and architecture]
wright 09: multiplanet system tend to havelower eccentricity,
Limbach turner 14: $e_m\propto N\expy{-1.2}$.
planet around metal rich star have higher e (dawson, murray-clay 13)
smaller planets tend to have lower e (hogg 10 raccomanded modelling e distro on basis of Bayesian analysis)
\end{workout}

\clearpage

\section{Popolazione esopianeti rilevati tramite transito}

Le osservabili sono la differenza di luminosit\'a durante il transito, la durata totale e di massima occultazione del transito, e il periodo: da questi \'e possibile ricavare $R_p$, a, i, $R_*$, $M_*$ facendo uso della relazione massa raggio appropriata per la fase evolutiva della stella, per stelle di sequenza principale
\begin{equation}
\frac{R_*}{\rsun{}}=(\frac{M_*}{\msun{}})\expy{0.8}
\end{equation}

La probabilit\'a che l'orbita di un pianeta sia allineata con l'osservatore in maniera da avere un transito \'e:
\begin{equation}
p=(\frac{R_*\pm R_p}{a})(\frac{1}{1-e^2})=0.005(\frac{R_*}{\rsun{}})(\frac{a}{1AU})\expy{-1}(\frac{1}{1-e^2})
\end{equation}
dove $\pm$ indica inclusione/esclusione di transiti a raso.

\begin{workout}[Probabilit\'a rivelazione pianeti con orbite eccentriche e durata transito]
	La maggiore probabilit\'a di transito per orbite eccentriche \'e compensata approssimativamente dalla minore probabilit\'a di rilevamento per la minor durata del transito.
\end{workout}

Considerando un numero totale di osservazioni $N_o$ durante il transito si hanno $N_t\approx N_o\frac{R_*}{\pi a}$; il rapport segnale rumore \'e $S/N=\sqrt{N_t}\frac{\delta}{\sigma}$, $\sigma$ precisione fotometrica e $\delta=(\frac{R_p}{R_*})^2$. Per osservazione limitata da statistica fotoni $\sigma\propto\frac{1}{\sqrt{N_{ph}}}\propto d$: fissato S/N (e tipo stellare) il numero di pianeti rivelati varia come volume a cui \'e estesa la ricerca per probabilit\'a di transito:
\begin{equation}
V_p*P\propto d^3*\frac{R_*}{a}\propto R_p^6/P\expy{\frac{5}{3}}
\end{equation}

Il grafico (\ref{fig:H12yield}) mostra i risultati del survey di ricerca di transiti attorno a stelle di tipo solare realizzato con satellite kepler: si nota un picco di densit\'a che da $P\approx\SI{3}{\day}$, $R\approx2\rearth$ a $P\approx\SI{50}{\day}$, $R\approx4\rearth$


\begin{workout}[ricopiare risultati howard12]
?
tab 4 pg 13
\end{workout}
%\begin{align*}
%&\TDy{\log{R}}{f}=k_RR\expy{\alpha}\\
%&k_R=2.9_{-0.4}^{+0.5},\ \alpha=-1.92\pm0.11
%\end{align*}

\begin{figure}[!ht]
	\centering
	\includegraphics[trim={0cm 3.5cm 0 0},clip,width=0.95\textwidth,keepaspectratio]{H12yield}
	\caption{Frequenza pianeti osservati in funzione del raggio $R_p$ e del periodo $P$. In bianco per ogni cella \'e indicato: a sinistra in alto il numero di pianeti rivelati con SNR maggiore di una certa soglia scelta dagli autori del survey, in parentesi il valore corretto tenendo conto della probabilit\'a di transito, e a destra in basso il numero di stelle del survey per cui \'e possibile rivelazione del pianeta con caratteristiche medie della cella e con SNR superiore a soglia scelta, in basso a destra la frequenza di pianeti corretta per probabilit\'a di transito e bias incompletezza del survey. Da \cite{howard2012planet}.}\label{fig:H12yield}
\end{figure}

%Il grafico (\subref{fig:howard2012planet}) mostra distribuzione radiale corretta per probabilit\'a di transito, la distribuzione dei periodi \'e mostrata in figura (\subref{fig:freqperstarvsPanfit}): i pianeti sono raggruppati per intervalli di raggio.
La frequenza dei pianeti in funzione del raggio decresce all'aumentare del raggio con andamento
\begin{equation}
\begin{aligned}
&\TDy{\log{R_p}}{f}=k_RR\expy{\alpha}\\
&\alpha=\num{-1.92+-0.11},\ K_R=2.9_{-0.4}^{+0.5}
\end{aligned}
\end{equation}
La frequenza in funzione del periodo (vedi figura \ref{fig:howard2012planet}) cresce rapidamente a partire da periodo $P_0$ per poi avere un andamento meno ripido:
\begin{equation}
\begin{aligned}
&\TDy{\log{P}}{f}=K_PP\expy{\beta}[1-\exp{-(\frac{P}{P_0})\expy{\gamma}}]\\
&k_P=\num{0.035+-0.023},\ \beta=\num{0.52+-0.25},\ P_0(\si{\day})=\num{4.8+-1.6},\ \gamma=\num{2.4+-0.3}
\end{aligned}
\end{equation}
Inoltre le costanti dipendo dal raggio planetario come mostrato in figura (\subref{fig:freqperstarvsPanfit}): si ipotizza che il diverso valore del raggio critico $P_0$ sia dovuto alla dipendenza dal raggio planetario del meccanismo di frenamento della migrazione.

\begin{figure}[!ht]
	\begin{subfigure}[b]{0.47\textwidth}
		\centering
		\includegraphics[trim={0cm 22cm 0 0},clip, width=0.95\textwidth,keepaspectratio]{freqvsRpl50d}
		\caption{Distribuzione raggi planetari: crescita esponenziale verso piccoli raggi. La regione barrata indica misure incomplete. Da \cite{howard2012planet}.}\label{fig:howard2012planet}
	\end{subfigure}
	~
	\begin{subfigure}[b]{0.5\textwidth} \centering
		\includegraphics[trim={0cm 19cm 0 0},clip,width=0.95\textwidth,keepaspectratio]{freqperstarvsPanfit}
		\caption{Distribuzione dei periodi orbitali raggruppate per diversi raggi planetari: indicato il diverso periodo critico a cui si ha drastica diminuzione nella distribuzione planetaria. Da \cite{howard2012planet}.}\label{fig:freqperstarvsPanfit}
	\end{subfigure}
\end{figure}

\begin{workout}[Transit duration]
\[T=(\frac{R_*P}{\pi a}\sqrt{1-b^2})\frac{\sqrt{1-e^2}}{1+e\sin{\omega}}\]
\end{workout}

\begin{workout}[Transity/Doppler discrepancy]
Howard12 pg 20
\end{workout}

\begin{workout}[transit distro observed refs]
Refs:Mthods of detecting exoplanets (radius distro: pg 136)
the occurrence within 0.25 au of solar-type stars from kepler
occurrence architecture exoplanetary
fressin 13: 
petigure 13: plateau PLANET POPULATION BELOW TWICE EARTH SIZE
bathala 13
exoplanet.eu
\end{workout}

\begin{workout}[radius distro]
Local maximum at $1\mjupiter{}$, flat distro in log(r) between $4-10\mearth{}$ below strong increase, at $1.7\mearth{}$ local minimum separating super-earth from sub-neptune
\end{workout}

\begin{workout}[planet model: relazione massa-raggio]
characterization of exoplanets from formation: MR relation pg 18
formazione: vedi \ref{ch:gasaccretion}
MR diagram chabrier 09 (perryman 143)
\end{workout}

\begin{workout}[KEPLER MULTI-PLANET SYSTEM]
THE CALIFORNIA-KEPLER SURVEY. V. PEAS IN A POD: PLANETS IN A KEPLER MULTI-PLANET SYSTEM ARE SIMILAR IN SIZE AND REGULARLY SPACED
III - A gap in radius distribution of small planets
\end{workout}

\begin{workout}[transiti refs: osservabilie probabilit\'a di rilevamento]
pg103
pg 117, eccentric orbit pg 121-122
From space: presence of structure on stellar surface Perryman 3.4, Eriksson Lindegren 07; simulation of stellar jitter: Svensson Ludwig 05, Ludwig 06.
Probability of randomlyoriented planet on circular/eccentric orbit (Borucki Summers 84/Barnes 07, Burke 08,  Seagroves 03, Kane von Braun 08):
\end{workout}

\begin{workout}[Transiti: Propriet\'a dei pianeti a transito]
pg 143, fig 6.33, 6.34, 6.35
Mass-radius relation: chabrier 09
\end{workout}

\begin{workout}[M-R diagram]
orbital migration 10.8 Perryman, tidal effect 10.9, 
fig11.2, 11,4 (EOS), 11.7, 11.8 (M-R)
Ternary diagram 11.9, M-R realtion
Fig 6.33/34/35: mass radius realtion

pg 144: theoretical model - La posizione di un pianeta nel diagramma M-R fornisce indicazione della composizione
fig 6.35
\end{workout}

\begin{workout}[Struttura orbitale e risonanze moto medio]
\end{workout}

\begin{workout}[Long term dynamical stability: mmr, fully packing, eccentricity, inclination distro]
tidal circularization
resonance capture and migration (pg 41)
barnes 08: secularly/resonant dominated
Planets near mmr (Petrovich 12): strength of resonance
P-P scattering leds to tightly packed planetary systems (Raymond barnes 09)
Architecture of planetary systems based om kepler data: number of planets and coplanarity (Fang Margot 12): e,i distro.
Spacing Kepler planets (Wu Pu 15)
Are planetary systems filled to capacity? (Fang Nargot 13)
\end{workout}

\begin{workout}[bimodal radius distro is observed by transit?]
	Characterization of exoplanets from their formation II. The planetary mass-radius relationship pg 21: bimodal radius distribution
\end{workout}

\begin{workout}[Occurence rate]
	\begin{itemize}
	\item Coralie hot jupiter: giant planets $M\sin{i}>0.2\mjupiter$ and $a<0.1au$; $5.6\%$ giant out to 4au
	\item Lick/Keck/AAT ($M\sin{i}>0.5\mjupiter$): $1.2\%$ stars have hot jupiter and $6.6\%$ giant within 5 au
	\item ...
\end{itemize}
\end{workout}

\section{Propriet\'a generali esopianeti}

\begin{workout}[Mass-Metallicity relation for giant planet: ''Mass-metallicity relation for giant planets'' (Thorngren 16)]
Solve equation for planetary structure to infer $M_c+M_e$ assuming $M_c\approx10\mearth{}$; search correlations between $M_p$, $M_z$ and $[Fe/H]$, $Z_p$ and $Z_p/Z*$.
Negative correlation between planet's metal enrichment relative to its parent star: MF11, mordasini 14.
\begin{align*}
&M_p=M_c+M_e,\ M_Z=M_c+Z_*M_eZ_p=M_z/M_p\\
&Z_p/Z_*=1+(M_c/M_p)(1-Z*_)/Z_*
\end{align*}
latter expression fall off more rapidly with planet mass  than $Z_p/Z_*\approx10(M_p/M_J)\expy{-0.5}$ (later accretion of planetesimal debris Mousis 09)
\end{workout}

\begin{wrapfigure}[18]{l}{0.61\textwidth}
	\centering \includegraphics[trim={0cm 12cm 0 0},clip, width=0.6\textwidth]{MR-model-obs}
	\caption{Relazione massa-raggio determinata sulla base di modello planetario dopo \SI{5}{\giga\year}. Per gli esopianeti sono indicate le curve per temperature di equilibrio di \SI{1000}{\kelvin} e \SI{2000}{\kelvin}. Da \cite{guillot2014giant}.}\label{fig:MR-model-obs}
\end{wrapfigure}

Se si conoscono raggio e massa di un pianeta \'e possibile determinarne la struttura interna e la composizione tramite un modello planetario: la posizione di un pianeta nel diagramma massa-raggio \'e determinata dalla storia di accrescimento del pianeta (pianeta roccioso, pianeta costituito in parte da ghiacci, pianeta con massa consistente di gas), dal bilancio termico.
%e propriet\'a di H/He e metalli a alte densit\'a (rottura H molecolare, ionizzazione da pressione, degenerazione elettronica, transizioni di fase).
La figura \ref{fig:MR-model-obs} mostra la relazione massa-raggio per pianeti nettuniani fino a regimi stellari. Considerando l'equazione di stato
\begin{equation}
P\propto\rho\expy{\gamma},\ \gamma=\frac{1}{n}+1
\end{equation}
la relazione massa raggio \'e
\begin{equation}
R\propto M\expy{(1-n)/(3-n)}
\end{equation}
per $M\approx\mjupiter$ $n\approx1$ quindi il raggio \'e approssimativamente costante. Questo fatto \'e visibile anche nella distribuzione dei raggi planetari (figura \subref{fig:probvsR-T}) nel picco attorno a $\rjupiter$.
La figura (\subref{fig:RfreqPl100ub}) mostra due picchi nella distribuzione radiale a $1.3\rearth{}$ e $2.4\rearth{}$ dovuti alle sottopopolazioni di pianeti rocciosi e gassosi per $P<\SI{100}{\day}$: i pianeti rocciosi hanno atmosfera di H/He che non contribuisce alle dimensione mentre i pianeti nettuniani hanno strato di gas consistente rispetto alla massa del pianeta. La valle a $1.7\rearth{}$ potrebbe essere il raggio di separazione tra pianeti terrestri con atmosfera tenue di H/He soggetta a evaporazione e pianeti nettuniani con massa consistente di H/He.


\begin{errata}[Equazione di stato politropa: relazione massa raggio]
	\begin{equation}
	P\propto\rho\expy{\gamma},\ \gamma=\frac{1}{n}+1
	\end{equation}
	si che per $M\approx\mjupiter$ $n\approx1$.
	Inoltre la relazione massa-raggio  \'e approssimabile
	\begin{equation}
	R\propto M\expy{(1-n)/(3-n)}
	\end{equation}
\end{errata}

%\begin{figure}[!ht] \centering\includegraphics[trim={0cm 14cm 0 0},clip, keepaspectratio,width=0.9\textwidth]{MRD}\caption{Diagramma mass-raggio per pianeti di transito. Le curve tratteggiate sono a densit\'a costante. Il simbolo $\circ$ indica i pianeti del sistema solare. Da \cite{perryman2011exoplanet}.}\label{fig:MRD}
%\end{figure}

%\includegraphics[trim={0cm 0 0 0},clip, keepaspectratio,width=0.9\textwidth]{freqvsRpl50d}\label{freqvsRpl50d}\caption{Da \cite{howard2012planet}}
% \includegraphics[trim={0cm 0 0 0},clip,width=0.95\textwidth]{keplerRfreq}\label{fig:keplerRfreq} \caption{Da \cite{fressin2013false}}
%\includegraphics[trim={0cm 0 0 0},clip, keepaspectratio,width=0.9\textwidth]{freqvsRpl50d}\label{fig:freqvsRpl50d}\caption{Da \cite{}.}

\begin{figure}[!ht]
\begin{subfigure}[b]{0.49\textwidth}
	\centering \includegraphics[trim={0 12cm 0 0},clip, width=0.95\textwidth]{RfreqPl100ub} 
	\caption{Distribuzione per i pianeti osservati tramite transito con $P<\SI{100}{\day}$: la valle a $1.7\mearth{}$ si ipotizza causato dall'evaporazione. Da \cite{fulton2017california}.}\label{fig:RfreqPl100ub}
\end{subfigure}
~
\begin{subfigure}[b]{0.39\textwidth}
	\centering\includegraphics[trim={0cm 0.5cm 0 0},clip, keepaspectratio,width=0.99\textwidth]{Rfreq}\caption{Distribuzione radiale pianeti rivelati tramite T. Da \cite{exoplanet.eu} (luglio 2018): la frequenza ha andamento crescente verso pianeti di raggio terrestre, andamento piatto tra $4-10\rearth{}$ e picco a $R\approx\rjupiter{}$.}\label{fig:probvsR-T}
\end{subfigure}
\end{figure}

\begin{workout}[sottosezione propriet\'a orbitali]

\end{workout}


{\let\clearpage\relax\let\cleardoublepage\relax
	\chapter{Dischi protoplanetari: struttura e condizioni iniziali per simulazione formazione pianeti}
}

%\section{Modello disco di accrescimento }
Per riprodurre le propriet\'a statistiche della popolazione planetaria osservata \'e fondamentale determinare la distribuzione delle caratteristiche dei dischi di accrescimento che, combinate usando metodi di Montecarlo, vanno a costituire le condizioni iniziali dei modelli di formazione.

Nei modelli di disco di accrescimento $1+1D$ si parametrizza il trasporto di momento angolare ponendo:
\begin{equation}
\nu=\alpha c_s H
\end{equation}
dove $\alpha$ \'e parametro da fissare, $c_s$ la velocit\'a del suono e H lo spessore del disco. L'evoluzione della densit\'a superficiale del disco \'e determinata dall'equazione:
\begin{equation}
\PDy{t}{\Sigma}=3\frac{1}{r}\PDof{r}[r\expy{1/2}\PDof{r}(\nu\Sigma r\expy{1/2})]\label{eq:sigmaevol}
\end{equation}



La struttura verticale \'e determinata dalla componente lungo z dell'attrazione del corpo centrale
%profilo termico determinato da equilibrio termico
\begin{equation}
\TDy{z}{P}=-\rho g_z=-\frac{GM_*}{r^2+z^2}\frac{z}{r}\rho=-\Omega^2\rho z
\end{equation}
mentre la struttura radiale del disco di gas \'e determinata dall'equilibrio idrostatico
\begin{equation}
\frac{v_g^2}{r}=\frac{GM_*}{r^2}+\frac{1}{\rho}\TDy{r}{P}
\end{equation}
dove ho introdotto $v_g$ velocit\'a a cui il gas orbita attorno alla stella centrale.


\begin{figure}[!ht]
	\begin{subfigure}[b]{0.39\textwidth}
		\centering
		\includegraphics[trim={0cm 10cm 0 0},clip, keepaspectratio,width=0.98\textwidth]{SED-contours}
		\caption{Struttura dischi protoplanetari (flusso a $\lambda=\SI{880}{\micro\meter}$) e densit\'a spettrale. La linea tratteggiata viola indica la distribuzione di corpo nero della fotosfera stellare. Da \cite{andrews2010protoplanetary}.}\label{fig:SED-contours}
	\end{subfigure}
	~
	\begin{subfigure}[b]{0.55\textwidth}
		\includegraphics[trim={0cm 15cm 0 0},clip, keepaspectratio,width=0.98\textwidth]{diskdensity}
		\caption{Densit\'a superficiale dischi protoplanetari osservati nella regione di Ophiuchus ricavata risolvendo modello disco di accrescimento che riproduca densit\'a spettrale di energia osservata. I rettangoli grigi indicano densit\'a superficiale per la MMSN nella regione di Saturno, Urano e Nettuno. Da \cite{andrews2010protoplanetary}.}\label{fig:diskdensity}
	\end{subfigure}
\end{figure}

\begin{errata}[Strutture attorno a stelle giovani]
Attorno alle stelle giovani sovente si trovano strutture opache nel visibile ma otticamente sottili nelle onde infrarosse: risolvendo il modello di disco protoplanetario, avente come input la distribuzione della dimensione della polvere e rapporto gas/polvere, per riprodurre le caratteristiche spettrali e flusso termico osservati, note anche le caratteristiche della stella centrale, si determinano i parametri caratteristici del disco.
\end{errata}

La figura (\subref{fig:SED-contours}) mostra le strutture attorno a stelle giovani del cluster Ophiuchus secondo profili di intensit\'a a \SI{880}{\micro\meter}. Se $\nu\propto r\expy{\gamma}$ una soluzione di \eqref{eq:sigmaevol} per $t=0$ \'e
\begin{equation}
\Sigma=(2-\gamma)\frac{M_d}{2\pi R_c^2}(\frac{r}{R_c})\expy{-\gamma}\Exp{-(\frac{r}{R_c})\expy{2-\gamma}}
\end{equation}
\cite{andrews2010protoplanetary} hanno osservato lo spettro emesso da oggetti nella regione di formazione stellare Ophiuchus determinando massa, dimensioni caratteristiche e viscosit\'a: $M_d=0.004-0.143\msun{}$, $\gamma=0.4-1.1$, $R_c=\SIrange{14}{198}{\astronomicalunit}$.

\begin{wrapfigure}[17]{l}{0.5\textwidth}
	\includegraphics[trim={0cm 19cm 0 0},clip, keepaspectratio,width=0.48\textwidth]{selfsimilaralphadisk}
	\caption{Soluzione equazione \eqref{eq:sigmaevol} per $\nu\propto r\expy{\gamma}$ con $\gamma=1$. Da \cite{armitage2007lecture}.}\label{fig:selfsimilaralphadisk}
\end{wrapfigure}

Tramite metodi di montecarlo si determinano numerose combinazioni delle condizioni iniziali del disco e dei parametri scelti per descrivere la sua struttura.
Le variabili di montecarlo hanno distribuzione determinata dalle osservazioni.

\begin{workout}[modello globale: evoluzione dal semplice al complesso]
	Global model of planets formation and evolution: more detailed submodels
\end{workout}

Il modello pi\'u semplice di disco di accrescimento usa distribuzione esponenziale per andamento densit\'a superficiale e temperature, assumendo disco otticamente sottile, e relazione $L_*\propto M_*^4$ di sequenza principale. Le lacune di questo modello sono
\begin{itemize}
	\item i dischi sono otticamente spessi con transizioni nell'opacit\'a
	\item non \'e presente evoluzione temporale consistente
\end{itemize}

Modelli pi\'u recenti  risolvono l'equazione per l'evoluzione viscosa:
\begin{align}
	&\TDy{t}{\Sigma}=\frac{1}{r}\PDof{r}[3r\expy{1/2}\PDof{r}(\nu\Sigma r\expy{1/2})]+\dot{\Sigma}_w(r)+\dot{\Sigma}_{embryo}(r)\label{eq:diskaccrphev-m18}\\
	&\dot{\Sigma}_w(a)=\left\{\begin{array}{c}0\quad a<R_g\\\frac{\dot{M}_w}{2\pi(a_{max}-R_g)a}\quad \text{altrimenti}\\\end{array}\right.
\end{align}
con $\dot{\Sigma}_w$ contributo della foto-evaporazione (\cite{veras2004outward}), $R_g=\SI{5}{\astronomicalunit}$ e $a_{max}$ estremo del disco,  $\dot{\Sigma}_{embryo}$ contributo di accrescimento di gas e componente solida dei pianeti.
La densit\'a superficiale iniziale \'e assunta essere:
\begin{equation}
\Sigma(a,t=0)=\Sigma_0(\frac{r}{1AU})\expy{p_g}\Exp{[-(\frac{r}{R_o})\expy{2+p_g}]}(1-\sqrt{\frac{r}{R_i}})
\end{equation}
dove i parametri possono essere scelti casualmente secondo una distribuzione di probabilit\'a ricavata dalle osservazioni.

La distribuzione di massa e la frazione di dischi protoplanetari per ammassi stellari di et\'a diversa \'e mostrata in figura (\ref{fig:initdistro}). La massa \'e determinata misurando il flusso di emissione termica della polvere: la distribuzione di $\log{M_{disk}}$  \'e gaussiana e per il cluster Ophiuchus la distribuzione \'e fittata da gaussiana corrispondente a massa media $M_{disk}=0.042\msun{}$.

Il tempo caratteristico del disco \'e determinato dalla viscosit\'a $\alpha$: fissata $\alpha$ compatibile con tempo caratteristico osservato $\tau_{disk}^{obs}\approx\SI{3}{\mega\year}$ e assumendo la distribuzione uniforme nel logaritmo di $\dot{M}_w$, si calcola  $t_{disk}(\alpha,\Sigma_0,\dot{M}_w)$ determinando gli estremi dello fotoevaporazione per riprodurre tempi di vita osservati.

Valori tipici sono $\dot{M}_w=\SIrange{5e-10}{3e-8}\msun{}/\si{\year}$ per $\alpha=\num{7e-3}$.

Nelle popolazione planetaria considerata $\alpha$ \'e fissato compatibilmente con le osservazioni  ed \'e omogeneo e costante.

\begin{figure}[!ht]
	\includegraphics[trim={0cm 10cm 0 0},clip, keepaspectratio,width=0.7\textwidth]{initdistro}
	\caption{Distribuzione caretteristiche dischi di accrescimento. Da \cite{mordasini2018planetary}.}\label{fig:initdistro}\end{figure}

\begin{workout}[viscosit\'a disco]
	Nella popolazione planetaria simulata $\alpha$ \'e fissato sulla base delle osservazioni  ed \'e omogeneo e costante.
\end{workout}

\begin{workout}[Modello di disco di accrescimento - (Mordasini18: 4) - Introduzione descrizione fenomeni grazie a PPS]
	Refs: garaud lin 07, chiang goldreich 97
	Un modello di disco  di accrescimento usato nelle simulazioni considera l'evoluzione della densit\'a superficiale tramite l'equazione \eqref{eq:diskaccrphev-m18}
	\begin{align}
		&\TDy{t}{\Sigma}=\frac{1}{r}\PDof{r}[3r\expy{1/2}\PDof{r}(\nu\Sigma r\expy{1/2})]+\dot{\Sigma}_w(r)+\dot{\Sigma}_p(r)\label{eq:diskaccrphev-m18}\\
		&\dot{\Sigma}_w(a)=\left\{\begin{array}{c}0\\\frac{\dot{M}_w}{2\pi(a_{max}-R_g)a}\\\end{array}\right.
	\end{align}
	Densit\'a superficiale iniziale:
	\begin{equation}
	\Sigma(a,t=0)=\Sigma_0(\frac{r}{1AU})\expy{p_g}\Exp{[-(\frac{r}{R_o})\expy{2+p_g}]}(1-\sqrt{\frac{r}{R_i}})
	\end{equation}
	4-Mordasini18 (Hayashi81). $p_g\approx1$ (Andrews10).
\end{workout}

\begin{workout}[Some refs]
	\begin{itemize}
		\item Lamb 32: amount of energy converted in turbulent motion
	\end{itemize}
\end{workout}

%Jansky=10-23erg/(s*cm2*Hz)
\begin{workout}[Refs dischi protoplanetari]
	Chronology of  early stages: apai lauretta 10 (book)
\end{workout}

\begin{workout}[Condizioni iniziali: formazione planetesimi, caratteristiche dischi protoplanetari]
	Chronology of  early stages: apai lauretta 10 (book)
\end{workout}

\begin{wrapfigure}[6]{l}{0.5\textwidth}
	\centering
	\includegraphics[trim={0cm 13cm 0 0},clip, width=0.49\textwidth,keepaspectratio]{clusterage-fstar}
	\caption{Frazione di stelle con disco protoplanetario in funzione dell'et\'a del cluster. Da \cite{mamajek2009initial}. }\label{fig:clusterage-fstar}
\end{wrapfigure}

In figura (\ref{fig:clusterage-fstar}) \'e riportata la frazione di stelle con dischi protoplanetari in funzione dell'et\'a del cluster. \'E possibile stimare tempo di vita caratteristico del disco considerando andamento esponenziale per frazione di dischi: secondo (\cite{mamajek2009initial}) $\tau_c\approx\SI{2.5}{\mega\year}$, mentre (\cite{haisch2001disk}) hanno determinato $\tau_c\approx\SI{6}{\mega\year}$.

\vspace{0.2\textheight}

\begin{workout}[Profilo densit\'a verticale disco isotermo]
	Thin vertically isothermal disk
	\begin{align}
		&\rho(z)=\frac{\Sigma}{h\sqrt{2\pi}}\exp{-\frac{z^2}{2h^2}}\\
		&
	\end{align}
\end{workout}

\begin{workout}[Profilo termico del disco]
	
\end{workout}

\begin{workout}[Momento torcente in disco Kepleriano]
	\begin{align}
		&\sigma_{ij}=2\Sigma\nu D_{ij}=\Sigma\nu\begin{pmatrix}0&-\Omega\partial_r(r\Omega)\\-\Omega+\partial_r(r\Omega)&0
		\end{pmatrix}\\
	\end{align}
	Il momento torcente esercitato da S su $C^c$ \'e
	\begin{equation}
	T=\iint_{S^c}(\vec{r}\wedge\nabla\sigma)\,ds=\int_S\vec{r}\wedge(\sigma\,dl\vec{n})
	\end{equation}
	Nel caso S sia circonferenza di raggio r:
	\begin{equation}
	T_{\nu}=3\pi r^2\Omega_0\Sigma\nu
	\end{equation}
\end{workout}

\begin{workout}[Struttura verticale del disco, profilo termico]
	Pressione gas determinata da equilibrio idrostatico
	Passive disk (Refs: Ref: spectral energy distribution of T-Tauri star with passive circumstellar disk)	Dust passively reirradiate star light
\end{workout}

\begin{workout}[Viscous/turbulent disk]
	laminar (high momentum diffusion)
	mass inflow Gullbring98 \SIrange{e-9}{e-7}{\per\year}$\msun{}$
\end{workout}
\begin{workout}[MRI]
	fig 10.3:
	gammie 1996
\end{workout}

\begin{workout}[alpha prescription]
	$\vec{v}=(u_r,r\Omega)$, stress tensor $\sigma_{r\phi}=-\Sigma\exv{u_ru_{\phi}}$.
	Enhanced turbulent viscosity: $-\Sigma\exv{u_ru_{\phi}}=\Sigma\nu r\TDy{r}{\Omega}$, $\nu=v_TH$, $\alpha=v_T/c_s$
\end{workout}

\begin{workout}[Accretion disk sources]
	Lodato: classical disk physics
	o218, 296: disk formation, constrains from solar system
	the alpha disk pg 18: theory of turbulent accretion disk
\end{workout}


\begin{workout}[Descrizione trasporto momento angolare tramite parametrizzazione viscosit\'a]
	Mass conservation + momentum conservation in viscous flow: angular momentum evolution. Phenomena: Shear viscosity - turbulence - MRI.
	Nei modelli 1D per disco di accrescimento si parametrizza viscosit\'a tramite $\nu=\frac{\eta}{\rho}\to\alpha c_s H$: la viscosit\'a molecolare \'e troppo bassa  per trasportare all'esterno il momento angolare sui tempi-scala osservati.
	Theory of turbulent accretion disk: pg 18, 8.
	Flusso di x-momentum lungo y $\rho\exv{u_xu_y}$, dove considero la velocit\'a dell'elemento di fluido $(v_x+u_x,u_y)$ dove le $u$ rappresentano fluttuazioni della velocit\'a: $\sigma_{xy}=-\rho\exv{u_xu_y}$.
	Ricordando la definizione di tensore degli stress e con $\vec{v}=v_x(y)\hat{x}$
	\begin{equation}
	\sigma_{ij}=\eta(\partial_jv_i+\partial_iv_j-\frac{2}{3}\delta_{ij}\partial_kv_k+\zeta\delta_{ij}\partial_kv_k \to \eta\TDy{y}{v_x}
	\end{equation}
	\begin{equation}
	\PDof{t}(\rho v_i)+\PDof{x_i}(\rho v_iv_j+T_{ij})=F_i
	\end{equation}
	Fluttuazioni: $\sigma_{ij}=-\rho\exv{u_iu_j}$.
	Viscosit\'a molecolare: $\exv{u_iu_j}=-\nu\TDy{x_j}{v_i}$.
	Turbulence: Reynpold stress $\tau_{ij}=-\rho\exv{v_jv_i}$, $\exv{u_iu_j}=-\nu_T\TDy{x_j}{v_i}$
	\begin{equation}
	\PDy{t}{\Sigma}=3\frac{1}{r}\PDof{r}[r\expy{1/2}\PDof{r}(\nu\Sigma r\expy{1/2})]
	\end{equation}
\end{workout}

\begin{workout}[Photoievaporation: X-EUV-FUV (Alexander13: The Dispersal of Protoplanetary Disks)]
	Protoplanetary disc evolution and dispersal: the implications of X-ray photoevaporation.
	(Photoevaporation: veras armitage 2003, Alexander13, Mordasini12. Internal/External).
	EUV($E\approx13.6eV$): ionization, FUV($E\approx6-13.6eV$): dissociation, X-ray
\end{workout}

\begin{workout}[classificazione YSO]
	section{YSO: distribuzione propriet\'a dischi protoplanetari}
	La classificazione empirica degli young stellar object (YSO) si basa sul valore di
	\begin{equation}
	\alpha_{IR}=\TDly{\lambda}{\lambda F_{\lambda}}
	\end{equation}
	tra \SIrange{2.2}{25}{\micro\meter}.
	\begin{itemize}
		\item Classe 0: Picco emissione in regione da lontano infrarosso a sub-millimetrica. Nubi molecolari compatte/protostelle.
		\item Classe 1: $\alpha_{IR}>0$ a \SI{2}{\micro\meter}. La stella accresce la maggior parte della sua massa finale.
		\item Classe 2: $0\geq\alpha_{IR}\geq-1.5$. Stella T-Tauri con disco otticamente spesso a $\lambda\leq\SI{10}{\micro\meter}$.
		\item Classe 3: $\alpha_{IR}\leq-1.5$. Stella T-Tauri con disco otticamente sottile a $\lambda\leq\SI{10}{\micro\meter}$.
	\end{itemize}
\end{workout}

\begin{workout}[Fit SED]
	Disk model
	\begin{align}
		\nu\propto R\expy{\gamma}\\
		\Sigma=(2-\gamma)\frac{M_d}{2\pi R_c^2}(\frac{R}{R_c})\expy{-\gamma}\\Exp{-(\frac{R}{R_c})\expy{2-\gamma}}
	\end{align}
\end{workout}


\begin{workout}[Initial conditionfor protostellar collapse]
	Andre 2000 - 
\end{workout}

\begin{workout}[physical processes in protoplanetary disk: IRexcess, accretion rate and lifetime,]
	(armitage 17): fig 1, accretion rate and lifetime (2.2 pg 5-6), inference from dust continuum (disk mass, evidence for mm-particle growth
	Le stelle giovani sono caratterizzate da eccesso infrarosso fra \SIrange{2.5}{10}{\micro\meter}:
	\begin{equation}
	\alpha_{IR}=\TDly{\nu}{\nu F_{\nu}}
	\end{equation}
	Sorgenti con SED declinante in mid infrared ($2.5-10\si{\micro\meter}$): $1.5<\alpha_{IR}=<0$. Active/passive: mass infall convert G into thermalradiation/reprocessed starlight.
	Refs: Perrymann 10.3 - disk formation pg 218 - 
	Classification based on slope of SED between $2-25\si{\micro\meter}$: $\alpha_{IR}=\TDly{\nu}{\nu F_{\nu}}=$ (Gail Hoppo 2010)
	Spitzer: forming regions within 500pc
	Formation: disk quickly forms as more distant material with high angular momentum, centrifugal radius $R(t)\propto\Omega^2 t^3$: . Class 0-I : protostellar disk, gravitational unstable (cloud collapse: the collapse of the cores of slowly rotating isothermal clouds, Terebey shu cassen 1984, selfsimilar collapse of isothermal spheres and star formation, shu 77). Singular isothermal sphere: $\rho=\frac{a^2}{2\pi G}r\expy{-2}$, $M(r)=\frac{2a^2}{G}r$.
\end{workout}

\begin{workout}[disco protoplanetario: distribuzioni condizioni iniziali]
	
	Initial disk mass
	(infall phase end, no more self-gravitational instabilities) - stability(shu90), MMSN hayashi81/weidenshilling77, observations(Andrews10,Manara16) points to $(0.1-10)\%$ stellar mass, distro log-normal with mean $0.01M_*$.
	
	{Disk lifetime}
	$1-10My$ with mean $3My$ (Haisch01, Mamajek09)
	
\end{workout}

\begin{workout}[YSO: classificazione ]
	SED: emissine a lunghezze d'onda millimetriche probano tutto il volume del disco, otticamente sottile a quelle lunghezze (Beckwith 90, Beckwith Sargent91).
	$S_{\nu}\propto B_{\nu}(1-\exp{-\tau})\approx B_{\nu}\tau$: emission produced near cold disk mid-plane.
	Rayleigh-jeans: $B_{\nu}\propto T$ and, since $\tau=\kappa\Sigma$, $S_{\nu}\propto \kappa\Sigma T$.
	(SED: Andrews williams 07)
	Survey: 0.3'', 345Ghz(870$\micro m$) 1Myr old Ophiuchus stars forming region
	The absolute chronology and thermal processing of solids in the solar protoplanetary disk (Connelly Ivanova 12)
\end{workout}

\begin{workout}[Selfsimilar initial disk density solution]
	Refs: Lynden-bel pringle 74
	Se la viscosit\'a \'e statica e distribuita secondo $\nu\propto r\expy{\gamma}$:
	\begin{align}
		&R_c=R_1\mathcal{T}\expy{1/(2-\gamma)}\\
		&M_d=M_{d,0}\mathcal{T}\expy{-1/2(2-\gamma)}\\
	\end{align}
\end{workout}

\begin{workout}[Analitic disk model]
	Chambers 09: On analytical model for the evolution of a viscous, irradiated disk
	On location of snowline in protoplanetary disk (lecar chiang 06)
	On the snow line in dusty PPD (Sasselov lecar 99)
	Accretion disks around young object I: Detailed vertical structure (D'alessio Lizzano 98)
\end{workout}

\begin{workout}[YSO properties distro]
	Refs: ''Protoplanetary Disk Structures in Ophiuchus''
	da dove sono ricavete nei PPS?
	mass: MMSN, andrews 10, manara 16
	lifetime: IR/UV excess: haisch 01, mamajek 09
	initial embryo starting position: relative spacing of few hill radii (kokubo ida 00), fill disk thinking of asyntituc isolation mass (Ida Lin 10), trapped (hasegawa pudritz 11, cridland 16)
	Hueso 05: evolution of protoplanetary disk, meyer 06 Formation and evolution of planetary systems, Udry 07: statistical properties of exoplanets)
	Distribuzione di probabilit\'a per condizioni iniziali.
	
	Initial condition for planet formation (protoplanetry disk \cite{meyer2006formation}): disk metallicity, mass, lifetime, \ldots
	
	{Metallicity and Dust/Gas}
	$[M/H]$ distro modelled as normal: $\mu=-0.02$, $\sigma=0.22$ (photosphere of solar-like stars in solar neighborhood): Santos05.
	$f_{dg}=f_{dg,\odot}10\expy{[M/H]}$ with $f_{dg,\odot}=0.01-0.02$.
\end{workout}

\begin{workout}[Classificazione YSO: convenzioni]
	Sorgenti con SED declinante in mid infrared ($2.5-10\si{\micro\meter}$): $1.5<\alpha_{IR}=<0$. Active/passive: mass infall convert G into thermalradiation/reprocessed starlight.
	Refs: Perrymann 10.3 - disk formation pg 218 - 
	Classification based on slope of SED between $2-25\si{\micro\meter}$: $\alpha_{IR}=\TDly{\nu}{\nu F_{\nu}}=$ (Gail Hoppo 2010)
	Spitzer: forming regions within 500pc
	Formation: disk quickly forms as more distant material with high angular momentum, centrifugal radius $R(t)\propto\Omega^2 t^3$: . Class 0-I : protostellar disk, gravitational unstable (cloud collapse: the collapse of the cores of slowly rotating isothermal clouds, Terebey shu cassen 1984, selfsimilar collapse of isothermal spheres and star formation, shu 77). Singular isothermal sphere: $\rho=\frac{a^2}{2\pi G}r\expy{-2}$, $M(r)=\frac{2a^2}{G}r$
\end{workout}

\begin{workout}[Refs dischi protoplanetari]
	Ciesla Dullemond10 
	Perryman ch 10
	williams cieza 11
	Andrews 09-10
	mamajek 2009 - Initial conditions of planet formation: lifetimes of primordial disks
	infrared excess: gail hoppe10
\end{workout}

%\section{''Condizioni iniziali'': disco di accrescimento}

\begin{workout}[Composizione disco accrescimento]
	(composizione polvere tab 11.2 perry pg 585 e nomenclatura refrattari-volatili;)
\end{workout}

\begin{workout}[Disco di accrescimento: calibrazione massa, tempo di vita, evaporazione con osservazioni]
	
\end{workout}

%Nei modelli unidimensionali per disco di accrescimento si descrive il trasporto di momento angolare verso l'esterno introducendo la viscosit\'a aumentata $\nu=\frac{\eta}{\rho}\to\alpha c_s H$.

\begin{workout}[Strees tensor]
	appendix a pg 66 planet formation and migration vs crida PHD torque calculation pg 24
	stress tensor in keplerian disk: crida PHD
	pg 26
	\begin{align}
		&\sigma_{ij}=d_{ij}-PI{ij}\\
		&d_{ij}=2\Sigma\nu(D_{ij}-\frac{1}{3}(\div{\vec{v}}I_{ij})+\Sigma\zeta(\div{\vec{v}})I_{ij}\\
		&D_{ij}=\frac{1}{2}(\partial_iv_k+\partial_kv_i)
	\end{align}
\end{workout}

\begin{workout}[Dust settling toward midplane]
	chiang 01; Dullemond Dominik 04; D'Alessio 06
\end{workout}

\begin{workout}[''The TW Hya Disk at \SI{870}{\micro\meter}:  Comparison of CO and Dust Radial Structures'': introduzione profilo pi\'u ripido per componente solida]
peeble drift: peeble drift inward from peeble formation line 
\end{workout}

\begin{workout}[Caratteristiche fittate proto-dischi]
	\begin{table}[!ht]
		\begin{tabular}{|cccccc|}
			Nome&$M_d$&$\gamma$&$R_c$&$H_{100}$&\\
			AS 205&0.029&0.9&46&19.6&0.11\\
		\end{tabular}
	\end{table}
\end{workout}

%\section{Modello disco di accrescimento e distribuzione condizioni iniziali}
%Refs: Sec4 mordasini09
%\cleardoublepage

\begin{workout}[planet luminosity]
	\begin{align}
		&L=L_{cont}+L_{acc}\\
		&L_{cont}=-\frac{E_t(t+dt)-E_t(t)-E_{gas,acc}}{dt}\\
		&E_{gas,acc}=dt\,\dot{M}_{gas}u_{int}\\
		&L_{acc}=G\frac{\dot{M}_{core}M_{core}}{R_{core}}
	\end{align}
\end{workout}
