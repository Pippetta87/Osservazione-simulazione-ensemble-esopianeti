\section{Altre tecniche osservative}
\begin{workout}[Astrometry]
L'ellisse percorsa dalla stella attorno al baricentro stella pianeta ha ampiezza angolare
\begin{equation}
\alpha=\frac{M_p}{M_p+M_*}a\approx\frac{M_p}{M_*}a=(\frac{M_p}{M_*})(\frac{a}{1AU})(\frac{d}{1pc})\expy{-1}\si{\arcsec}
\end{equation}

Variation in photocentre, radial velocity and total flux
Astrometry: Photonoise $\sigma_{ph}=\frac{\lambda}{4\pi D}\frac{1}{S/N}$
pg 65

\end{workout}

\begin{workout}[Timing: ritardo segnale]
Pg 75
$\tau_p=\frac{1}{c}\frac{a\sin{i}M_p}{M_*}$
\end{workout}

\begin{workout}[Microlensing]
$\frac{M_L}{\msun{}}=0.123\frac{\theta_E^2}{\upvarpi_{rel}}$
Pg 83. fig 5.7. Applications pg 93
(Gould 10: ''fREQUENCY OF SOLAR LIKE SYSTEM AND ICE DAS GIANT BEYOND SNOW LINE FROM MICROLENSING...'')
\end{workout}

\begin{workout}[Imaging: star-planet brightness ration]
pg 149
\begin{equation*}
\frac{f_p()}{f_*()}=
\end{equation*}
\end{workout}
\begin{workout}[Imaging: Observational limits]
ground based (pg 158) space based imaging (pg162)
fig 7.10
\end{workout}