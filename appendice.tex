\chapter{Tecniche osservative}

\section{Velocit\'a radiale}


\begin{workout}[Radial velocity: keplerian observables (Appendice RV)]
La coordinata z della stella $z=r(t)\sin{i}\sin{(\omega+\nu)}$ e per la velocit\'a radiale
\begin{align}
&v_r=\dot{z}=K[\cos{(\omega+\nu)}+e\cos{\omega}]\\
&K=\frac{2\pi}{P}\frac{a_*\sin{i}}{(1-e^2)\expy{1/2}}=(\frac{2\pi G}{P})\expy{1/3}\frac{M_p\sin{i}}{(M_p+M_*)\expy{2/3}}\frac{1}{(1-e^2)\expy{1/1}}
\end{align}
Le osservabili $e$, $P$, $t_p$, $\omega$, $K=f(a,e,P,i)$ sono fittate per ogni pianeta.
La velocit\'a radiale della stella \'e
\begin{equation}
v_r(t)=K[\cos{(\omega+\nu(t)}+e\cos{\omega}]+\gamma+d(t-t_0)
\end{equation}
Per sistemi multipli consideriamo la sovrapposizione lineare di $n_p$ termini \eqref{eq:vrsignal}: viene sottratto il segnale Kepleriano dominante fino a che resta solo rumore.
\end{workout}

\begin{workout}[Radial velocity: keplerian observables (Appendix RV)]
La coordinata z della stella $z=r(t)\sin{i}\sin{(\omega+\nu)}$ e per la velocit\'a radiale
\begin{align}
&v_r=\dot{z}=K[\cos{(\omega+\nu)}+e\cos{\omega}]\\
&K=\frac{2\pi}{P}\frac{a_*\sin{i}}{(1-e^2)\expy{1/2}}=(\frac{2\pi G}{P})\expy{1/3}\frac{M_p\sin{i}}{(M_p+M_*)\expy{2/3}}\frac{1}{(1-e^2)\expy{1/1}}
\end{align}
Le variabili $e$, $P$, $t_p$, $\omega$, $K=f(a,e,P,i)$ sono fittate per ogni pianeta.
La velocit\'a radiale della stella \'e
\begin{equation}
v_r(t)=K[\cos{(\omega+\nu(t)}+e\cos{\omega}]+\gamma+d(t-t_0)
\end{equation}
\end{workout}


\section{Altre tecniche osservative}
\begin{workout}[Astrometry]
L'ellisse percorsa dalla stella attorno al baricentro stella pianeta ha ampiezza angolare
\begin{equation}
\alpha=\frac{M_p}{M_p+M_*}a\approx\frac{M_p}{M_*}a=(\frac{M_p}{M_*})(\frac{a}{1AU})(\frac{d}{1pc})\expy{-1}\si{\arcsec}
\end{equation}

Variation in photocentre, radial velocity and total flux
Astrometry: Photonoise $\sigma_{ph}=\frac{\lambda}{4\pi D}\frac{1}{S/N}$
pg 65

\end{workout}

\begin{workout}[Timing: ritardo segnale]
Pg 75
$\tau_p=\frac{1}{c}\frac{a\sin{i}M_p}{M_*}$
\end{workout}

\begin{workout}[Microlensing]
$\frac{M_L}{\msun{}}=0.123\frac{\theta_E^2}{\upvarpi_{rel}}$
Pg 83. fig 5.7. Applications pg 93
(Gould 10: ''fREQUENCY OF SOLAR LIKE SYSTEM AND ICE DAS GIANT BEYOND SNOW LINE FROM MICROLENSING...'')
Survey: Cassan 12, gaudi 12 `'Microlensing Surveys for Exoplanets''
\end{workout}



\begin{workout}[Imaging: star-planet brightness ration]
pg 149
\begin{equation*}
\frac{f_p()}{f_*()}=
\end{equation*}
\end{workout}
\begin{workout}[Imaging: Observational limits]
ground based (pg 158) space based imaging (pg162)
fig 7.10
Survey: bowler 2016
\end{workout}

\chapter{Interazione disco-pianeta}

\section{Horseshoe drag: unsaturated}

HORSESHOEDRAGINTHREE-DIMENSIONALGLOBALLYISOTHERMALDISKS