\documentclass[twoside,11pt,fleqn]{memoir}%%% seminario planetari PPS
%openany

%% rimuove workout errata dalla versione da consegnare
\def\versione{bozza}%%VERSIONE
\def\bozza{bozza}
\def\consegna{consegna}
%%%PENALITIES
%\widowpenalty=10000
%\clubpenalty=10000
%% colors
\usepackage[usenames,dvipsnames]{xcolor}
\definecolor{antiquefuchsia}{rgb}{0.57, 0.36, 0.51}
\definecolor{violetw}{rgb}{0.93, 0.51, 0.93}
\definecolor{Veronica}{rgb}{0.63, 0.36, 0.94}
\definecolor{atomictangerine}{rgb}{1.0, 0.6, 0.4}
\definecolor{darkgray}{rgb}{0.66, 0.66, 0.66}
\definecolor{brightcerulean}{rgb}{0.11, 0.67, 0.84}
\definecolor{cadmiumorange}{rgb}{0.93, 0.53, 0.18}
\definecolor{ochre}{rgb}{0.8, 0.47, 0.13}
\definecolor{midnightblue}{rgb}{0.1, 0.1, 0.44}
\definecolor{grey}{rgb}{0.7, 0.75, 0.71}
\definecolor{bf}{RGB}{88, 86, 88}
\definecolor{bb}{RGB}{177, 177, 177}
%%indice analitico: necessari per printindex
\usepackage{makeidx}
\makeindex
%%%%%%%%%%%%%%%%%%%%%%%%%%%%%%%%%%% importa pacchetti
\usepackage{usepkg}
%%%%%%%%%%%%%%%%%%%%%%%%%%%%%%%%%%%%%
\addtocontents{toc}{\cftpagenumbersoff{part}} %% part senza numero pagina in toc
%%%%%%%%%%%%%%%%%% titletoc, titlesec setting.
%%%%%%%%%%%%%%%%%%      pagestyle
\usepackage{titleT}
\pagestyle{plain}
%%%%%%%%%%%%%%%%%% length
\usepackage{length}
%%%%%%%%%%%% Hyperref package
\usepackage{hyperref}
\hypersetup{
    colorlinks,
    citecolor=black,
    filecolor=black,
    linkcolor=black,
    urlcolor=black
}
%%%%%%%%%%%%%%%%%%%%%%%% Things need to be loaded after hyperref
%\usepackage{cleveref}
%%%%%%%%%%%%%%%%%Geometry package
\usepackage[a4paper,lmargin=80px,rmargin=40px,tmargin=40px,bmargin=20px,nofoot,footskip=20px]{geometry}
%http://tex.stackexchange.com/questions/211248/problem-with-cropmarks-on-geometry-package
%http://www.ctex.org/documents/packages/layout/geometry.pdf
%%%%%%%%%%%%%%%%%%%%%%%%%%%%%%%%%%% Funzioni generali
\usepackage{functions}
%http://tex.stackexchange.com/questions/246/when-should-i-use-input-vs-include
\usepackage{sources}
%%%%%%%%%%%%%%%%%%%%%%%%%%%%%%%%%%% Funzioni per questo file main
\usepackage{mathOp}
\usepackage{LocalF}
\usepackage{ads}
%%%%%%%%%%%%%%%%%%%%%%%%%%%%%%%%%
\raggedbottom %http://tex.stackexchange.com/questions/102084/annoying-paragraph-spacing-issue-with-memoir
%% CAption subcaption figure %%% captionsetup
%\captionsetup[subfigure]{labelformat}
    \makeatletter
    \renewcommand\@memmain@floats{%
      \counterwithout{figure}{section}
      \counterwithout{table}{section}
  \counterwithout{figure}{chapter}
    \counterwithout{figure}{subfigure}
      \counterwithout{table}{chapter}
      \counterwithin{figure}{part}
      \counterwithin{table}{part}
	\renewcommand{\thesubfigure}{\arabic{part}-\arabic{figure}.\alph{subfigure}}
	\renewcommand{\thefigure}{\arabic{part}-\arabic{figure}}
	\renewcommand{\thetable}{\arabic{part}-\arabic{table}}
      }
    \makeatother
%%%%
%%% SOlve page pre chapter at begginning page without number%@ps
\chapterstyle{plain}
%%%
%\makeatletter
%\newcommand{\titolo}{\@title}
%\makeatother
\author{ }
\title{Simulazione popolazioni planetarie}
\date{\today}
%%%
%\outputonly{1}% output in pdf only page number
\begin{document}
\maketitle
\pagenumbering{roman}
\tableofcontents*
\mainmatter
\pagenumbering{arabic}
\aliaspagestyle{chapter}{plain}
\cleartorecto
%%Intro
Il crescente numero di esopianeti osservati ha stimolato lo studio dei processi di formazione dei sistemi planetari tramite modelli di formazione globale: 
I modelli di formazione planetaria fino alla scoperta del primo esopianeta (Queloz 1995) erano vincolati a riprodurre le caratteristiche peculiari del sistema

{\let\clearpage\relax\let\cleardoublepage\relax
\part{Vincoli ai modelli di formazione planetaria}
}
Descrivo le tecniche osservative, le osservabili e effetti di selezione, quindi le propriet\'a degli esopianeti nei diagrammi $(a-M)$, $(a-R)$, $(M-R)$: quali osservabili determinano le diverse tecniche osservative, struttura pianeta (sistema solare, esopianeti).
{\let\clearpage\relax\let\cleardoublepage\relax
\chapter{Sistema solare}
}


\begin{workout}[Sistema solare - andamento densit\'a e composizione]
asteroid mercury mars mass depletion pg 298
\end{workout}

\begin{workout}[et\'a corpi sistema solare (age and chronology)]
helioseismology and meteoritic dating
\end{workout}


\begin{workout}[Struttura interna]
pg 256: interior and atmosphere, 293 solar system, 143 properties of transit
\end{workout}

\begin{workout}[Full packing/spacing]

\end{workout}

\begin{workout}[Modello pianeta gassoso: Giove]

\end{workout}

\begin{workout}[Migrazione II: realistica]\end{workout}
grand tack (produce dinamical shack up) 
eccesso pianeti gioviani a 1au in simulazioni??
\begin{workout}[Noble gas enrichment]

\end{workout}


\begin{workout}[Orbit of terrestrial planets]
Constrain for core accretion model ofgiant planets: dynamical shakeup Nagasawa 05 Thommes08c
\end{workout}

\begin{workout}[Earth post-oligarchic growth]
perryman pg229
\end{workout}


\begin{workout}[Planetesima migration: pluto eccentricity, large population of Kuiper belt object]

\end{workout}



\begin{workout}[Planet obliquity:]
Schlichting sari 07: The effect of Semi-Collisional Accretion on Planetary Spins
Inconsistent with isotropic distro (Tremaine 1991). Randomly directed component of spin angular momentum (Kokubo Ida 07) cause large then observed i,e (Harris Ward 82).
Planetesimal/protoplanet collision would imply stachastic rather than smooth accretion.
Excittion of giant planet spin obliquity if spin axis precession freq pass through resonance with orbital prec freq during migration.
Uranus: tilt due to collision or migration: Bergstralh 91 or Bou\'e Laskar 10.
Non of obliquity of terrestrial planet are belived primordial (secular orbit perturbation)+tidal dissipation.
Effects of inhomogenous infall on obliquity (tremaine 91, bate 10)
\end{workout}



{\let\clearpage\relax\let\cleardoublepage\relax
\chapter{Alcune caratteristiche della popolazione di esopianeti}
}

La popolazione degli esopianeti osservata riflette quella reale tramite gli effetti di selezione dovuti alla probabilit\'a direvilamento della tecnica particolare. Le osservazioni pi\'u numerose sono basate sull'osservazione della velocit\'a lungo la linea di vista dovuto al moto della stella attorno al baricentro del sistema stella pianeta (RV) o differenza nella luminosit\'a dovuto al transito del pianeta tra stella e osservatore.

\begin{figure}[!ht]
\includegraphics[width=0.9\textwidth]{obsMa}\label{fig:Maplot}
\caption{Da \cite{mordasini2018}.}
\end{figure}

\section{Popolazione esopianeti rilevati tramite RV}

Se l'asse z \'e lungo la linea di vista la posizione della stella \'e $z=r(t)\sin{i}\sin{(\omega+\nu)}$ e per la velocit\'a radiale:
\begin{align}
&v_r=\dot{z}=K[\cos{(\omega+\nu)}+e\cos{\omega}]\label{eq:vrsignal}\\
&K=\frac{2\pi}{P}\frac{a_*\sin{i}}{(1-e^2)\expy{1/2}}=(\frac{2\pi G}{P})\expy{1/3}\frac{M_p\sin{i}}{(M_p+M_*)\expy{2/3}}\frac{1}{(1-e^2)\expy{1/1}}
\end{align}
dove $K$ \'e la semi-ampiezza.

\begin{workout}[Simulated radialo detection limit]
Perrymann pg 34 fig 2.25 d
paragrafo detectability and selection effects pg 14 perryman
\end{workout}

\begin{workout}[Radial velocity: dtectability and selection effects (pg 14)]
fig 2.6,fig 2.18
\end{workout}


\begin{workout}[Distribuzioni elementi orbitali: periodo e massa]

La distribuzione di $M_p\sin{i}$ osservata \'e mostrata in figura \ref{fig:RVmpdistro}, la distribuzione di periodo orbitale \ref{fig:PdistroM30}/\ref{fig:Pdistrom30} perpianeti di massa maggiore/minore di $30\mearth{}$, la distribuzione dei pianeti in funzione della metallicit\'a stellare in \ref{fig:freqZstar}.

\begin{figure}[!ht]
\begin{subfigure}[b]{0.47\textwidth}
\includegraphics[width=0.9\textwidth]{freqvsM}
\caption{Diagramma $(M_p,a$ dei pianeti extrasolari osservati (The extrasolar planet encyclopedia -2017): evidenziati in rosso i pianeti rivelati tramite RV, celeste tramite transito, rosa tramite imaging e verde tramite microlensing. Da \cite{howard2012planet}.}\label{fig:Mdistro}
\end{subfigure}
~
\begin{subfigure}[b]{0.47\textwidth}
\includegraphics[width=0.9\textwidth]{PfreqvsFeH}\label{fig:freqZstar}
\caption{Da \cite{mayor2011harps}}
\end{subfigure}
\end{figure}

\begin{figure}[!ht]
\begin{subfigure}[b]{0.47\textwidth}
\centering
\includegraphics[width=0.9\textwidth]{freqvsPgiant}
\caption{Da \cite{mayor2011harps}}\label{fig:PdistroM30}
\end{subfigure}
~
\begin{subfigure}[b]{0.47\textwidth}
\centering
\includegraphics[width=0.9\textwidth]{freqvsPlowM}\label{fig:Pdistrom30}
\caption{Da \cite{mayor2011harps}}
\end{subfigure}
\end{figure}

\end{workout}

\begin{workout}[multplanet system: mean motion resonances, orbital spacing]
wright09: ten multiplanet sysytem and systematic
Winn, Fabricky 15
fig 2.30 pg 39 Perrymann
\end{workout}

%\begin{figure}[]
%\begin{subfigure}[b]{0.47\textwidth}
%\centering
%\includegraphics[keepaspectratio,width=0.95\textwidth]{midlmodes}
%\caption{I picchi della densit\'a spettrale si dispongono su creste in cui \'e concentrata la potenza in accordo al modello. Determinata usando i primi 144 giorni di osservazione di MDI con $l\leq300$. Da \cite{chr02helioseismology}.}\label{fig:midlmodes}
%\end{subfigure}
%~
%\begin{subfigure}[b]{0.5\textwidth}
%\centering
%\includegraphics[keepaspectratio,width=0.95\textwidth]{nrmodesLAWE}
%\caption{Modi adiabatici calcolati sulla base di un modello solare. Da \cite{chr02helioseismology}.}\label{fig:nrmodesLAWE}
%\end{subfigure}
%\end{figure}

\begin{workout}[Radial velocity: keplerian observables (Appendice RV)]
La coordinata z della stella $z=r(t)\sin{i}\sin{(\omega+\nu)}$ e per la velocit\'a radiale
\begin{align}
&v_r=\dot{z}=K[\cos{(\omega+\nu)}+e\cos{\omega}]\label{eq:vrsignal}\\
&K=\frac{2\pi}{P}\frac{a_*\sin{i}}{(1-e^2)\expy{1/2}}=(\frac{2\pi G}{P})\expy{1/3}\frac{M_p\sin{i}}{(M_p+M_*)\expy{2/3}}\frac{1}{(1-e^2)\expy{1/1}}
\end{align}
Le osservabili $e$, $P$, $t_p$, $\omega$, $K=f(a,e,P,i)$ sono fittate per ogni pianeta.
La velocit\'a radiale della stella \'e
\begin{equation}
v_r(t)=K[\cos{(\omega+\nu(t)}+e\cos{\omega}]+\gamma+d(t-t_0)
\end{equation}
Per sistemi multipli consideriamo la sovrapposizione lineare di $n_p$ termini \eqref{eq:vrsignal}: viene sottratto il segnale Kepleriano dominante fino a che resta solo rumore.
\end{workout}

\begin{workout}[Radial velocity: keplerian observables (Appendix RV)]
La coordinata z della stella $z=r(t)\sin{i}\sin{(\omega+\nu)}$ e per la velocit\'a radiale
\begin{align}
&v_r=\dot{z}=K[\cos{(\omega+\nu)}+e\cos{\omega}]\label{eq:vrsignal}\\
&K=\frac{2\pi}{P}\frac{a_*\sin{i}}{(1-e^2)\expy{1/2}}=(\frac{2\pi G}{P})\expy{1/3}\frac{M_p\sin{i}}{(M_p+M_*)\expy{2/3}}\frac{1}{(1-e^2)\expy{1/1}}
\end{align}
Le osservabili $e$, $P$, $t_p$, $\omega$, $K=f(a,e,P,i)$ sono fittate per ogni pianeta.
La velocit\'a radiale della stella \'e
\begin{equation}
v_r(t)=K[\cos{(\omega+\nu(t)}+e\cos{\omega}]+\gamma+d(t-t_0)
\end{equation}
\end{workout}

Per sistemi multipli consideriamo la sovrapposizione lineare di $n_p$ termini \eqref{eq:vrsignal}: viene sottratto il segnale Kepleriano dominante fino a che resta solo rumore.

\begin{workout}[Radial velocity: dtectability and selection effects (pg 14)]
fig 2.6,fig 2.18
\end{workout}


\section{Popolazione esopianeti rilevati tramite transito}

Le osservabili sono la differenza di luminosit\'a durante il transito, la durata totale e di massima occultazione del transito, e il periodo: da questi \'e possibile ricavare $R_p$, a, i, $R_*$, e facendo uso della relazione massa raggio appropriata per la fase evolutiva della stella $M_*$

La probabilit\'a che l'orbita di un pianeta sia allineata con l'osservatore in maniera da avere un transito, per orbite circolari ed eccentriche, \'e:
\begin{align*}
&p=\frac{R_*}{a}=0.005(\frac{R_*}{\rsun{}})(\frac{a}{1AU})\expy{-1}\\
&p=(\frac{R_*\pm R_p}{a})(\frac{1}{1-e^2})
\end{align*}
La maggiore probabilit\'a \'e compensata approssimativamente dalla minore durata del transito.

\begin{workout}[Radial velocity: dectability and selection effects]
Considerando un numero totale di osservazioni $N_o$ durante il transito si hanno $N_t\approx N_o\frac{R_*}{\pi a}$; il rapport segnale rumore \'e $S/N=\sqrt{N_t}\frac{\delta}{\sigma}$, $\sigma$ precisione fotometrica, $\sigma\propto\frac{1}{\sqrt{N_{ph}}}\propto d$. Per fissato S/N e tipo stellare il numero di pianeti rivelati varia come
\begin{equation}
V_p*P\propto d^3*\frac{R_*}{a}\propto R_p^6/P\expy{\frac{5}{3}}
\end{equation}
\end{workout}


\begin{workout}[transit survey Kepler: coughlin 2016]
PLANETARY CANDIDATES OBSERVED BY KEPLER. VII. THE FIRST FULLY UNIFORM CATALOG BASED ON THE ENTIRE 48 MONTH DATASET (Q1–Q17 DR24
\end{workout}


\begin{workout}[transiti refs: osservabilie probabilit\'a di rilevamento]
pg103
pg 117, eccentric orbit pg 121-122
From space: presence of structure on stellar surface Perryman 3.4, Eriksson Lindegren 07; simulation of stellar jitter: Svensson Ludwig 05, Ludwig 06.
Probability of randomlyoriented planet on circular/eccentric orbit (Borucki Summers 84/Barnes 07, Burke 08,  Seagroves 03, Kane von Braun 08):
\end{workout}

\begin{workout}[Transiti: Propriet\'a dei pianeti a transito]
pg 143, fig 6.33, 6.34, 6.35
Mass-radius relation: chabrier 09
\end{workout}


\begin{workout}[Atmosphere and starting conditions]
fig 11.12
\end{workout}


\begin{workout}[M-R diagram]
orbital migration 10.8 Perryman, tidal effect 10.9, 
fig11.2, 11,4 (EOS), 11.7, 11.8 (M-R)
Ternary diagram 11.9, M-R realtion
Fig 6.33/34/35: mass radius realtion

pg 144: theoretical model - La posizione di un pianeta nel diagramma M-R fornisce indicazione della composizione
fig 6.35
\end{workout}


%\section{Struttura interna, composizione/diagramma $(M,R)$}

\cleartorecto

{\let\clearpage\relax\let\cleardoublepage\relax
\part{Formazione sistemi planetari: descrizione modello core accretion}\label{part:CAdesc}
}
\begin{workout}[Intro a modelli core accretion]
Nei modelli globali la formazione planetaria \'e simulata partendo dalla fase finale di accrescimento di massa sulla stella centrale (per semplicit\'a considero stelle di $1\msun{}$ singole): il collasso di nube molecolare produce strutture appiattite la cui evoluzione \'e determinata dal trasporto di momento angolare verso l'esterno, l'interazione con l'oggetto centrale (campi magnetici vento stellare) e con l'ambiente circostante.
L'ipotesi su cui si basano le simulazioni considerata \'e che la componente polverosa formi corpi pi\u massicci fino a masse di frazioni di masse terrestri  ed infine accrescere gas: lo scenario di core accretion (CA).
\end{workout}

%\deplaced{Accretion disk}{dopo sistema solare}
%\abortedchapter{Schema di accrescimento componente solida}{Parte descrizione modelli globali: distro planetesimi e posizione iniziale embrioni}


{\let\clearpage\relax\let\cleardoublepage\relax
\chapter{Evoluzione struttura planetaria: accrescimento di gas.}\label{chap:gasaccretion}
}% e fase isolata

\begin{wrapfigure}[15]{l}{0.5\textwidth}
\includegraphics[trim={0cm 2cm 0 0},clip, keepaspectratio,width=0.48\textwidth]{massenvvscore}
\caption{Massa planetaria in funzione della massa del core. Da \cite{alibert2005models}.}\label{fig:massenvvscore}
\end{wrapfigure}

Se il protopianeta raggiunge un valore critico di massa, dipendente debolmente da luminosit\'a generata da accrescimento dei pianetesimi e opacit\'a, per mantenere equilibrio idrostatico l'inviluppo gassoso del pianeta si contrae su tempi scala di Kelvin-Helmholtz. In questa seconda fase l'accresscimento di massa diviene cos\'i rapido da esaurire il gas disponibile nella regione: curva verticale di figura (\ref{fig:massenvvscore}). L'accrescimento di gas has come limite superiore il flusso di gas dovuto a evoluzione viscosa del disco.

Da modelli numerici risulta
\begin{equation}
M_c^{crit}=10\mearth{}(\frac{\dot{M}_c}{\num{e-7}\mearth{}\si{\per\year}})\expy{q}(\frac{\kappa}{\SI{0.1}{\square\meter\per\kilo\gram}})\expy{s}
\end{equation}
con $q,s\approx0.2-0.3$ (\cite{ikoma2000formation}).

\vspace{2cm}

\section{Accrescimento limitato da velocit\'a di raffreddamento}

Seguendo (\cite{mordasini2012characterization}) la struttura del pianeta \'e determinata integrando le equazioni di conservazione di massa, momento e l'equazione del trasporto di energia:
\begin{align}
&\TDy{r}{m}=4\pi r^2\rho\\
%&\TDy{r}{l}=0\\
&\TDy{r}{P}=-\frac{Gm}{r^2}\rho\\
&\TDy{r}{T}=\frac{T}{P}\TDy{r}{P}\nabla(T,P)\\
&\nabla(T,P)=\TDly{P}{T}=\min{(\nad{},\nrad{})}
\end{align}
dove $\nrad{}$ e $\nad{}$ indicano il gradiente radiativo e adiabatico.

La luminosit\'a del pianeta \'e determinata tramite
\begin{equation}
E_t=E_g+E_i=\int_0^M\frac{Gm}{r}\,dm+\int_{M_z}^Mu\,dm=-\xi\frac{GM^2}{2R}
\end{equation}
che sostituita nell'equazione di conservazione dell'energia da:
\begin{equation}
-\TDof{t}E_t=L=L_M+L_R+L_{\xi}=\xi\frac{GM}{R}\dot{M}-\xi\frac{GM^2}{2R^2}\dot{R}+\frac{GM^2}{2R}\dot{\xi}
\end{equation}
con $\dot{M}=\dot{M}_Z+\dot{M}_{XY}$.

Ho indicato la massa di gas legata al core con
\begin{equation}
M_{XY}=4\pi\int_{R_c}^R\rho(r')r'^2\,dr'
\end{equation}
con $R_C$ \'e il raggio del core.

Condizioni al bordo:
\begin{align}
&R=\frac{R_B}{1+R_B/(k_lR_H )},\ P=P_{neb}\\
&\tau=\max{(\rho_{neb}\kappa_{neb}R),2/3)},\ T_i^4=\frac{3\tau L_{int}}{8\pi\sigma R^2}\\
&T^4=T_{neb}^4+T_{int}^4,\ L(R)=L_{int}
\end{align}
$k_{liss}=3-4$, quindi $R_p\approx \min{(R_B,k_{liss}R_H)}$ e introduco il raggio di Bondi
\begin{equation}
R_B=G\frac{M_c}{c_0^2}\approx\SI{4e10}{\cm}a(AU)\expy{1/2}\frac{M_c}{\mearth{}}
\end{equation}
definito come raggio in cui energia termica e potenziale gravitazionale si equivalgono.%il core perturba la pressione del gas del disco.

\begin{workout}[envelope mass as function of core mass]
Armitage 17 eq 232 (`'lecture nite on formation and early evolution of PS)
\begin{equation}
M_{env}\approx\int_{R_c}^{R_o}4\pi r^2\rho\,dr\propto\frac{\sigma}{\kappa_RL}(\frac{\mu m_pGM_t}{4k_b})^4\ln{\frac{R_o}{R_c}}
\end{equation}
\end{workout}

\begin{workout}[Rfes per espressione raggio bondi]

\end{workout}


\begin{workout}[Hydrostatic equilibrium hypothesis]
Characterization of exoplanets from their formation I (eq 10)
\end{workout}


\begin{workout}[Rfes espressione accrescimento gas]
Rate di accrescimento limitato dalla velocit\'a di raffreddamento:
\begin{equation}
\dot{M}_{XY}\propto\ \frac{M_p}{M_*}<(H_P/R_p)^3/\sqrt{3}
\end{equation}
quindi la massa di gas aumenta esponenzialmente a partire da $M_c\approx10\mearth{}$.
\end{workout}

\section{Accrescimento limitato dalla disponibilit\'a di gas del disco}

Se $H\approx R$ il pianeta perturba in maniera non trascurabile il disco (il rate di accrescimento di gas \'e maggiore di quello fornito dal disco). Il raggio del pianeta \'e determinato dalle condizioni al bordo per materia accresciuta tramite free-fall da $R_H$ a $R$:
\begin{align}
&\dot{M}_{XY}=\dot{M}_{XY,max},\ v_{ff}^2=2GM(\frac{1}{R}-\frac{1}{R_H})\\
&P=P_{neb}+\frac{\dot{M}_{XY}}{4\pi r^2}v_{ff}+\frac{2g}{3\kappa},\ \tau=\max{(\rho_{neb}\kappa_{neb}R,2/3)}\\
&T_{int}^4=\frac{3\tau L_{int}}{8\pi\sigma R^2},\ T^4=(1-A)T_{neb}^4+T_{int}^4
\end{align}

In questa fase la velocit\'a di accrescimento di gas \'e determinata dall'evoluzione viscosa del disco:
\begin{equation}
\dot{M}_{e,visc}=f_{hyd}3\pi\nu\Sigma_g
\end{equation}
$f_{hyd}\approx0.9$ valore determinato da simulazioni idrodinamiche (\cite{lubow1999disk}).

\begin{workout}[Bondi accretion rate]
Unperturbed viscous flow
\begin{equation}
\dot{M}_{e,B}\approx\frac{\Sigma_g}{H}(\frac{R_H}{3})^3\Omega
\end{equation}
\end{workout}

\begin{workout}[Detached phase accretion rate]
Characterization of exoplanets from their formation pg 8
\end{workout}

\begin{workout}[Wien displacement]
$\lambda_{max}T\approx \SI{3e-3}{\meter\kelvin}$
\end{workout}

\begin{workout}[Critical core mass]
From toward deterministic
\begin{equation}
M_{c,crit}\approx10(\frac{\dot{M}_c}{\num{e-6}\mearth{}\si{\per\year}})\expy{0.2-0.3}(\frac{\kappa}{\SI{1}{\square\cm\per\gram}})\expy{0.2-0.3}\mearth{}
\end{equation}•
\end{workout}

\begin{workout}[gas accretion refs]
Lissauer 09: Models of Jupiter’s growth incorporating thermal and hydrodynamic constraints
Rafikov 10: ''Constraint on giant planet production by core accretion''
Rafikov 04 Atmospheres of protoplanetary cores: critical mass for nucleated instability.
Refs: Planet formation models: the interplay with the planetesimal disc (Fortier 2013), Characterization of exoplanets from their formation I. Models of combined planet formation and evolution (Mordasini 12)
\end{workout}

\begin{workout}[Planet-Disk exchange in hydrodynamic manner]
Ormel 15/ Cimerman 17
\end{workout}

%\section{Fase isolata}

\begin{workout}[Fase isolata: fonti energia]
fonti energia: tidal heating radiogeninc heat, star flux
\end{workout}


{\let\clearpage\relax\let\cleardoublepage\relax
\chapter{Evoluzione orbite proto-pianeti: migrazione planetaria.}
}

\begin{wrapfigure}[10]{l}{0.5\textwidth}
\includegraphics[trim={0cm 11cm 0 0},clip, keepaspectratio,width=0.48\textwidth]{pdres}
\caption{Simulazione evoluzione orbite planetarie in disco di accrescimento fino a cattura in risonanza $2:1$: le regioni pi\'u dense sono in rosso. Da \cite{kley2012planet}.}\label{fig:pdres}
\end{wrapfigure}

L'osservazione di una sottopopolazione di esopianeti giganti su orbite strette, per cui sembra improbabile una formazione in loco, e la presenza di sistemi multipli in risonanza sono indicatori di evoluzione orbitale, d'altra parte nel Sistema solare fenomeni di migrazione fornisco una spiegazione all'orbita di plutone, alla presenza di numerosi oggetti della fascia di Kuiper in risonanza $3:2$ con Nettuno e al periodo di collisioni intense testimoniato da craterizzazione (Late heavy bombardment).

\vspace{1.5cm}

Fenomeni che possono dar luogo a migrazione planetaria:
\begin{itemize}
\item Interazione con disco proto-planetario. Considerando le perturbazioni lineari nella densit\'a del disco prodotte dal potenziale del pianeta si determina la risultante dei momenti torcenti $\Gamma$ dovuti alle risonanze di Lindblad e alla regione di corotazione. Velocit\'a edirezione della migrazione dipendono dalle caratteristiche del disco, massa del pianeta e moto relativo pianeta-gas.


\begin{errata}[Elenco schematicamente alcuni risultati da \cite{armitage2007lecture} e \cite{crida2006planetary}]
 La migrazione di tipo I ha tempi caratteristici
\begin{equation}
\tau_I\propto\frac{M_p}{\Gamma}\propto M_p\expy{-1}
\end{equation}
per pianeta $5\mearth{}$ a \SI{5}{\astronomicalunit} si ha $\tau_I=\SI{0.5}{\mega\year}$.

La migrazione di tipo II \'e caratterizzata da formazione di un gap attorno all'orbita del pianeta: la transizione tra migrazione I e II avviene se $R_H>H$ e momento torcente dovuto alla viscosit\'a del disco \'e minore del momento esercitato dal pianeta sul disco. La velocit\'a di migrazione di tipo II \'e determinata dall'evoluzione viscosa del disco:
\begin{equation}
\dot{a}_p=-\frac{3}{2}\frac{\nu}{r}
\end{equation}
Il momento esercitato sul pianeta dalla regione di corotazione pu\'o dar luogo a feedback positivo su pianeta di massa intermedia con velocit\'a radiale $\dot{a}_p$.
\end{errata}

\item Interazione con disco residuo di planetesimi

\item Interazione tra due o pi\'u pianeti giganti

\item Interazione con stella in sistema di stelle binarie

\item Interazione mareale con la stella

\end{itemize}

%{\let\clearpage\relax\let\cleardoublepage\relax
%\chapter{N-body interactions inside proto disk}
%}


\begin{workout}[Long-term evolution: laplace equations]
The HARPS search for southern extra-solar planets-XXVIII. Up to seven planets orbiting HD 10180: probing the architecture of low-mass planetary systems
\end{workout}

\cleartorecto

{\let\clearpage\relax\let\cleardoublepage\relax
\part{Popolazioni planetarie sintetiche e confronto con osservazioni}
}
Per costruire un modello di formazione globale \'e necessario introdurre una descrizione semplificata dei processi fisici che portano alla formazione di un pianeta; quindi generare una popolazione planetaria a partire da distribuzione di condizioni iniziali del disco di accrescimento, ricavata dalle osservazioni che riproduca la distribuzione di caratteristiche osservate, tenendo conto dei bias osservativi.

{\let\clearpage\relax\let\cleardoublepage\relax
\chapter{Simulazione polazioni planetarie}
}

\section{Distribuzione iniziale planetesimi}
Assumendo conversione completa di polvere in planetesimi, la densit\'a superficiale di planetesimi \'e
\begin{equation}
\Sigma_p(t=0,r)=f_{dg}\eta_{ice}\Sigma_g(t=0,r)
\end{equation}

\begin{workout}[Initial planetesimal distro]
Dust converted early everywhere fully-efficient (mordasini18: pg 12),Thommes pg8)
\begin{align*}
&\Sigma_p(t=0,r)=f_{dg}\eta_{ice}\Sigma_g(t=0,r)\\
&\dot{\Sigma}_p(r)=-\frac{1}{2\pi aB_LR_H}\dot{M}_c
\end{align*}
Icelines Mordasini 1141: inside where T exceeds sublimation
\end{workout}

\section{Accrescimento dei protopianeti e gas}

La massa dei core varia secondo
\begin{align}
&\dot{M}_c=\Omega\Sigma_pR^2_{capt}F_G\\
&\dot{\Sigma}_p(r)=-\frac{1}{2\pi aB_LR_H}\dot{M}_c
\end{align}

Si risolvono numericamente le equazioni di conservazione di massa, equilibrio idrostatico, conservazione/trasporto energia 1D

\begin{workout}[Attached phase: boundary conditions]
\begin{align}
&P_{pl}=P_{Neb}\\
&T_{Pl}=T_{Neb}\\
&R_{Pl}=Min(R_H,R_H)
\end{align}
\end{workout}


\begin{workout}[Detached phase: transition condition and boundary condition]
Transition attached/detched phase $\approx10\mearth{}$.
Accretion shock for free-falling materials from Hill radius (more realistic circumplanetary disk: Papailoizou Nelson 05)
\begin{align}
&\dot{M}_{XY}^{max}\\
&v_{ff}^2=2GM(\frac{1}{R}-\frac{1}{R_H})\\
&P=P_{neb}+\frac{\dot{M}_{XY}}{4\pi R^2}v_{ff}+\frac{2g}{3\kappa}\\
&\tau=max(\rho_{neb}\kappa_{neb}R,2/3)\\
&T^4_{int}=\frac{3\tau L_{int}}{8\pi\sigma R^2},\ T^4=(1-A)T_{neb}^4+T_{int}^4
\end{align}
$R\approx1.5-5\rjupiter{}$ depending on entropy: THE PLANETARY ACCRETION SHOCK:I. FRAMEWORK FOR RADIATION-HYDRODYNAMICAL SIMULATIONS AND FIRST RESULTS (m17), Characterization of exoplanets from their formation III: The statistics of planetary luminosities (m17)
Bondi accretion rate: $\dot{M}_{e, Bondi}\approx\frac{\Sigma}{H}(R_H/3)^3\Omega$ or viscous accretion rate $\dot{M}_{e, visc}\approx f_{lub}3\pi\nu\Sigma_g$
\end{workout}

\begin{workout}[Higher mass gap formation reduces accretion rate]
\begin{equation}
f_{va04}=1.668(\frac{M_p}{\mjupiter{}})\expy{1/3}\exp{-\frac{M_p}{1.5\mjupiter{}}}+0.04
\end{equation}
\end{workout}


{\let\clearpage\relax\let\cleardoublepage\relax
\chapter{Funzione di massa sistemi planetari}
}

{Impact of planet migration model on planetary populations: effects of saturation, cooling and stellar irradiation}
Outward migration helps some planets to become massive, accumulation zone at certain semiaxis, at what mass corotation saturate?
TOdo: mifration model

\cleartorecto



{\let\clearpage\relax\let\cleardoublepage\relax
\backmatter
}
%\newgeometry{margin=60px,tmargin=20px}%%APPENDICE
\appendix
\part{Appendice}
\section{Altre tecniche osservative}
\begin{workout}[Astrometry]
L'ellisse percorsa dalla stella attorno al baricentro stella pianeta ha ampiezza angolare
\begin{equation}
\alpha=\frac{M_p}{M_p+M_*}a\approx\frac{M_p}{M_*}a=(\frac{M_p}{M_*})(\frac{a}{1AU})(\frac{d}{1pc})\expy{-1}\si{\arcsec}
\end{equation}

Variation in photocentre, radial velocity and total flux
Astrometry: Photonoise $\sigma_{ph}=\frac{\lambda}{4\pi D}\frac{1}{S/N}$
pg 65

\end{workout}

\begin{workout}[Timing: ritardo segnale]
Pg 75
$\tau_p=\frac{1}{c}\frac{a\sin{i}M_p}{M_*}$
\end{workout}

\begin{workout}[Microlensing]
$\frac{M_L}{\msun{}}=0.123\frac{\theta_E^2}{\upvarpi_{rel}}$
Pg 83. fig 5.7. Applications pg 93
(Gould 10: ''fREQUENCY OF SOLAR LIKE SYSTEM AND ICE DAS GIANT BEYOND SNOW LINE FROM MICROLENSING...'')
Survey: Cassan 12, gaudi 12 `'Microlensing Surveys for Exoplanets''
\end{workout}



\begin{workout}[Imaging: star-planet brightness ration]
pg 149
\begin{equation*}
\frac{f_p()}{f_*()}=
\end{equation*}
\end{workout}
\begin{workout}[Imaging: Observational limits]
ground based (pg 158) space based imaging (pg162)
fig 7.10
Survey: bowler 2016
\end{workout}
%\restoregeometry
\printbibliography
%\listoffigures
\ifx\versione\bozza
\woc
\erratac
\fi

\end{document}
