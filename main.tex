\documentclass[twoside,11pt,fleqn]{memoir}%%% seminario planetari PPS
%openany

%% rimuove workout errata dalla versione da consegnare
\def\versione{bozza}%%VERSIONE
\def\bozza{bozza}
\def\consegna{consegna}
%%%PENALITIES
%\widowpenalty=10000
%\clubpenalty=10000
%% colors
\usepackage[usenames,dvipsnames]{xcolor}
\definecolor{antiquefuchsia}{rgb}{0.57, 0.36, 0.51}
\definecolor{violetw}{rgb}{0.93, 0.51, 0.93}
\definecolor{Veronica}{rgb}{0.63, 0.36, 0.94}
\definecolor{atomictangerine}{rgb}{1.0, 0.6, 0.4}
\definecolor{darkgray}{rgb}{0.66, 0.66, 0.66}
\definecolor{brightcerulean}{rgb}{0.11, 0.67, 0.84}
\definecolor{cadmiumorange}{rgb}{0.93, 0.53, 0.18}
\definecolor{ochre}{rgb}{0.8, 0.47, 0.13}
\definecolor{midnightblue}{rgb}{0.1, 0.1, 0.44}
\definecolor{grey}{rgb}{0.7, 0.75, 0.71}
\definecolor{bf}{RGB}{88, 86, 88}
\definecolor{bb}{RGB}{177, 177, 177}
%%indice analitico: necessari per printindex
\usepackage{makeidx}
\makeindex
%%%%%%%%%%%%%%%%%%%%%%%%%%%%%%%%%%% importa pacchetti
\usepackage{usepkg}
%%%%%%%%%%%%%%%%%%%%%%%%%%%%%%%%%%%%%
\addtocontents{toc}{\cftpagenumbersoff{part}} %% part senza numero pagina in toc
%%%%%%%%%%%%%%%%%% titletoc, titlesec setting.
%%%%%%%%%%%%%%%%%%      pagestyle
\usepackage{titleT}
\pagestyle{plain}
%%%%%%%%%%%%%%%%%% length
\usepackage{length}
%%%%%%%%%%%% Hyperref package
\usepackage{hyperref}
\hypersetup{
    colorlinks,
    citecolor=black,
    filecolor=black,
    linkcolor=black,
    urlcolor=black
}
%%%%%%%%%%%%%%%%%%%%%%%% Things need to be loaded after hyperref
%\usepackage{cleveref}
%%%%%%%%%%%%%%%%%Geometry package
\usepackage[a4paper,lmargin=80px,rmargin=40px,tmargin=40px,bmargin=20px,nofoot,footskip=20px]{geometry}
%http://tex.stackexchange.com/questions/211248/problem-with-cropmarks-on-geometry-package
%http://www.ctex.org/documents/packages/layout/geometry.pdf
%%%%%%%%%%%%%%%%%%%%%%%%%%%%%%%%%%% Funzioni generali
\usepackage{functions}
%http://tex.stackexchange.com/questions/246/when-should-i-use-input-vs-include
\usepackage{sources}
%%%%%%%%%%%%%%%%%%%%%%%%%%%%%%%%%%% Funzioni per questo file main
\usepackage{mathOp}
\usepackage{LocalF}
\usepackage{ads}
%%%%%%%%%%%%%%%%%%%%%%%%%%%%%%%%%
\raggedbottom %http://tex.stackexchange.com/questions/102084/annoying-paragraph-spacing-issue-with-memoir
%% CAption subcaption figure %%% captionsetup
%\captionsetup[subfigure]{labelformat}
    \makeatletter
    \renewcommand\@memmain@floats{%
      \counterwithout{figure}{section}
      \counterwithout{table}{section}
  \counterwithout{figure}{chapter}
    \counterwithout{figure}{subfigure}
      \counterwithout{table}{chapter}
      \counterwithin{figure}{part}
      \counterwithin{table}{part}
	\renewcommand{\thesubfigure}{\arabic{part}-\arabic{figure}.\alph{subfigure}}
	\renewcommand{\thefigure}{\arabic{part}-\arabic{figure}}
	\renewcommand{\thetable}{\arabic{part}-\arabic{table}}
      }
    \makeatother
%%%%
%%% SOlve page pre chapter at begginning page without number%@ps
\chapterstyle{plain}
%%%
\makeatletter
\newcommand{\titolo}{\@title}
\makeatother
\author{ }
\title{Studio delle oscillazioni solari}
\date{\today}
%%%
%\outputonly{1}% output in pdf only page number
\begin{document}
\pagenumbering{roman}
\tableofcontents*
\mainmatter
\pagenumbering{arabic}
\aliaspagestyle{chapter}{plain}
\cleartorecto
%%Intro
Il crescente numero di esopianeti osservati ha stimolato lo studio dei processi di formazione dei sistemi planetari tramite modelli di formazione globale:

{\let\clearpage\relax\let\cleardoublepage\relax
\part{Proprieta esopianeti osservati}
}
Descrivo le tecniche osservative, le osservabili e effetti di selezione, quindi le propriet\'a degli esopianeti nei diagrammi $(a-M)$, $(a-R)$, $(M-R)$.
\section{Velocit\'a radiale}
\begin{workout}{Radial velocity: keplerian observables}
La coordinata z della stella $z=r(t)\sin{i}\sin{(\omega+\nu)}$ e per la velocit\'a radiale
\begin{align}
&v_r=\dot{z}=K[\cos{(\omega+\nu)}+e\cos{\omega}]\label{eq:vrsignal}\\
&K=\frac{2\pi}{P}\frac{a_*\sin{i}}{(1-e^2)\expy{1/2}}=(\frac{2\pi G}{P})\expy{1/3}\frac{M_p\sin{i}}{(M_p+M_*)\expy{2/3}}\frac{1}{(1-e^2)\expy{1/1}}
\end{align}

Le osservabili $e$, $P$, $t_p$, $\omega$, $K=f(a,e,P,i)$ sono fittate per ogni pianeta.
La velocit\'a radiale della stella \'e
\begin{equation}
v_r(t)=K[\cos{(\omega+\nu(t)}+e\cos{\omega}]+\gamma+d(t-t_0)
\end{equation}
\end{workout}
Per sistemi multipli consideriamo la sovrapposizione lineare di $n_p$ termini \eqref{eq:vrsignal}: viene sottratto il segnale Kepleriano dominante fino a che resta solo rumore.
\begin{workout}{Radial velocity: dtectability and selection effects (pg 14)}
fig 2.6,fig 2.18
\end{workout}
\begin{workout}{properties of observed planeta}
fig 2.26pag 35
\end{workout}
\section{Astrometry}
L'ellisse percorsa dalla stella attorno al baricentro stella pianeta ha ampiezza angolare
\begin{equation}
\alpha=\frac{M_p}{M_p+M_*}a\approx\frac{M_p}{M_*}a=(\frac{M_p}{M_*})(\frac{a}{1AU})(\frac{d}{1pc})\expy{-1}\si{\arcsec}
\end{equation}

\begin{workout}{Variation in photocentre, radial velocity and total flux}
Astrometry: Photonoise $\sigma_{ph}=\frac{\lambda}{4\pi D}\frac{1}{S/N}$
pg 65
\end{workout}

\section{Timing}
\begin{workout}{Timing: ritardo segnale}
Pg 75
$\tau_p=\frac{1}{c}\frac{a\sin{i}M_p}{M_*}$
\end{workout}
\section{Microlensing}
\begin{workout}{Microlensing}
$\frac{M_L}{\msun{}}=0.123\frac{\theta_E^2}{\upvarpi_{rel}}$
Pg 83. fig 5.7. Applications pg 93
(Gould 10: ''fREQUENCY OF SOLAR LIKE SYSTEM AND ICE DAS GIANT BEYOND SNOW LINE FROM MICROLENSING...'')
\end{workout}
\section{Transiti}
pg 103
\begin{workout}{transiti: osservabilie probabilit\'a di rilevamento}
pg 117, eccentric orbit pg 121-122
\end{workout}
\begin{workout}{Transiti: Propriet\'a dei pianeti a transito}
pg 143, fig 6.33, 6.34, 6.35
Mass-radius relation: chabrier 09
\end{workout}

\section{Imaging}
pg 149
\begin{workout}{Imaging: star-planet brightness ration}
\begin{equation*}
\frac{f_p()}{f_*()}=
\end{equation*}
\end{workout}
\begin{workout}{Imaging: Observational limits}
ground based (pg 158) space based imaging (pg162)
fig 7.10
\end{workout}

\section{Struttura interna}

pg 256 
\begin{workout}{M-R diagram}
fig11.2, 11,4 (EOS), 11.7, 11.8 (M-R)
Ternary diagram 11.9, M-R realtion
\end{workout}

\begin{workout}{Atmosphere and starting conditions}
fig 11.12
\end{workout}

\cleartorecto

\part{Modelli di formazione globali}
\chapter{Accretion disk model}

\chapter{Formazione protopianeti}

\chapter{Migrazione}
\cleartorecto

{\let\clearpage\relax\let\cleardoublepage\relax
\part{Conclusioni}
}
Cosa penso di PPS.
{\let\clearpage\relax\let\cleardoublepage\relax
\backmatter
}
%\newgeometry{margin=60px,tmargin=20px}%%APPENDICE
%\appendix
%\part{Appendice}
%\subfile{appendix}
%\restoregeometry
\printbibliography
%\listoffigures
\ifx\versione\bozza
\woc
\erratac
\fi
\end{document}
